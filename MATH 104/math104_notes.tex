\documentclass[openany]{book}
% !TeX TXS-program:compile = txs:///pdflatex/[--shell-escape]
\usepackage{macros}
\usepackage{notes}

%% PICTURES DIRECTORY %%
\graphicspath{{C:/Users/Michael/Pictures/}}

%% RENEW TITLE PAGE %%
\renewcommand{\mytitle}[2]{%
	\title{#1}
	\author{Michael Pham}
	\date{#2}
	\maketitle
	\newpage
	\mytoc
	\newpage
}

% REDEFINING CHAPTER FORMATTING %
\newif\iftoc\titleformat{\chapter}[display]{\cabin}{}{2in}{
	\raggedleft
	\iftoc
	\vspace{2in}
	\else
	{\LARGE\textsc{Week}~{\cantarell\thechapter}} \\
	\fi
	\Huge\scshape\bfseries
}[\vspace{-20pt}\rule{\textwidth}{0.1pt}\vspace{0.0in}]
\titlespacing{\chapter}{0pt}{
	\iftoc
	-100pt+1in
	\else
	-130pt+1in
	\fi
}{0pt}

\begin{document}
\mytitle{Math 104: Real Analysis}{Spring 2025}

\chapter{An Introduction}
\section{Lecture -- 1/22/2025}
\subsection{Administrivia}
\begin{miscbox}{Basic Information}
	\begin{itemize}
		\item \textbf{Office Hours}: Evans 1067, MWF after class.
		\item \textbf{Textbook}: Elementary Analysis by Ross.
		\begin{itemize}
			\item Accompanying Textbook: Principles of Mathematical Analysis by Rudin.
		\end{itemize}
		\item \textbf{Homework}: Assigned each Friday, and due Sunday of the next week on Gradescope.
		\item \textbf{Exam}: Two exams.
		\begin{itemize}
			\item Midterm: March 10th, in class.
			\item Final: May 13th, 3-6pm.
		\end{itemize}
		\item \textbf{Score}: The breakdown is as follows:
		\begin{itemize}
			\item Homework: 25\%
			\item Midterm: 25\%
			\item Final: 50\%
		\end{itemize}
	\end{itemize}
\end{miscbox}

\subsection{Notation}
\begin{defn}[Sets and Elements]
	We define a set $S$ to be a collection of elements.
	
	We denote elements $x$ of $S$ to be $x \in S$. Similarly, for elements $y$ not in $S$, we write $y \not\in S$.
\end{defn}
\begin{defn}[Set Operations]
	We use $S_1 \subseteq S_2$ to denote $S_1$ to be a subset of $S_2$.
	
	We say that $S_1 \cap S_2$ is the intersection of the two sets; that is, it is the set of elements that are contained in both $S_1$ and $S_2$.
	
	Similarly, $S_1 \cup S_2$ is the union.
\end{defn}

\begin{example}[Common Sets]
	We denote the following:
	\begin{itemize}
		\item $\NN = \brc{1, 2, \ldots}$.
		\item $\ZZ = \brc{\ldots, -2, -1, 0, 1, 2, \ldots}$.
		\item $\QQ = \brc{\frac{p}{q} : p, q \in \ZZ, q \neq 0}$.
		\item $\RR$ is the real numbers.
	\end{itemize}
	
	Note that we use $\NN$ to denote the positive integers in this class.
\end{example}

\subsection{Visualizing the Reals}
There are two ways we can think of real numbers:
\begin{itemize}
	\item To begin with, we can think of $\RR$ as a line:
	% INSERT NUMBER LINE HERE
	\item Alternatively, we can think of them as decimal expansions: $\pi = 3.14\ldots$.
\end{itemize}

\subsection{A Basic Proof}
\begin{example}[Irrationality of $\sqrt{2}$]
	Suppose we want to prove the following: Show that $\sqrt{2}$ isn't rational.
	
	One approach is to use contradiction:
	
	\begin{proof}
		We first suppose that $\sqrt{2}$ is rational. Then, there exists $p, q \in \ZZ$ where $q \neq 0$ such that $\sqrt{2} = \frac{p}{q}$. Note that we assume that we can't further reduce this fraction.
		
		Then, we note that $\frac{p^{2}}{q^{2}} = 2$. Then, we have that $p^{2} = 2q^{2}$.
		
		We note that if $p, q$ have a common factor, we can cancel it out, leading to a contradiction.
		
		So, it must be that $p, q$ have no common factors. Then, we know that $p^{2} = 2q^{2}$ implies that $p$ is even. In other words, $p = 2m$ for some $m \in \ZZ$.
		
		So, we have that $4m^{2} = 2q^{2} \implies 2m^{2} = q$. But then, we note that $q$ is also even. But if $q, p$ are both even, then there exists a common factor. There is thus a contradiction here.
		
		Therefore, we conclude that $\sqrt{2}$ cannot be rational.
	\end{proof}
\end{example}

\subsection{Definitions and Examples}
\begin{defn}[Absolute Value]
	We define the absolute value $\lvert \cdot \rvert$ on $\RR$ as follows:
	\begin{equation*}
		\forall a \in \RR, \quad \lvert a \rvert = \begin{cases}
			a & a \geq 0 \\
			-a & a < 0
			\end{cases}
	\end{equation*}
\end{defn}

\begin{thm}[Properties of Absolute Value]
	We have the following:
	\begin{itemize}
		\item $\lvert a \rvert \geq 0$.
		\item $\lvert ab \rvert = \lvert a \rvert\lvert b \rvert$
		\item $\lvert a + b \rvert \leq \lvert a \rvert + \lvert b \rvert$.
	\end{itemize}
\end{thm}

\begin{defn}[Intervals]
	The following are closed intervals:
	\begin{itemize}
		\item $\br{a,b}$.
		\item $[a, \infty)$.
		\item $(-\infty, b]$
	\end{itemize}
	
	And the following are open intervals:
	\begin{itemize}
		\item $(a,b)$
	\end{itemize}
\end{defn}

\begin{defn}[Min and Max]
	Let $S \subseteq \RR$. Then, we say that:
	\begin{itemize}
		\item $x \in S$ is the maximum in $S$ if $\forall y \in S$, $x \geq y$.
		\item $y \in S$ is the minimum in $S$ if $\forall x \in S$, $y \leq x$
	\end{itemize}
\end{defn}

\begin{rmk}[Existence of Min/Max]
	We note that for any $S \subseteq \RR$, the maximum or minimum may not exist necessarily. For example, consider the open interval $(a,b)$ -- it has neither a minimum nor maximum.
	
	On the other hand, some half-closed or half-opened interval may have a minimum or maximum.
\end{rmk}

\begin{rmk}[Uniqueness of Min/Max]
	We note that for a set, if a minimum or maximum exists, it must be unique.
\end{rmk}

\begin{defn}[Upper and Lower Bounds]
	Let $S \subseteq \RR$. Then, we say the following:
	\begin{itemize}
		\item We call $M \in \RR$ an upper-bound if for all $x \in S$, $M \geq x$. If such an $M$ exists, then we call $S$ bounded from above.
		\item We call $m \in \RR$ a lower-bound of $S$ if $\forall x \in S$, $m \leq x$. Then, if such an $m$ exists, we say that $S$ is bounded from below.
	\end{itemize}
\end{defn}

\begin{example}[Boundedness]
	Suppose we have the following intervals:
	\begin{itemize}
		\item $\pr{a,b}$.
		\item $\br{a,b}$.
		\item $(a,b]$.
	\end{itemize}
	
	Then, $a$ is the lower bound of all of these intervals, and $b$ is the upper bound of all of these intervals.
	
	On the other hand, suppose we have the interval $\pr{a, \infty}$. Then, $a$ is the lower-bound, but the interval isn't bounded from above.
\end{example}

\begin{defn}[Supremum and Infimum]
	We define the supremum of a set $S$, denoted by $\sup S$, to be the least upper bound of $S$.
	
	Similarly, we define the infimum of a set $S$, denoted by $\inf S$, to be the greatest lower bound of $S$.
\end{defn}

\begin{thm}[Completeness Theorem]
	We note that $\RR$ is complete. That is, for any $S \subseteq \RR$, if it is bounded from above, then $\sup S$ exists.
\end{thm}
\begin{cor}
	If $S \subseteq \RR$ is bounded below, then $\inf S$ exists.
\end{cor}

\begin{rmk}
	We note that the Completeness Theorem doesn't hold on $\QQ$. That is, for any $S \subseteq \QQ$, if $S$ is bounded above, it doesn't necessarily mean that $\sup S \in \QQ$ exists.
	
	A straightforward counterexample is $S \subseteq \QQ$ such that $S = \brc{\frac{p}{q} : p, q \in \NN, p^{2} < 2q^{2}}$. We note then that this set is equivalent to $\pr{0, \sqrt{2}} \cap \QQ$; then, $\sup S = \sqrt{2} \not\in \QQ$. 
\end{rmk}

\section{Lecture -- 1/24/2025}
Recall previously that we stated that if $S \subseteq \RR$ is bounded above, then $\sup S$ exists. And as a corollary, we say that if $S \subseteq \RR$ is bounded below, then $\inf S$ exists.

In this lecture, we will be proving the Completeness Theorem. To do this, we must first define what $\RR$ actually is.

\subsection{The Real Numbers}
To define $\RR$, we must first introduce the following notion of cuts:
\begin{defn}[Cut]
	Let us define a ``cut" to be a subset $A \subseteq \QQ$ such that it satisfies the following conditions:
	\begin{enumerate}
		\item $A \neq \emptyset$ and $A \neq \QQ$.
		\item If $r \in A, t \in \QQ, t < r$, then $t \in A$.
		\item $A$ has no maximum.
	\end{enumerate}
\end{defn}

Then, we define $\RR$ as follows:
\begin{defn}[$\RR$]
	We define $\RR$ to be all cuts in $\QQ$. 
\end{defn}

\begin{rmk}
	Recall from before, $A_{\sqrt{2}} = \brc{t \in \QQ : t < \sqrt{2}}$ corresponds to $\sqrt{2}$. Then, we can think of $A$ as being a cut, with the supremum being $\sqrt{2}$.
	
	In general, we note that $t \leftrightarrow A_t = \brc{x \in \QQ : x < t}$.
	
	A way to visualize this is with a number line as follows:
	% INSERT NUMBER LINE WITH DOT LABELED T, THEN LEFT OF IT IS CURLY BRACES LABELED "RATIONAL"
\end{rmk}

Now, for two cuts $A_{t_1}, A_{t_2}$, we say that $A_{t_1} \leq A_{t_2}$ if $A_{t_1} \subseteq A_{t_2}$.

% INSERT NUMBER LINE WITH AT1 AND AT2

Now, let us make the following claim:
\begin{prop}
	If $S \subseteq \RR$ is bounded above, then $\bigcup_{x \in S} A_x = \brc{t \in \QQ : \exists x \in S, t \in A_x}$ is a cut.
	
	We claim then that such a set is a cut.
\end{prop}

A way to think about this is that since $S$ is a set of real numbers, then for every $x \in S$, $A_x$ is just cuts. Then, we just have the union of all the cuts. Then, by taking the union of each of these cuts, we are approaching $\sup S$.

Now, let us prove this claim.
\begin{proof}
	We want to check that the union is in fact a cut.
	
	To begin with, we know that $S \neq \emptyset$; then, it follows that $\bigcup_{x \in S} A_x \neq \emptyset$ as well.
	
	Next, since $S$ is bounded above, there exists a cut $A_y$ such that for every $x \in S$, $A_x \subseteq A_y$. In other words, $x \leq y$. Then, it follows that the union $\bigcup_{x \in S} A_x \subseteq A_y \neq \QQ$. Thus, we know that it isn't equal to $\QQ$ either. Thus, we have satisfied the first condition.
	
	Next, let $r \in \bigcup_{x \in S} A_x$. Then, let $t \in \QQ$ such that $t < r$. We want to show that $t \in \bigcup_{x \in S} A_x$. To do this, we note that there exists some $x_0 \in S$ such that $r \in A_{x_0}$. Then, because $t < r$, we note then that $t \in A_{x_0}$ as well; in other words, it's in the union $\bigcup_{x \in S} A_x$.
	
	Finally, we want to show that the union doesn't contain a maximum. To do this, we proceed by contradiction: suppose that $t_0$ is a maximum. Then, there exists some $x_0 \in S$ such that $t_0 \in A_{x_0}$.
	
	Then, we note that $t_0$ is a maximum of $A_{x_0}$. However, note that $A_{x_0}$ is a cut; it has no maximum. In other words, this is a contradiction. Therefore, we see that the union has no maximum.
	
	Thus, with all three conditions satisfied, we have shown that the union is in fact a cut.
\end{proof}

Now, we make another claim:
\begin{prop}
	We claim that $\bigcup_{x \in S} A_x$ is, in fact, $\sup S$.
\end{prop}
\begin{proof}
	We observe that for all $x \in S$, we have that $A_x \subseteq \bigcup_{x \in S} A_x$. In other words, we note that $A_x \leq \bigcup_{x \in S} A_x$ is an upper bound.
	
	Now, let us suppose that $A_y$ is an upper bound of $S$. Then, $A_x \subseteq A_y$ for all $x \in S$. We conclude then that $\bigcup_{x \in S} A_x \subseteq A_y$. Thus, we see that $\bigcup_{x \in S} A_x \leq A_y$, meaning that it is, in fact, $\sup S$.
\end{proof}

\begin{rmk}
	We note that for cuts $A_{t_1}, A_{t_2}$, we can define operations on these two sets (which correspond to real numbers!).
\end{rmk}

\begin{thm}[Real Numbers are finite]
	Assume $b \in \RR$. Then, there exists an $n \in \NN$ such that $n > b$.
\end{thm}
\begin{proof}
	Suppose for contradiction that for any $n \in \NN$, $n \leq b$. Then, we note that $\NN$ is bounded above by $b$.
	
	However, we note that if $\NN$ is bounded above, $\sup \NN = x_0 \in \RR$. Then, $x_0 - 1$ isn't an upper bound of $\NN$ (since $x_0$ is the least upper bound).
	
	Then, it must be that $\exists n_0 \in \NN$ such that $x_0 - 1 < n_0$; in other words, we have that $x_0 < n_0 + 1 \in \NN$. However, this contradicts $x_0$ being the upper bound of $\NN$.
	
	Thus, our theorem is true.
\end{proof}

\begin{thm}[$\QQ$ is Dense in $\RR$]
	For all $a,b \in \RR$ such that $a < b$, then there exists an $r \in \QQ$ such that $a < r < b$.
\end{thm}
\begin{proof}
	By the previous theorem, we can find an $n \in \NN$ such that $\frac{1}{b-a} < n$. Then, this means that $1 < nb - na$. In other words, $na + 1 < nb$; since the distance between $na, nb$ is greater than $1$, it must be then that there exists some $m \in \ZZ$ such that $na < m < nb \implies a < \frac{m}{n} < b$.
	
	Note then that $\frac{m}{n} = r \in \QQ$. Thus, we have proved our claim as desired.
\end{proof}

\chapter{The Second Week}
\section{Lecture -- 1/27/2025}
\subsection{Infimum and Supremum Recap}
Recall from last lecture that we introduced the Completeness Theorem and its corollary.

Last time, we proved this using cuts. However, this time, we will be using the idea of infimums and supremums. First, we will prove the corollary:
\begin{cor}
	If $S \subseteq \RR$ is bounded below, then $\inf S$ exists.
\end{cor}
\begin{proof}
	Suppose we have a set $S$ which is bounded below. Now, let us define the set $-S = \brc{-x : x \in S}$.
	
	Then, we know that since $S$ is bounded below, then there exists some $y \in \RR$ such that for all $x \in S$, $y \leq x$. But this means then that $-x \leq -y$. In other words, $-y$ is an upper bound of $-S$.
	
	Then, by our theorem, we know then that $\sup (-S) = y_0$ exists. Now, we claim that $\inf S = -y_0$. To do this, we note that since $\sup (-S) = y_0$, then $\forall -x \in -S$, we have that $-x \leq y_0 \implies -y_0 \leq x$ for all $x \in S$.
	
	Then, we see that it is indeed a lower bound. Now, to show that $-y_0$ is in fact the infimum, we first suppose there exists some lower bound $t > -y_0$ for $S$. Then, this means that for every $x \in S$, we have $t \leq x \implies -x \leq -t$. This means then that $t$ is an upper bound of $-S$. However, since we know that $-y_0$ is the supremum of $-S$, we must have that $-y_0 \leq -t \implies t \leq y_0$.
	
	Thus, we have a contradiction and conclude that, indeed, $-y_0$ is the greatest lower bound of $S$. That is, $\inf S = -y_0$.
\end{proof}

\begin{hw}
	Assume $A, B \subseteq \RR$ are bounded above. Then, let us define $C = \brc{a + b : a \in A, b \in B}$.
	
	Then, we want to show that it is bounded above, and that $\sup C = \sup A + \sup B$.
\end{hw}
\begin{solution}
	First, by construction of $C$, we note that every element in $C$ consists of a sum of elements in $a, b$.
	
	By definition of supremum, we observe then that for every element $a \in A$, we have that $a \leq \sup A$. Similarly, we have $b \leq \sup B$ for all $b \in B$.
	
	Then, with that in mind, we observe that for all $a \in A$ and $b \in B$, we have that $a + b \leq \sup A + \sup B$. That is, we see that $\sup A + \sup B$ is in fact an upper bound of $C$.
	
	Now, we want to show that this is in fact the least upper bound of $C$. To do this, we claim that $\sup A + \sup B - \varepsilon$ for any $\varepsilon > 0$ isn't an upper bound of $C$.
	
	Then, we observe that since $\sup A - \frac{1}{2} \varepsilon$ isn't a supremum of $A$ by definition of supremum, there exists some $a_0 \in A$ such that $a_0 > \sup A - \frac{1}{2}\varepsilon$.
	
	Similarly, since $\sup B - \frac{1}{2}\varepsilon$ isn't a supremum of $B$, it follows then that there exists some $b_0 \in B$ such that $b_0 > \sup B - \frac{1}{2} \varepsilon$.
	
	Then, it follows that there exists an $a_0 + b_0 > (\sup A - \frac{1}{2}\varepsilon) _ (\sup B - \frac{1}{2} \varepsilon) = \sup A + \sup B - \varepsilon$, where $a_0 + b_0 \in C$. Thus, we see that, indeed, $\sup A + \sup B - \varepsilon$ can't be a supremum for any $\varepsilon > 0$.
	
	Thus, we can conclude that, indeed, $\sup A + \sup B$ is the supremum of $C$.
\end{solution}

\begin{rmk}
	Let $y_0$ be an upper bound of some set $S$. To show that $y_0$ is the supremum, we can proceed in two ways:
	\begin{enumerate}
		\item For any upper bound $t$, if $t \geq y_0$, then $y_0$ must be the smallest upper bound.
		\item For any $\varepsilon > 0$, we show that $y_0 - \varepsilon$ isn't an upper bound.
	\end{enumerate}
\end{rmk}

\subsection{Sequence and Limit}
\begin{defn}[Sequence]
	We define a real number sequence to be a function $s : \NN \rightarrow \RR$. Usually, we denote this as $\pr{S_1, S_2, \ldots}$ or $\pr{S_n}$
\end{defn}
\begin{rmk}
	The initial subscript can be any $m \in \ZZ$, including negative numbers.
\end{rmk}

\begin{example}[Different Sequences]
	Let $S_n = \frac{1}{n^{2}}$ for $n \in \NN$. Then, we have $\pr{1, \frac{1}{4}, \frac{1}{9}, \ldots}$.
	
	Let $S_n = \pr{-1}^{n}$ for $n \in \NN$. Then, we have $\pr{-1, 1, \ldots}$.
\end{example}

\begin{defn}[Limit of Sequence]
	We say that $\pr{S_n}$ converges to $r \in \RR$ if $\forall \varepsilon > 0, \exists N$ such that $\forall n > N$, we have $\lvert S_n - S \rvert < \varepsilon$.
	
	We denote the limit as: $\lim\limits_{n \rightarrow \infty} S_n = S$, or $S_n \rightarrow S$, or simply $\lim S_n = S$.
	
	% INSERT n->\infty above of the right arrow for second one
\end{defn}
\begin{rmk}
	If $\pr{S_n}$ has no limit, then we say that it diverges.
\end{rmk}

% INSERT NUMBER LINE WITH S_1, S_2, S_3, ... LABELLED ON IT, WITH LIMIT S AND INTERVAL OF +-EPSILON AROUND.
% THEN FOR ALL n > N, S_n INSIDE OF THE INTERVAL.

\begin{rmk}
	We say that $S$ is not the limit of $\pr{S_n}$ iff $\exists \varepsilon > 0$ such that $\forall N > 0, \exists n > N$, $\lvert S_n - S \rvert \geq \varepsilon$.
	
	In other words, there exists some interval size around $S_n$ such that for any value of $N$ we pick, we can always find some $n > N$ such that $S_n$ lies outside of our interval.
\end{rmk}

\begin{example}[Proving Limits]
	Prove that $\lim \frac{1}{n^{2}} = 0$.
\end{example}
To show that $\lim \frac{1}{n^{2}} = 0$, we want to show that for any $\varepsilon > 0$ we pick, we want to find some $N$ such that for any $n > N$, $\lvert \frac{1}{n^{2}} - 0 \rvert = \frac{1}{n^{2}} < \varepsilon$.

Now, with that in mind, we observe that since $n > N$, we have that $\frac{1}{n^{2}} < \frac{1}{N^{2}}$. Then, let $N = \frac{1}{\sqrt{\varepsilon}}$. 

Then, we have:
\begin{align*}
	\frac{1}{n^{2}} &< \frac{1}{N^{2}} \\
	&= \frac{1}{\pr{\frac 1 {\sqrt\varepsilon}}^{2}} \\
	&= \varepsilon
\end{align*}

Now, we will formally prove it:
\begin{proof}
	For all $\varepsilon > 0$, we let $N = \frac{1}{\sqrt{\varepsilon}}$.  Then, for all $n > N = \frac{1}{\sqrt{\varepsilon}}$, we have that:
	\begin{align*}
		\lvert \frac{1}{n^{2}} - 0 \rvert &= \frac {1}{n^{2}} \\
		&< \frac{1}{N^{2}} \\
		&= \varepsilon
	\end{align*}
	
	Thus, $\lim \frac{1}{n^{2}} = 0$.
\end{proof}
\begin{rmk}
	Note, we can also take something like $N = \frac{2}{\sqrt{\varepsilon}}$, since $\frac{1}{n^{2}} < \frac{\varepsilon}{4} < \varepsilon$.
	
	Typically, our choice of $N$ isn't unique.
\end{rmk}

\section{Lecture -- 1/29/2025}
Recall previously that we defined what a convergent sequence is:
\begin{defn}[Convergence of a Sequence]
	We say that a sequence $S_n$ converges to some number $S$ if for all $\varepsilon > 0$, there exists some $N$ such that for all $n > N$, we have that $\lvert S_n - S \rvert < \varepsilon$.
\end{defn}

Now, we look at some further examples:
\begin{example}
	We want to prove that $\lim \frac{3n+1}{7n-4} = \frac{3}{7}$.
	
	Now, the basic idea is that we can divide the top and bottom by $n$, and thus we have $\frac{3 + \frac{1}{n}}{7 - \frac{4}{n}}$; as $n \rightarrow \infty$, we see that it approaches $\frac{3}{7}$.
	
	So, right now, we want to find some $N$ such that for all $n > N$, we have:
	\begin{equation*}
		\lvert \frac{3n+1}{7n-4} - \frac{3}{7} \rvert < \varepsilon
	\end{equation*}
	
	Then, we can find the common denominator to get:
	\begin{align*}
		\lvert \frac{3n+1}{7n-4} - \frac{3}{7} \rvert &= \lvert \frac{7(3n+1) - 3(7n-4)}{7(7n-4)} \rvert \\
		&= \lvert \frac{19}{7(7n-4)} \rvert
	\end{align*}
	
	Now, because we assume that $n \geq 1$, we can remove the absolute value to get:
	\begin{equation*}
		\lvert \frac{19}{7(7n-4)} \rvert = \frac{19}{7(7n-4)}.
	\end{equation*}
	
	Now, we want to solve:
	\begin{align*}
		\frac{19}{7(7n-4)} &< \varepsilon \\
		\frac{19}{7\varepsilon} &< 7n-4 \\
		n &> \frac{\frac{19}{7\varepsilon} + 4}{7}
	\end{align*}
	
	Thus, if we let $N = \frac{\frac{19}{7\varepsilon} + 4}{7}$, we have proven the claim as desired.
	
	Thus, for the formal proof, we have:
	\begin{proof}
		For all $\varepsilon > 0$, we take $N = \frac{\frac{19}{7\varepsilon} + 4}{7}$. Then, for all $n > N$, we have that $\lvert S_n - S \rvert = \lvert \frac{19}{7(7n-4)} \rvert < \varepsilon$.
	\end{proof}
\end{example}

Now, how do we prove if a sequence doesn't have a limit?
\begin{example}
	We want to show that $a_n = \pr{-1}^{n}$ has no limit.
	
	The idea then is to use some kind of contradiction argument. So, let us suppose that $\lim \pr{-1}^{n} = a$, for some $a$.
	
	We can now visualize this sequence:
	% INSERT number line with -1 and 1 labelled, with some $a$ in-between ig
	
	Here, we see that the distance $\mathrm{dis}(a, -1) + \mathrm{dis}(a, 1) \geq \mathrm{dis}(-1, 1) = 2$.
	
	Then, at least one of $\mathrm{dis}(a, -1), \mathrm{dis}(a, 1) \geq 1$.
	
	Now, let us suppose for the sake of contradiction that $\lim a_n = \lim \pr{-1}^{n} = a$.
	
	By our assumption, take $\varepsilon = 1$. Then, there exists $N > 0$ such that for all $n > N$, $\lvert a_n - a \rvert < 1$.
	
	Then, take an even number $n > N$, we see that $\lvert 1 - a \rvert < 1$. For an odd number $n + 1$, we see that $\lvert a_{n+1} - a \rvert = \lvert -1 - a \rvert = \lvert 1 + a \rvert < 1$.
	
	Then, we have:
	\begin{align*}
		\lvert 1+a \rvert + \lvert 1 - a \rvert &< 2
	\end{align*}
	
	Then, using the Triangle Inequality, we have:
	\begin{align*}
		 \\
		\lvert 1 + a + 1 - a \rvert &\leq \lvert 1 + a \rvert + \lvert 1 - a \rvert < 2 \\
		2 &\leq \lvert 1 + a \rvert + \lvert 1 - a \rvert < 2 \\
		2 &< 2
	\end{align*}
	
	Thus, we have a contradiction.
\end{example}

\begin{example}
	We want to prove that:
	\begin{equation*}
		\lim \frac{4n^{3} + 3n}{n^{3} - 6} = 4
	\end{equation*}
	
	Then, for all $\varepsilon > 0$, the idea then is to find some $N$ such that for all $n > N$, we have:
	\begin{equation*}
		\lvert \frac{4n^{3} + 3n}{n^{3} - 6} - 4 \rvert < \varepsilon
	\end{equation*}
	
	Then, we observe that:
	\begin{align*}
		\lvert \frac{4n^{3} + 3n}{n^{3} - 6} - \frac{4(n^{3} - 6)}{n^{3} - 6} \rvert &= \lvert \frac{3n+24}{n^{3} - 6} \rvert
	\end{align*}
	
	Now, the issue is our denominator has a $n^{3}$ term. Then, when rewriting, we'll run into some issues.
	
	So, instead, we want to find some $a_n > 0$ such that $\lvert \frac{3n+24}{n^{3} - 6} \rvert < a_n$; then, it'll be easy to solve for $a_n < \varepsilon$.
	
	So, looking at this sequence, we see that since $n \geq 1$, we have that $3n + 24 \leq 3n + 24n = 27n$.
	
	Then, if $\frac{1}{2}n^{3} > 6$, then we have that $n^{3} - 6 > \frac{1}{2}n^{3}$. But, we have that $\frac{1}{2}n^{3} > 6 \iff n^{3} > 12 \iff n > 3$.
	
	So, when $n > 3$, we will always have:
	\begin{align*}
		\lvert \frac{3n+24}{n^{3} - 6} \rvert &< \frac{27n}{\frac{1}{2}n^{3}} \\
		&= \frac{54}{n^{2}} \\
		&< \varepsilon
	\end{align*}
	
	So, we have:
	\begin{equation*}
		n > \sqrt{\frac{54}{\varepsilon}}
	\end{equation*}
	
	Now, because we have $n > 3$ here, we need $N$ to be at least $3$. Then, we set $N = \max\brc{3, \sqrt{\frac{54}{\varepsilon}}}$.
	
	Now, for a formal proof, we can proceed as such:
	\begin{proof}
		For all $\varepsilon > 0$, take $N = \max\brc{3, \sqrt{\frac{54}{\varepsilon}}}$.
		
		Then, we observe that for all $n > N$, we have that:
		\begin{align*}
			\lvert \frac{4n^{3} + 3n}{n^{3} - 6} - 4 \rvert &= \lvert \frac{3n+24}{n^{3} - 6} \rvert \\
			&< \frac{27n}{\frac{1}{2}n^{3}} \\
			\frac{54}{n^{2}} &< 54 \cdot \frac{\varepsilon}{54} \\
			&= \varepsilon
		\end{align*}
	\end{proof}
\end{example}

Finally, we will show the following theorem:
\begin{thm}
	If $\pr{S_n}$ converges, then the limit is unique.
\end{thm}
\begin{proof}
	We will proceed by contradiction. Then, $\pr{S_n}$ converges but has two limits, $s, t$ such that $s \neq t$.
	
	Now, we take $\varepsilon = \frac{1}{2}\lvert t - s \rvert$. Since we have two limits, there exists $N_1$ and $N_2$ such that for every $n > N_1$ we have $\lvert S_n - S \rvert < \varepsilon$, and $n > N_2$ we have $\lvert S_n - T \rvert < \varepsilon$.
	
	Now, if we take $N = \max\brc{N_1, N_2}$. Then, we have for every $n > N$, we have $\lvert S_n - S \rvert < \varepsilon$ and $\lvert S_n - T \rvert < \varepsilon$.
	
	Now, we see that:
	\begin{align*}
		\lvert S_n - S \rvert &< \varepsilon \\
		\lvert S_n - T \rvert &< \varepsilon \\
		\lvert S_n - S \rvert + \lvert T - S_n \rvert &< 2\varepsilon \\
		\lvert S_n - S + T - S_n \rvert &\leq \lvert T - S_n \rvert + \lvert S_n - S \rvert < 2\varepsilon \\
		2\varepsilon = \lvert T-S \rvert &< 2\varepsilon 
	\end{align*}
	
	Thus, we have a contradiction.
\end{proof}

\section{Lecture -- 1/31/2025}
Recall the definition of limits:
\begin{defn}[Convergence of a Limit]
	We say that $\lim S_n = S$ if $\forall \varepsilon > 0, \exists N$ such that $\forall n > N$, we have $\lvert S_n - S \rvert < \varepsilon$.
\end{defn}

Now, we look at the following example:
\begin{example}
	Assume $\forall n \in \NN$, $S_n > 0$ and $\lim S_n = S > 0$. Then, we claim that $\lim \sqrt{S_n} = \sqrt{S}$. Then, for all $\varepsilon > 0$, we want to find $N$ such that for all $n > N$, we have that $\lvert \sqrt{S_n} - \sqrt{S} \rvert < \varepsilon$.
	
	To do this, we first observe that $\pr{\sqrt{S_n} - \sqrt{S}}\pr{\sqrt{S_n} + \sqrt{S}} = S_n - S$. Then, we observe that:
	\begin{equation*}
		\sqrt{S_n} - \sqrt{S} = \frac{S_n - S}{\sqrt{S_n} + \sqrt{S}}.
	\end{equation*}
	
	Now, since $\sqrt{S_n} > 0$, we observe that $\sqrt{S_n} + \sqrt{S} > \sqrt{S}$. So, this means that:
	\begin{align*}
		\abs{\sqrt{S_n} - \sqrt{S}} &= \abs{\frac{S_n - S}{\sqrt{S_n} + \sqrt{S}}} \\
		&= \frac{\abs{S_n - S}}{\abs{\sqrt{S_n} + \sqrt{S}}} \\
		&< \frac{\abs{S_n - S}}{\sqrt{S}} \\
		&< \varepsilon
	\end{align*}
	
	Then, we see that $\lvert S_n - S \rvert < \sqrt{S} \cdot \varepsilon$.
	
	Then we use the assumption that $\lim S_n = S$ to find $N$. As a proper proof, we have:
	\begin{innerproof}
		$\forall \varepsilon > 0$, we let $\varepsilon' = \sqrt{S} \cdot \varepsilon$. Then, there exists an $N$ such that for all $n > N$, we have $\lvert S_n - S \rvert < \varepsilon' = \sqrt{S} \cdot \varepsilon$.
		
		Then, we have:
		\begin{align*}
			\abs{\sqrt{S_n} - \sqrt{S}} &= \frac{\abs{S_n - S}}{\abs{\sqrt{S_n} + \sqrt{S}}} \\
			&< \frac{\abs{S_n - S}}{\sqrt{S}} \\
			&< \frac{\sqrt{S}\varepsilon}{\sqrt{S}} \\
			&= \varepsilon
		\end{align*}
	\end{innerproof}
\end{example}

\begin{thm}[Squeeze Theorem]
	Assume $a_n \leq s_n \leq b_n$, for all $n \in \NN$, and $\lim a_n = \lim b_n = S$, then we have that $\lim s_n = S$. 
\end{thm}
\begin{proof}
	For all $\varepsilon > 0$, there exists some $N_1$ such that for all $n > N_1$, we have $\lvert a_n - S \rvert < \varepsilon$.
	
	Similarly, there exists some $N_2$ such that $n > N_2$, we have that $\lvert b_n - S \rvert < \varepsilon$.
	
	Then, if we take $N = \max\brc{N_1, N_2}$, we have that for all $n > N$, we have $\lvert a_n - S \rvert < \varepsilon$ and $\lvert b_n - S \rvert < \varepsilon$.
	
	Recall from homework that we have the following property:
	\begin{lem}
		If $\lvert a - b \rvert \leq c$, then we have that $b - c \leq a \leq b + c$.
	\end{lem}
	
	Using this lemma, we see then that:
	\begin{align*}
		s - \varepsilon < a_n < s + \varepsilon \\
		s - \varepsilon < b_n < s + \varepsilon
	\end{align*}
	
	Next, we note that:
	\begin{align*}
		s_n \leq b_n < s + \varepsilon \\
		s_n \geq a_n > s-\varepsilon
	\end{align*}
	
	So, we have $s - \varepsilon < s_n < s + \varepsilon \implies \lvert s_n - s \rvert < \varepsilon$. In other words, $\lim s_n = S$.
\end{proof}

And from this theorem, we have the following corollary:
\begin{cor}
	If for all $n \in \NN$, we have $\lvert s_n \rvert \leq t_n$, and $\lim t_n = 0$, then ew have that $\lim s_n = 0$.
\end{cor}
\begin{proof}
	Since $\lvert s_n \rvert \leq t_n$, we have that $-t_n \leq s_n \leq t_n$, and since $\lim t_n = 0$, we have $0 \leq s_n \leq 0$. 
\end{proof}

\subsection{Properties of Limits}
\begin{thm}
	Assume that $\lim s_n = s$, and $\lim t_n = t$. Then, we have:
	\begin{itemize}
		\item $\forall k \in \RR$, we have $\lim ks_n = ks$.
		\item $\lim(s_n + t_n) = s + t$.
		\item $\lim (s_n \cdot t_n) = s \cdot t$.
		\item If $\forall n \in \NN, s_n \neq 0$, we have $\lim \frac{t_n}{s_n} = \frac{t}{s}$.
	\end{itemize}
\end{thm}
\begin{proof}
	For the first property, we proceed as such:
	\begin{innerproof}
		We note that for all $\varepsilon > 0$, we let $\varepsilon' = \frac{\varepsilon}{\abs{k}}$.
		
		Then, since $\lim s_n = s$, we have that there exists $N$ such that for all $n > N$, we have $\lvert s_n - s \rvert < \varepsilon' = \frac{\varepsilon}{\abs{k}}$.
		
		Then, we note that $\lvert ks_n - ks \rvert = \abs{k}\lvert s_n - s \rvert < \frac{\varepsilon}{\abs{k}}$. Thus, we have $\lvert s_n - s \rvert < \varepsilon$ as desired.
	\end{innerproof}
	
	For the second property, we prove it as follows:
	\begin{innerproof}
		For all $\varepsilon > 0$, we set $\varepsilon' = \frac{\varepsilon}{2}$.
		
		Then, since $\lim s_n = s, \lim t_n = t$, we note that there exists $N_1, N_2$ such that for all $n > N_1$, and $n > N_2$, we have:
		\begin{itemize}
			\item $\lvert s_n - s \rvert < \varepsilon' = \frac{\varepsilon}{2}$.
			\item $\lvert t_n - t \rvert < \varepsilon' = \frac{\varepsilon}{2}$.
		\end{itemize}
		
		Then, we can simply take $N = \max\brc{N_1, N_2}$, we have:
		\begin{itemize}
			\item $\lvert s_n - s \rvert < \varepsilon' = \frac{\varepsilon}{2}$.
			\item $\lvert t_n - t \rvert < \varepsilon' = \frac{\varepsilon}{2}$.
		\end{itemize}
		
		Then, we observe that:
		\begin{align*}
			\lvert (s_n + t_n) - (s + t) \rvert &= \lvert (s_n - s) + (t_n - t) \rvert \\
			&\leq \lvert s_n - s \rvert + \lvert t_n - t \rvert \\
			&< \varepsilon' + \varepsilon' \\
			&= \frac{\varepsilon}{2} + \frac{\varepsilon}{2} \\
			&= \varepsilon
		\end{align*}
	\end{innerproof}
\end{proof}

\begin{defn}[Bounded Sequences]
	We say that a sequence $s_n$ is bounded if there exists some $M > 0$ such that for all $n \in \NN$, we have $\lvert s_n \rvert \leq M$.
\end{defn}

\begin{thm}
	If a sequence $s_n$ is convergent -- that is, $\lim s_n = s$ -- then it is bounded.
\end{thm}
\begin{proof}
	The intuition here is that convergence tells us that our sequence, after a certain point $N$, will converge to within some interval around our limit $s$. So, we can take the maximum of this interval.
	
	However, it says nothing for $n < N$. To solve this issue, we can take the maximum value of $S_n$ for all of these $n$ before, along with the maximum of our interval at the end. 
	
	Suppose that $s_n$ is convergent. Then, this means that there exists some $N$ such that for all $n > N$, we have $\lvert S_n - S \rvert < 1$. Then, this implies that $S - 1 \leq S_n \leq S + 1$.
	
	Take, we can simply take $M = \max\brc{\lvert S_1 \rvert, \lvert S_2 \rvert, \ldots, \lvert S_k \rvert}$, where $k$ is the largest natural number less than or equal to $N$.
	
	Then, we see that for all $n \leq N$, we have $\lvert S_n \rvert \leq M$ by definition of $M$. Meanwhile, for any $n > N$, we have $\lvert S_n \rvert = \lvert S_n - S + S \rvert \leq \lvert S_n - S \rvert + \lvert S \rvert$.
	
	And we see that that since $\lvert S_n - S \rvert < 1$, we have that $\lvert S_n \rvert < 1 + \lvert S \rvert < M$.
\end{proof}

\chapter{On the Third Week of Hell}
\section{Lecture -- 2/3/2025}
Once more, we recall that we say that $\lim s_n = s$ if for all $\varepsilon > 0$, there exists some $N$ such that for all $n > N$, we have $\abs{s_n - s} < \varepsilon$.

Furthermore, we recall by the Squeeze Theorem that if $a_n \leq s_n \leq b_n$, and we have that $\lim a_n = \lim b_n = s$, then $\lim s_n = s$.

And there is also a corollary that if $\abs{s_n} \leq t_n$, and $\lim t_n = 0$, then $\lim s_n = 0$.

Finally, we note the Binomial Expansion:
\begin{thm}[Binomial Theorem]
	We say that:
	\begin{equation*}
		\binom n k = \frac{n!}{k!(n-k)!}
	\end{equation*}
	
	Then, the Binomial Theorem tells us that $(1+x)^{n}$ is equal to:
	\begin{equation*}
		(1+x)^{n} = \sum_{k=0}^{n} \binom n k x^{k}
	\end{equation*}
\end{thm}

For example, let $n \geq 2$, then we see that $(1+x)^{n}$ is equal to:
\begin{equation*}
	(1+x)^{n} = 1 + nx + \frac{n(n-1)}{2}x^{2} + \cdots
\end{equation*}

Now, we introduce the following claims:
\begin{itemize}
	\item $\lim \frac{1}{n^{p}} = 0$, for all $p > 0$.
	\item $\lim a^{n} = 0$ if $\abs{a} < 1$.
	\item $\lim n^{\frac{1}{n}} = 1$.
	\item $\lim a^{\frac{1}{n}} = 1$ for all $a > 0$.
\end{itemize}

\begin{example}
	Suppose we want to prove that $\lim \frac{1}{n^{p}} = 0$. 
	
	Recall from before that we've already done this for the case of $p = 2$. Then, for any $\varepsilon > 0$, we can take $N = \sqrt[p]{\frac{1}{\varepsilon}}$. Then, we observe that:
	\begin{equation*}
		\abs{\frac 1 {n^{p}}} = \frac{1}{n^{p}} < \frac{1}{N^{p}} = \frac{1}{\frac{1}{\varepsilon}} = \varepsilon
	\end{equation*}
	
	Thus, we observe that, indeed, the limit is equal to $0$.
\end{example}

\begin{example}
	Let us prove the second claim.
	
	Let $a = 0$, then we see that $a^{n} = 0^{n} = 0$. So, trivially, we see that $\lim 0 = 0$.
	
	Now, in the case where $0 < \abs{a} < 1$. So, we can let $b = \frac{1}{\abs{a}} - 1$. Now, note that since $\abs{a} < 1$, we see then that $b > 0$. Next, note that:
	\begin{align*}
		\frac 1 {\abs{a}} &= b + 1 \\
		\abs{a} &= \frac 1 {b+1}
	\end{align*}
	
	Now, we look at $\abs{a^{n}} = \pr{\frac{1}{b+1}}^{n}$. Then, by the binomial expansion, we note that $(b+1)^{n} \geq 1 + bn > bn$. So, we have then that:
	\begin{align*}
		\abs{a^{n}} < \frac{1}{bn}.
	\end{align*}
	
	Note that $b$ is a constant, then we see that $\lim \frac{1}{bn} = 0$. Now, because $\abs{a^{n}} < \frac{1}{bn}$, and $\lim \frac{1}{bn} = 0$, it follows then that $\lim a^{n} = 0$ by the Squeeze Theorem's corollary.
\end{example}

\begin{example}
	For our third claim, we observe that $s_n = n^{\frac{1}{n}} - 1$. Then, we see that $n^{\frac{1}{n}} = s_n + 1 \implies n = \pr{s_n + 1}^{n}$.
	
	Now, if $n \geq 2$, we observe that:
	\begin{equation*}
		n = (1+s_n)^{n} \geq 1 + ns_n + \frac{n(n-1)}{2}s_n^{2}
	\end{equation*}
	
	Now, we can drop the first two terms to see then that:
	\begin{align*}
		n &> \frac{n(n-1)}{2}s_n^{2} \\
		1 &> \frac{n-1}{2}s_n^{2} \\
		s_n^{2} &< \frac{2}{n-1}
	\end{align*}
	
	So, we have:
	\begin{equation*}
		0 < s_n < \sqrt{\frac{2}{n-1}} = 0
	\end{equation*}
	
	Then, by the Squeeze Theorem, we see then that since $0 < s_n < 0$, it follows then that $\lim s_n = 0$.
\end{example}
\begin{rmk}
	Note here that we originally wanted to prove $\lim n^{\frac{1}{n}} = 1$. However, note that $\lim s_n = c \implies \lim s_n + k = c + k$. So, in this case, we simply added $k = -1$.
\end{rmk}

\begin{example}
	Now, we prove that last claim.
	
	If $a \geq 1$, we note then that if $n > a$, we have $a^{\frac{1}{n}} \leq n^{\frac{1}{n}}$.
	
	Then, since $a \geq 1$, we have then that $1 \leq a^{\frac{1}{n}} \leq n^{\frac{1}{n}}$. And we know that $\lim n^{\frac{1}{n}} = 1$ from the previous claim. Thus, we can apply the Squeeze Theorem and see that, indeed, $\lim a^{\frac{1}{n}} = 1$ for $a \geq 1$.
	
	On the other hand, if $0 < a < 1$, we note then that $\frac{1}{a} \geq 1$. Then, with that in mind, we observe that:
	\begin{equation*}
		\lim \pr{\frac 1 a}^{n} = 1.
	\end{equation*}
	
	Then, with that in mind, we observe that $\lim a^{n} = \lim \frac{1}{\pr{\frac{1}{a}}^{n}} = 1$.
\end{example}

\begin{example}
	Let $s_n = \frac{n^{3} + 6n + 7}{4n^{3} + 3n + 4}$. Now, we want to show that $\lim s_n = \frac{1}{4}$.
	
	To do this, we first observe that we can divide the top and bottom by $n^{3}$ to get:
	\begin{align*}
		s_n &= \frac{1+ \frac 6 n + \frac 7{n^{3}}} {4 + \frac 3 {n^{2}} - \frac 4{n^{3}}} \\
		\lim s_n &= \frac 1 4
	\end{align*}
\end{example}

Now, we discuss infinities. First, recall that $\pm \infty$ are not real numbers. Furthermore, note that while $(a, \infty)$ is an open interval, note that $[a, \infty)$ is a closed interval.

We can think of $\infty$ as being both open and closed. Namely, if we consider $(-\infty, \infty) = \RR$, note that this is both open and closed.

Now, we introduce the definition of a sequence's limit being infinity:
\begin{defn}[Divergence to Infinity]
	We say that if $\lim s_n = \infty$, then for all $M > 0$, there exists some $N$ such that for all $n > N$, we have that $s_n > M$.
	
	Then, we say that $s_n$ diverges to $\infty$.
\end{defn}

\begin{defn}[Divergence to Negative Infinity]
	We say that $\lim s_n = -\infty$ if for all $M > 0$, we there exists some $N$ such that for all $n > N$, we have that $s_n < -M$.
	
	We say then that $s_n$ diverges to $-\infty$.
\end{defn}

\begin{rmk}
	Note that limit theorems may not hold for $\pm \infty$. For example, $(\infty) + (-\infty)$ may not be defined.
	
	More concretely, let us consider $s_n = n$ and $t_n = \sqrt{n}$. Then, we observe that $\lim s_n - t_n = \infty$.
	
	However, if we let $s_n = n$ and $t_n = \begin{cases}
	\sqrt{n} & n = 2k \\ n^{2} & n = 2k+1 \end{cases}$. Then, while $\lim s_n - t_n$ isn't defined.
	
	Furthermore, we note that $\frac{\infty}{\infty}$ may not be defined.
	
	On the other hand, $\infty + \infty = \infty$, and $\pr{\infty}\pr{\infty} = \infty$.
\end{rmk}
\end{document}