\documentclass{article}
% PREAMBLE %
%BEGIN_FOLD
% PACKAGES
\usepackage{amsmath}
\usepackage{amssymb}
\usepackage{amsthm}
\usepackage{cabin} % section title font
\usepackage[default]{cantarell} % default font
\usepackage[shortlabels]{enumitem}
\usepackage{fancyhdr}
\usepackage{graphicx}
\usepackage{hyperref}
\usepackage{mathtools}
\usepackage[framemethod=TikZ]{mdframed}
\usepackage[scr]{rsfso} % power set symbol
\usepackage{stmaryrd}
\usepackage{tasks} % vaguely remember this being important for something...?
\usepackage{tikz} % diagrams
\usepackage{titlesec}
\usepackage{thmtools}
\usepackage{varwidth}
\usepackage{verbatim} % longer comments
\usepackage{xcolor}
\usepackage{xparse}
%

% COLOURS
\definecolor{darkgreen}{HTML}{19A514}
\definecolor{lightgreen}{HTML}{9DFF9A}
\definecolor{darkblue}{HTML}{3E5FE4}
\definecolor{lightblue}{HTML}{BCDEFF}
\definecolor{darkred}{HTML}{CC3333}
\definecolor{lightred}{HTML}{FFA9A9}
\definecolor{darkpurple}{HTML}{A933CD}
\definecolor{lightpurple}{HTML}{F0BAFF}
\definecolor{darkyellow}{HTML}{D2D22A}
\definecolor{lightyellow}{HTML}{FFFFAE}
\definecolor{hyperlinkblue}{HTML}{3366CC}
%

% PAGE SETUP
% BASIC %
\setlength\parindent{0pt} % paragraph indentation
\setlength{\parskip}{5pt} % spacing between paragraphs
\usepackage[margin=1in]{geometry} % margin size

% HEADER/FOOTER %
\pagestyle{fancy}
\fancyhf{}
\fancyfoot[R]{\thepage} % page number on bottom right
\fancyhead[R]{\textit{\leftmark}} % section title
\renewcommand{\headrulewidth}{0pt} % removing horizontal line at the top

% HYPERLINK FORMATTING %
\hypersetup{
	colorlinks,    
	linkcolor=hyperlinkblue,
	urlcolor=hyperlinkblue,
	pdftitle={...},
	pdfauthor={Michael Pham},
}
%

% ENVIRONMENTS STYLES
% SOLUTION ENVIRONMENT %
\newenvironment{solution}{\begin{proof}[Solution]}{\end{proof}}

% PURPLE BOX %
\declaretheoremstyle[
mdframed={
	backgroundcolor=lightpurple,
	linecolor=darkpurple,
	rightline=false,
	topline=false,
	bottomline=false,
	linewidth=2pt,
	innertopmargin=8pt,
	innerbottommargin=8pt,
	innerleftmargin=8pt,
	leftmargin=-2pt,
	skipbelow=2pt,
	nobreak
},
headfont=\normalfont\bfseries\color{darkpurple}
]{purplebox}

% GREEN BOX %
\declaretheoremstyle[
mdframed={
	backgroundcolor=lightgreen,
	linecolor=darkgreen,
	rightline=false,
	topline=false,
	bottomline=false,
	linewidth=2pt,
	innertopmargin=8pt,
	innerbottommargin=8pt,
	innerleftmargin=8pt,
	leftmargin=-2pt,
	skipbelow=2pt,
	nobreak
},
headfont=\normalfont\bfseries\color{darkgreen}
]{greenbox}

% YELLOW BOX %
\declaretheoremstyle[
mdframed={
	backgroundcolor=lightyellow,
	linecolor=darkyellow,
	rightline=false,
	topline=false,
	bottomline=false,
	linewidth=2pt,
	innertopmargin=8pt,
	innerbottommargin=8pt,
	innerleftmargin=8pt,
	leftmargin=-2pt,
	skipbelow=2pt,
	nobreak
},
headfont=\normalfont\bfseries\color{darkyellow}
]{yellowbox}

% BLUE BOX %
\declaretheoremstyle[
mdframed={
	backgroundcolor=lightblue,
	linecolor=darkblue,
	rightline=false,
	topline=false,
	bottomline=false,
	linewidth=2pt,
	innertopmargin=8pt,
	innerbottommargin=8pt,
	innerleftmargin=8pt,
	leftmargin=-2pt,
	skipbelow=2pt,
	nobreak
},
headfont=\normalfont\bfseries\color{darkblue}
]{bluebox}

% RED BOX %
\declaretheoremstyle[
mdframed={
	backgroundcolor=lightred,
	linecolor=darkred,
	rightline=false,
	topline=false,
	bottomline=false,
	linewidth=2pt,
	innertopmargin=8pt,
	innerbottommargin=8pt,
	innerleftmargin=8pt,
	leftmargin=-2pt,
	skipbelow=2pt,
	nobreak
},
headfont=\normalfont\bfseries\color{darkred}
]{redbox}
%

% ENVIRONMENTS
% PURPLE BOXES (theorems, propositions, lemmas, and corollaries) %
\declaretheorem[style=purplebox,name=Theorem,within=section]{thm}
\declaretheorem[style=purplebox,name=Theorem,sibling=thm]{theorem}
\declaretheorem[style=purplebox,name=Theorem,numbered=no]{thm*, theorem*}
\declaretheorem[style=purplebox,name=Proposition,sibling=thm]{prop, proposition}
\declaretheorem[style=purplebox,name=Proposition,numbered=no]{prop*, proposition*}
\declaretheorem[style=purplebox,name=Lemma,sibling=thm]{lem, lemma}
\declaretheorem[style=purplebox,name=Lemma,numbered=no]{lem*, lemma*}
\declaretheorem[style=purplebox,name=Corollary,sibling=thm]{cor, corollary}
\declaretheorem[style=purplebox,name=Corollary,numbered=no]{cor*, corollary*}

% GREEN BOXES (definitions) %
\declaretheorem[style=greenbox,name=Definition,sibling=thm]{definition, defn}
\declaretheorem[style=greenbox,name=Definition,numbered=no]{definition*, defn*}

% BLUE BOXES (problems) %
\declaretheorem[style=bluebox,name=Problem,numberwithin=section]{homework}
\declaretheorem[style=bluebox,name=Problem,numbered=no]{homework*, hw*}

% RED BOXES %
\declaretheorem[style=redbox,name=Remark,sibling=thm]{remark, rmk}
\declaretheorem[style=redbox,name=Remark, numbered=no]{remark*, rmk*}
\declaretheorem[style=yellowbox,name=Warning,sibling=thm]{warn}
\declaretheorem[style=yellowbox,name=Warning,numbered=no]{warn*}
%

% PROOF FORMATTING
\renewcommand\qedsymbol{$\blacksquare$}
\newenvironment{innerproof}{\renewcommand{\qedsymbol}{$\square$}\proof}{\endproof}
%

% CUSTOM COMMANDS
% basic %
\newcommand{\Mod}[1]{\ (\mathrm{mod}\ #1)}
\newcommand{\floor}[1]{\left\lfloor{#1}\right\rfloor}
\newcommand{\ceil}[1]{\left\lceil{#1}\right\rceil}
\newcommand{\norm}[1]{\left\lVert{#1}\right\rVert}

% logic %
\newcommand*\xor{\oplus}
\newcommand{\all}{\forall}
\newcommand{\bland}{\bigwedge}
\newcommand{\blor}{\bigvee}

% matrices %
\newcommand\aug{\fboxsep=- \fboxrule\!\!\!\fbox{\strut}\!\!\!}\makeatletter 

% sets %
\newcommand{\CC}{\mathbb{C}}
\newcommand{\NN}{\mathbb{N}}
\newcommand{\QQ}{\mathbb{Q}}
\newcommand{\RR}{\mathbb{R}}
\newcommand{\ZZ}{\mathbb{Z}}

% probability stuff %
\newcommand{\E}{\mathbb{E}}
\newcommand{\Var}{\mathrm{Var}}
\newcommand{\var}{\mathrm{Var}}
\newcommand{\cov}{\mathrm{cov}}
\newcommand{\corr}{\mathrm{Corr}}

% linalg stuff %
\DeclareMathOperator*{\Span}{\mathrm{Span}}
\DeclareMathOperator*{\Null}{\mathrm{Null}}
\DeclareMathOperator*{\Range}{\mathrm{Range}}
\DeclareMathOperator*{\vspan}{\mathrm{span}}
\DeclareMathOperator*{\vnull}{\mathrm{null}}
\DeclareMathOperator*{\vrange}{\mathrm{range}}
\newcommand{\innerproduct}[2]{\left\langle{#1}, {#2}\right\rangle}
\DeclareMathOperator*{\proj}{\mathrm{proj}}

% set theory stuff %
\DeclareMathOperator*{\codom}{\mathrm{codom}}
\DeclareMathOperator*{\dom}{\mathrm{dom}}
\DeclareMathOperator*{\ran}{\mathrm{ran}}
\DeclareMathOperator*{\fld}{\mathrm{fld}}
\newcommand{\image}[1]{\left\llbracket {#1} \right\rrbracket}

% bracketing %
\newcommand{\pr}[1]{\left( {#1} \right)}
\newcommand{\br}[1]{\left[ {#1} \right]}
\newcommand{\brc}[1]{ \left\{  {#1} \right\}}
\newcommand{\ang}[1]{\langle {#1} \rangle}
%


% CUSTOM HOMEWORK
\NewDocumentEnvironment{hw}{m o}{\IfNoValueTF{#2}{
		% if hw:
		\renewcommand*{\thehomework}{\thesubsection}\setcounter{subsection}{#1}\begin{homework}}
		{% else, ahw:
			\renewcommand*{\thehomework}{\thesubsection\alph{homework}}\setcounter{subsection}{#1}\setcounter{homework}{\inteval{#2-1}}\begin{homework}}}
		{\end{homework}}
%

% MISC. RENAMING
\renewcommand*{\thehomework}{\thesubsection}
\renewcommand{\listtheoremname}{Problems}
%

% LIST OF PROBLEM SETUP
%solution by mrclrchtr: https://tex.stackexchange.com/questions/475799/how-can-i-change-the-format-of-an-entry-in-listoftheorems-of-thmtools

% reformats (Name) to ''Name''
\renewcommand\thmtformatoptarg[1]{ ''#1''}

% swaps number with theorem name
% swaps hw
\makeatletter
\def\ll@hw{%
	\thmt@thmname~
	\protect\numberline{\csname the\thmt@envname\endcsname}%
	\ifx\@empty
	\thmt@shortoptarg
	\else
	\protect\thmtformatoptarg{\thmt@shortoptarg}
	\fi
}
\makeatother

% swaps ahw
\makeatletter
\def\ll@ahw{%
	\thmt@thmname~
	\protect\numberline{\csname the\thmt@envname\endcsname}%
	\ifx\@empty
	\thmt@shortoptarg
	\else
	\protect\thmtformatoptarg{\thmt@shortoptarg}
	\fi
}
\makeatother

% changes spacing
\makeatletter
\renewcommand*{\numberline}[1]{\hb@xt@3em{#1}}
\makeatother

%


% TITLE PAGE
\newcommand{\mytitle}[2]{%
	\title{#1}
	\author{Michael Pham}
	\date{#2}
	\maketitle
	\newpage
	\listoftheorems
	\newpage
}
%

%
%END_FOLD
%

\begin{document}
	\mytitle{Math 135: Homework 4}{Spring 2024}
	
	\setcounter{section}{2}
	\section{Relations and Functions}
	\begin{hw}{32}[1]
		Show that $R$ is symmetric if and only if $R^{-1} \subseteq R$.
	\end{hw}
	\begin{solution}
		First, we begin with the forward direction.
		\begin{innerproof}
			To begin with, suppose that $R$ is symmetric.
			
			Then, for some element $\ang{y,x} \in R^{-1}$, we note then that $\ang{x,y} \in R$. However, because $R$ is symmetric, it follows then that $\ang{y,x} \in R$ as well, thus showing that $R^{-1} \subseteq R$.
		\end{innerproof}
	
		For the other direction, we proceed as follows:
		\begin{innerproof}
			Suppose that $R^{-1} \subseteq R$. Then, suppose that $\ang{x,y} \in R$. Then, this means that $\ang{y,x} \in R^{-1}$. But because $R^{-1} \subseteq R$, it follows then that $\ang{y,x} \in R$ as well. Thus, $R$ is symmetric.
		\end{innerproof}
	\end{solution}

	\begin{hw}{32}[2]
		Show that $R$ is transitive if and only if $R \circ R \subseteq R$.
	\end{hw}
	\begin{solution}
		First, we prove the forward direction.
		\begin{innerproof}
			Suppose that $R$ is transitive. Then, for some $\ang{x,z} \in R \circ R$, we see that there exists some $y$ such that $\ang{x,y} \in R$ and $\ang{y,z} \in R$. However, since $\ang{x,y} \in R$ and $\ang{y,z} \in R$, then by transitivity of $R$, $\ang{x,z} \in R$; therefore, $R \circ R \subseteq R$.
		\end{innerproof}
	
		For the other direction, we observe the following:
		\begin{innerproof}
			Suppose that $R \circ R \subseteq R$. Then, let us suppose that there exists some $\ang{x,y}$ and $\ang{y,z}$ in $R$. Then, we see that $\ang{x,z} \in R \circ R$. And since $R \circ R \subseteq R$, it follows then that $\ang{x,z} \in R$, thus showing that $R$ is transitive as desired.
		\end{innerproof}
	\end{solution}

	\begin{hw}{33}
		Show that $R$ is a symmetric and transitive relation if and only if $R = R^{-1} \circ R$.
	\end{hw}
	\begin{solution}
		To begin with, we proceed with the forward direction.
		\begin{innerproof}
			Suppose that $R$ is a symmetric and transitive relation.
			
			Then, let us take some element $\ang{x,y} \in R$. Then, since $R$ is symmetric, we have that $\ang{y,x} \in R$. And by transitivity, we see that $\ang{x,x} \in R$ as well.
			
			Then, since $\ang{y,x} \in R$, then this means that $\ang{x,y} \in R^{-1}$. From here, note that $\ang{x,x} \in R$ and $\ang{x,y} \in R^{-1}$, so we have that $\ang{x,y} \in R^{-1} \circ R$; $R \subseteq R^{-1} \circ R$.
			
			Then, for some $\ang{x,z} \in R^{-1} \circ R$, we note that there exists some $y$ such that $\ang{x,y} \in R$ and $\ang{y,z} \in R^{-1}$. Since $\ang{y,z} \in R^{-1}$, it follows that $\ang{z,y} \in R$, and thus by symmetry, $\ang{y,z} \in R$ as well. Then, we have $\ang{x,y} \in R$ and $\ang{y,z} \in R$; therefore, by transitivity, we have that $\ang{x,y} \in R$. Therefore, $R^{-1} \circ R \subseteq R$.
			
			Thus, we conclude $R = R^{-1} \circ R$.
		\end{innerproof}
	
		Now, we prove the backwards direction.
		\begin{innerproof}
			Suppose that $R = R^{-1} \circ R$.
			
			Then, suppose we had some $\ang{x,z} \in R$. Then, by our assumption, $\ang{x,z} \in R^{-1} \circ R$ too. This means that there exists a $y$ such that $\ang{x,y} \in R$ and $\ang{y,z} \in R^{-1}$. Then, $\ang{z,y} \in R$ and $\ang{y,x} \in R^{-1}$. So, by definition, we have that $\ang{z,x} \in R^{-1} \circ R = R$. Thus, we see that $R$ is indeed symmetric.
			
			For transitivity, suppose that $\ang{x,y} \in R$ and $\ang{y,z} \in R$. Then, we see that $\ang{z,y} \in R$ by symmetry, meaning that $\ang{y,z} \in R^{-1}$.
			
			Then, since $\ang{x,y} \in R$ and $\ang{y,z} \in R^{-1}$, then by definition, $\ang{x,z} \in R^{-1} \circ R = R$. Thus, $\ang{x,z} \in R$ as well, meaning that $R$ is transitive as desired.
		\end{innerproof}
	\end{solution}

	\begin{hw}{36}
		Assume that $f : A \rightarrow B$ and that $R$ is an equivalence relation on $B$. Define $Q$ to be
		\begin{equation*}
			\brc{\ang{x,y} \in A \times A : \ang{f(x), f(y)} \in R}.
		\end{equation*}
	
		Show that $Q$ is an equivalence relation on $A$.
	\end{hw}
	\begin{solution}
		First, reflexivity: suppose that $x \in A$ and $f(x) \in B$. Then, we see that $\ang{f(x), f(x)} \in R$, and thus $\ang{x,x} \in Q$ as well. Therefore, we see that $Q$ is reflexive as desired.
		
		Next, we will show that $Q$ is symmetric. Suppose that $\ang{x,y} \in Q$. Then, we observe that $\ang{f(x), f(y)} \in R$, so we know that $\ang{f(y), f(x)} \in R$ as well, and thus $\ang{y,x} \in Q$. Therefore, $Q$ is symmetric.
		
		For transitivity, suppose that $\ang{x,y} \in Q$ and $\ang{y,z} \in Q$. Then, we have that $\ang{f(x), f(y)}$ and $\ang{f(y), f(z)}$ are both in $R$. But it follows then that $\ang{f(x), f(z)} \in R$. Then, we have $\ang{x,z} \in Q$ as well, and thus $Q$ is transitive.
		
		Thus, $Q$ meets all requirements needed to be an equivalence relation on $A$.
	\end{solution}
	
	\begin{hw}{37}
		Assume that $\Pi$ is a partition of a set $A$. Define the relation $R_{\Pi}$ to be as follows:
		\begin{equation*}
			xR_{\Pi}y \iff (\exists B \in \Pi)(x \in B \land y \in B).
		\end{equation*}
	
		Show that $R_{\Pi}$ is an equivalence relation on $A$.
	\end{hw}
	\begin{solution}
		First, we check reflexivity. For some $x \in A$, we observe that there exists some $B \in \Pi$ such that $x \in B \in \Pi$. Then, by definition, this means that there exists a $B \in \Pi$ such that $x \in B$ and $x \in B$; thus, $\ang{x,x} \in R_{\Pi}$, meaning it's reflexive.
		
		For symmetry, we observe that for some $\ang{x,y} \in R_{\Pi}$, it means that there exists some $B \in \Pi$ such that $x \in B$ and $y \in B$. But this is the same as saying $\pr{\exists B \in \Pi}(y \in B \land x \in B)$ by commutativity; thus, $\ang{y,z} \in R_{\Pi}$ as well, making it symmetric.
		
		Finally, if $\ang{x,y} \in R_{\Pi}$ and $\ang{y,z} \in R_{\Pi}$, then $(\exists B \in \Pi)(x \in B \land y \in B) \land (\exists C \in \Pi)(y \in C \land z \in C)$. But then, this means that we have $y \in B \cap C$. Since each element in $\Pi$ are disjoint by definition, it must be then that $B = C$. Thus, we have that $x \in B$ and $z \in B$, meaning that $\ang{x,z} \in R_{\Pi}$, making it transitive.
		
		With all conditions met, we conclude that $R_{\Pi}$ is an equivalence relation as claimed.
	\end{solution}

	\begin{hw}{41}[1]
		Let $\RR$ be the set of real numbers, and $Q$ to be a relation on $\RR \times \RR$ by $\ang{u,v} Q \ang{x,y} \iff u+y = x + v$.
		
		Show that $Q$ is an equivalence relation on $\RR \times \RR$.
	\end{hw}
	\begin{solution}
		First, we show that $Q$ is reflexive. Suppose that we have some $u \in Q$. Then, we see that $u+u = 2u = u+u$. Thus, $\ang{u,u} Q \ang{u,u}$, making it reflexive as desired.
		
		Next, for symmetry, we observe that for $\ang{u,v}Q\ang{x,y}$, it follows that $u+y = x+v$. Then, this means that $x+v = u+y$ as well, and thus $\ang{x,y}Q\ang{u,v}$.
		
		For transitivity, suppose $\ang{u,v}Q\ang{w,x}$ and $\ang{w,x}Q\ang{y,z}$. Then, we have the following:
		\begin{align*}
			u + x  &= v + w \\
			w + z &= x + y \\
			u - y &= v - z \\
			u + z &= y + v
		\end{align*} 
	
		So by definition, we have that $\ang{u,v} Q \ang{z,y}$, meaning that $Q$ is transitive.
		
		Thus, we conclude that $Q$ is indeed an equivalence relation on $\RR \times \RR$.
	\end{solution}

	\begin{hw}{41}[2]
		Is there a function $G: (\RR \times \RR) / Q \rightarrow \pr{\RR \times \RR} / Q$ satisfying the equation:
		\begin{equation*}
			G\pr{\br{\ang{x,y}}_{Q}} = \br{\ang{x+2y, y+2x}}_{Q}?
		\end{equation*}
	\end{hw}
	\begin{solution}
		First, let us define a function $F : \RR \times \RR \rightarrow \RR \times \RR$ such that $\ang{x,y} \mapsto \ang{x+2y, y+2x}$.
		
		Now, we observe that for any $\ang{x,y} \in Q$, we have $\ang{a,b}Q\ang{c,d}$ such that $a+d = b + c$. Note then that $2a+2d = 2b+2c$.
		
		Then, we observe that $x+2y = \ang{a,b} + 2\ang{c,d} = \ang{a+2c, b+2d}$ and $y + 2x = \ang{c,d} + 2\ang{a,b} = \ang{c+2a, d+2b}$.
		
		With this in mind, we see that $a + 2c + d + 2b = c + 2a + b + 2d$; $\ang{x+2y, y+2x} \in Q$. In other words, $F\pr{\ang{a,b}} Q F\pr{\ang{c,d}}$ as well, meaning that $F$ is compatible with $Q$. And because $F$ is compatible with $Q$, it follows that there exists a $G$ that satisfies the equation as desired. 
	\end{solution}

	\begin{hw}{43}
		Assume that $R$ is a linear ordering on a set $A$. Show that $R^{-1}$ is also a linear ordering on $A$.
	\end{hw}
	\begin{solution}
		Suppose that $R$ is a linear ordering on a set $A$.
		
		Let $\ang{x,y} \in R^{-1}$ and $\ang{y,z} \in R^{-1}$. Then, we have that $\ang{y,x} \in R$ and $\ang{z,y} \in R$. Then, we have that $\ang{z,x} \in R$ by transitivity, and thus $\ang{x,z} \in R^{-1}$. Thus, we see that $R^{-1}$ is transitive.
		
		For trichotomy, we observe that for any $x,y \in A$, we have exactly one of $\ang{x,y}\in R$, $x = y$, or $\ang{y,x} \in R$ occurring. But then this means that we have only one of $\ang{y,x} \in R^{-1}$, $y = x$, or $\ang{x,y} \in R^{-1}$ happening. Thus, $R^{-1}$ satisfies trichotomy as well.
		
		Therefore, with all condition satisfied, we see that $R^{-1}$ is a linear ordering.
	\end{solution}

	\begin{hw}{44}
		Assume that $<$ is a linear ordering on a set $A$. Assume that $f : A \rightarrow A$ and that $f$ has the property that whenever $x < y$ then $f(x) < f(y)$.
		
		Show that $f$ is one-to-one and that whenever $f(x) < f(y)$, $x < y$.
	\end{hw}
	\begin{solution}
		To begin with, let $x_{1}, x_{2} \in A$ such that $f(x_{1}) = f(x_{2})$. Then, we have three options for $x_{1}$ and $x_{2}$:
		\begin{enumerate}
			\item $x_{1} < x_{2}$; in this case, we see that $f(x_{1}) < f(x_{2})$, which would contradict the fact that $f(x_{1}) = f(x_{2})$.
			\item $x_{2} < x_{1}$. But if this was the case, then we'd have $f(x_{2}) < f(x_{1})$, again leading to contradiction.
			\item $x_{1} = x_{2}$. This must be true by trichotomy of $<$.
		\end{enumerate}
	
		Thus, we see that if $f(x_{1}) = f(x_{2})$, then $x_{1} = x_{2}$; i.e. $f$ is one-to-one.
		
		Next, suppose that $f(x) < f(y)$. However, for the sake of contradiction, let us say that $x \not< y$. Then, we have two options:
		\begin{enumerate}
			\item $y < x$. But then this would mean that $f(y) < f(x)$, resulting in a contradiction.
			\item $y = x$. But in this case, we have that $f(x) = f(y)$; again, this is a contradiction.
		\end{enumerate}
	
		Therefore, by trichotomy, we conclude that $x < y$.
	\end{solution}

	\begin{hw}{45}
		Assume that $<_{A}$ and $<_{B}$ are linear orderings on $A$ and $B$ respectively. Define the binary relation $<_{L}$ on $A \times B$ by:
		\begin{equation*}
			\ang{a_{1}, b_{1}} <_{L} \ang{a_{2}, b_{2}} \iff a_{1} <_{A} a_{2} \lor \pr{a_{1} = a_{2} \land b_{1} <_{B} b_{2}}. 
		\end{equation*}
	
		Show that $<_{L}$ is a linear ordering on $A\times B$.
	\end{hw}
	\begin{solution}
		First, we show that this is indeed transitive.
		
		Suppose we have $\ang{a_{1}, b_{1}} <_{L} \ang{a_{2}, b_{2}}$ and $\ang{a_{2}, b_{2}} <_{L} \ang{a_{3}, b_{3}}$. Then, we have:
		\begin{align*}
			\pr{a_{1} <_{A} a_{2} \lor \pr{a_{1} = a_{2} \land b_{1} <_{B} b_{2}}} \land \pr{a_{2} <_{A} a_{3} \lor \pr{a_{2} = a_{3} \land b_{2} <_{B} b_{3}}}
		\end{align*}
	
		This can be rewritten as such:
		\begin{align*}
			&\pr{a_{1} <_{A} a_{2} \land a_{2} <_{A} a_{3}} \lor \pr{a_{1} <_{A} a_{2} \land \pr{a_{2} = a_{3} \land b_{2} <_{B} b_{3}}} \\ 
			&\lor \pr{\pr{a_{1} = a_{2} \land b_{1} <_{B} b_{2}} \land a_{2} <_{A} a_{3}} \lor \pr{\pr{a_{1} = a_{2} \land b_{1} <_{B} b_{2}} \land \pr{a_{2} = a_{3} \land b_{2} <_{B} b_{3}}}
		\end{align*}
	
		But we note then that this is equivalent to:
		\begin{align*}
			\pr{a_{1} <_{A} a_{3}} \lor \pr{a_{1} <_{A} a_{3}} \lor \pr{a_{1} <_{A} a_{3}} \lor \pr{a_{1} = a_{3} \land b_{1} <_{B} b_{3}} 
		\end{align*}
		
		Or, in other words,
		\begin{equation*}
			a_{1} <_{A} a_{3} \lor (a_{1} = a_{3} \land b_{1} <_{B} b_{3}).
		\end{equation*}
	
		Then by definition, we see that indeed, $\ang{a_{1}, b_{1}} <_{L} \ang{a_{3}, b_{3}}$, thus making it transitive as desired.
		
		To show trichotomy, we observe that for any ordered pairs $\ang{a_{1}, b_{1}}$ and $\ang{a_{2}, b_{2}}$, since $A$ is a linear ordering, we must have one of the following:
		\begin{enumerate}
			\item $a_{1} <_{A} a_{2}$; in this case, we see that $\ang{a_{1}, b_{1}} <_{L} \ang{a_{2}, b_{2}}$.
			\item $a_{2} <_{A} a_{1}$; then, we have $\ang{a_{2}, b_{2}} <_{L} \ang{a_{1}, b_{1}}$.
			\item $a_{1} = a_{2}$. If so, then we have three more cases to consider.
		\end{enumerate}
	
		In the case that $a_{1} = a_{2}$, then since $B$ is a linear ordering, we have:
		\begin{enumerate}
			\item $b_{1} <_{B} b_{2}$; in this case, we see that $\ang{a_{1}, b_{1}} <_{L} \ang{a_{2}, b_{2}}$.
			\item $b_{2} <_{B} b_{1}$; then, we have $\ang{a_{2}, b_{2}} <_{L} \ang{a_{1}, b_{1}}$.
			\item $b_{1} = b_{2}$. Since this is the case, we have that $\ang{a_{1}, b_{1}} = \ang{a_{2}, b_{2}}$, and we see that $\ang{a_{1}, b_{1}} \not<_{L} \ang{a_{2}, b_{2}}$.
		\end{enumerate}
	
		Thus, we see that $<_{L}$ satisfies trichotomy, and therefore is a linear ordering.
	\end{solution}

	\begin{hw}{30}[1]
		This is a homework question.
	\end{hw}
	\begin{solution}
		This is a solution.
		
		\begin{innerproof}
			This is some proof of another statement used in the overall solution.
		\end{innerproof}
	
		And this is the end of the solution.
	\end{solution}
\end{document}