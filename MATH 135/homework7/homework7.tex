\documentclass{article}
\usepackage{homework}
\usepackage{macros}
\usepackage{amsmath}

\graphicspath{{C:/Users/Michael/Pictures/}}

%% LIST OF PROBLEMS SETUP %%
\renewcommand\thmtformatoptarg[1]{:\enspace#1}
\makeatletter
\def\ll@homework{
	\thmt@thmname~
	\protect\numberline{\csname the\thmt@envname\endcsname}%
	\ifx\@empty
	\thmt@shortoptarg
	\else
	\protect\thmtformatoptarg{\thmt@shortoptarg}
	\fi
}
\makeatother

\makeatletter
\renewcommand*{\numberline}[1]{\hb@xt@3em{#1}}
\makeatother	

%% RENEW TITLE PAGE %%
\renewcommand{\mytitle}[2]{%
	\title{#1}
	\author{Michael Pham}
	\date{#2}
	\maketitle
	\newpage
	\listoftheorems
	\newpage
}

\begin{document}
\mytitle{Math 135: Homework 7}{Spring 2024}

\setcounter{section}{5}
\section{Cardinal Numbers and the Axiom of Choice}
\begin{hw}{1}[0][0]
	Show that the equation
	\begin{equation*}
		f(m,n) = 2^{m} (2n + 1) - 1
	\end{equation*}
defines a one-to-one correspondence between $\omega \times \omega$ and $\omega$.
\end{hw}
\begin{solution}
	We will want to show that $f(m,n)$ is both one-to-one and onto.
	
	First, we will show that it is one-to-one. Suppose that we have $f(m, n) = f(m', n')$. We want to show then that $(m,n) = (m', n')$.
	
	To do this, we observe the following:
	\begin{align*}
		f(m,n) &= 2^{m}(2n + 1) - 1 \\
		f(m', n') &= 2^{m'}(2n' + 1) - 1
	\end{align*}
	
	Then, we have:
	\begin{align*}
		f(m,n) &= f(m', n') \\
		2^{m}(2n + 1) - 1 &= 2^{m'}(2n' + 1) - 1 \\
		2^{m-m'}(2n+1) &= 2n' + 1
	\end{align*}

	From here, we note that for any $n \in \omega$, $2n+1$ and $2n' + 1$ are both odd numbers; they can't have $2$ as one of their factors. In other words, we require for $2^{m-m'} = 2^{0} = 1$ for the equality above to be true.
	
	Then, this yields us:
	\begin{align*}
		m-m' &= 0 \\
		m &= m'
	\end{align*}

	Furthermore, since this is the case, we have:
	\begin{align*}
		2n+1 &= 2n' + 1 \\
		2n - 2n' &= 0 \\
		2(n-n') &= 0 \\
		n-n' &= 0 \\
		n &= n'
	\end{align*}

	Thus, we have $m = m'$ and $n = n'$; in other words, we have that if $f(m,n) = f(m', n')$, then $(m,n) = (m', n')$ as desired. Thus, $f$ is indeed one-to-one.
	
	Next, to show onto, we want to show that for any $k \in \omega$, there exists $(m, n) \in \omega \times \omega$ such that $f(m,n) = k$.
	
	For the case of $k = 0$, we observe that if we let $m = n = 0$, then we have:
	\begin{align*}
		2^{0}(2(0) + 1) - 1 &= 1(0 + 1) - 1 \\
		&= 1 - 1 \\
		&= 0
	\end{align*}

	So, there exists $m,n$ such that $f(m,n) = k = 0$.
	
	And for $k = 1$, we observe that if we let $m = 1$ and $n = 0$, then we have:
	\begin{align*}
		2^{1}(2(0) + 1) - 1 &= 2(0 + 1) - 1 \\
		&= 2 - 1 \\
		&= 1
	\end{align*}

	So, there exists $m, n$ such that $f(m,n) = k = 1$.
	
	Now, for $k > 1$, we note that by the Fundamental Theorem of Arithmetic, $k$ has some unique prime factorisation:
	\begin{equation*}
		k = \prod_{i=1}^{j} p_i^{n_i},
	\end{equation*}
	where $p_1 < p_2 < \cdots < p_n$, and the $n_i$ are positive integers.
	
	Note that this prime factorisation will contain a $2^{m}$ term, where $m$ is non-negative (with $m = 0$ if $k$ is odd). Then, the product of the remaining primes in the unique factorisation of $k$ will be an odd number; i.e. there exists some $n \in \omega$ such that $2n + 1 = \prod_{i=2}^{j} p_i^{n_i}$.
	
	Now with this in mind, we first note that all $k' \in \omega$ must have some unique prime factorisation which we can rewrite as $k' = 2^{m}(2n + 1)$, for some $m, n \in \omega$.
	
	And if this is the case, then we have that for all $k \in \omega$, we have $k = k' - 1 = 2^{m}(2n + 1) - 1$. Thus, we have shown that for all $k \in \omega$, there exists $(m,n) \in \omega \times \omega$ such that $f(m,n) = k$. Thus, $f$ is indeed onto.
	
	Therefore, we can conclude that $f$ is a one-to-one correspondence between $\omega \times \omega$ and $\omega$.
\end{solution}

\begin{hw}{2}[0][0]
	Show that in Fig. 32 we have:
	\begin{align*}
		J(m,n) &= \br{1 + 2 + \cdots + (m+n)} + m \\
		&= \frac{1}{2}\br{(m+n)^{2} + 3m + n}
	\end{align*}
\end{hw}
\begin{solution}
	\begin{comment}
		To begin with, let us define $I: \omega \times \omega \rightarrow \omega \times \omega$ to be as follows:
		\begin{equation*}
			\ang{m,n} \mapsto \begin{cases}
				\ang{m+1, n-1} & n \neq 0 \\
				\ang{0, m+1} & n = 0
			\end{cases}
		\end{equation*}
		
		Then, let $f: \omega \rightarrow \omega \times \omega$ to be defined as:
		\begin{align*}
			f(0) &= \ang{0,0} \\
			f(n^{+}) &= I(f(n))
		\end{align*} 
		
		Now, for $J$ to indeed be a bijection, we require that the following holds:
		\begin{enumerate}
			\item $J \circ f = 1_\omega$
			\item $f \circ J = 1_{\omega \times \omega}$
		\end{enumerate}
		
		First, we will show that $J\circ f = 1_\omega$.
		\begin{innerproof}
			We proceed by induction on $n$.
			
			Observe that $(J \circ f)(0) = 0$. So, we now suppose that our claim holds for $n$, and we look at $n^{+}$.
			
			We observe that $(J \circ f)(n^{+}) = J\circ I(f(n))$. Then, for $b \neq 0$ we have:
			\begin{align*}
				J \circ I(f(n)) &= g(I(\ang{a,b})) \\
				&= \dfrac{1}{2}\pr{ \pr{a+1+b-1}^{2} + 3(a+1) + (b-1)} \\
				&= J(\ang{a,b}) + 1 \\
				&= n + 1 \\
				&= n^{+}
			\end{align*}
			
			And if $b = 0$, then we have:
			\begin{align*}
				g(I(\ang{a,b})) &= g(\ang{0, a+1}) \\
				&= \dfrac{1}{2}\pr{ (a+1)^{2} + 3(0) + (a+1)} \\
				&= \dfrac{1}{2}\pr{a^{2} + a + 1} \\
				&= g\pr{\ang{0,a}} + 1 \\
				&= n+1\\
				&= n^{+}
			\end{align*}
			
			Thus, we conclude that $J \circ f = 1_\omega$ as desired.
		\end{innerproof}
		
		For the other direction, proceed as follows:
		\begin{innerproof}
			We induct on $n$.
			
			We see that for $n = 0$, we have $(f \circ J)(0, 0) = 0$.
			
			Then, suppose that our claim holds for $n$. So, we now check $n^{+}$.
			
			Observe that:
			\begin{align*}
				(f \circ J)(0, n^{+}) &= (f \circ J)(0, n+1) \\
				&= f(J(0, n+1)) \\
				&= f\pr{ \dfrac{1}{2}\br{ \pr{n+1}^{2} + 3(0) + n+1} } \\
				&= f\pr{\dfrac{1}{2}}
			\end{align*}
		\end{innerproof}
	\end{comment}
	We will proceed by showing that $J$ is both one-to-one and onto.
	
	First, we note that in the $0^{th}$ column of the diagram, each of the entry corresponds to the sum $\sum_{i=0}^{k} i$, where $k$ is the $k^{th}$ row of the entry, starting at $k = 0$. This column is circled in blue below:
	
	\begin{center}
		\includegraphics[scale=0.5]{135_hw7_omegatimesomega}
	\end{center}
	 
	 Now, we will prove injectivity. To do this, we will want to show that if $\ang{m,n} \neq \ang{m', n'}$, then it follows that $J(m,n) \neq J(m', n')$.
	 
	 Now, we note that for $\ang{a,b}$, we have $a+b = k$. With this in mind, we observe that if $\ang{m,n} \neq \ang{m', n'}$, then this means that $m+n = k \neq k' = m'+n'$.
	 
	 With this in mind then, without loss of generality we will assume that $m+n < m' + n'$. In other words, $k < k'$.
	 
	 Now, we observe that if $k < k'$, then it follows that $k+1 \leq k'$.
	 
	 Next, referring back to the diagram, we note that $m+n = k$ represents each diagonal. For example, if $m+n = 2$, then it will be the second diagonal (which has values $3,4,5$).
	 
	 From here, we observe that for $m+n = k$, the minimum value that $J(m,n)$ can be will be the first value of the $k^{th}$ diagonal. In other words, it'll be $\frac{1}{2}k(k+1)$. Note that this is $J(0, k)$
	 
	 On the other hand, we observe that the maximum value that $J(m,n)$ can be will be at the bottom of the diagonal; in other words, it'll be $\frac{1}{2}k(k+1) + k$. This will be equal to $J(k, 0)$
	 
	 Now, for injectivity, we want to show that for $k < k'$, we have that $J(m,n) < J(m', n')$. In other words, the maximum value of $J(m,n)$ will be less than the minimum value of $J(m',n')$.
	 
	 To do this, we note then that we have for $m+n = k$ and $m'+n' = k'$ where $k < k'$ (i.e. $k+1 \leq k'$):
	 \begin{align*}
	 	J(m', n') &\geq J(0, k') \\
	 	&= \frac{1}{2}\br{ k'(k'+1)} \\
	 	&\geq \frac{1}{2}\br{ (k+1)(k+2)} \\
	 	&= \frac{1}{2}k^{2} + \frac{3}{2}k + 1 \\
	 	&= \frac{1}{2}k^{2} + \frac{1}{2}k + \frac{2}{2}k + 1 \\
	 	&= \frac{1}{2}k(k+1) + k + 1 \\
	 	&> \frac{1}{2}k(k+1) + k \\
	 	&= J(k, 0) \\
	 	&\geq J(m,n)
	 \end{align*}
 
 	In other words, we see that, indeed, $J(m',n') > J(m,n)$. Following through with similar steps, we can then also show that if $m+n = k > m'+n' = k'$, then $J(m,n) > J(m',n')$. In other words, we have shown that if $\ang{m,n} \neq \ang{m', n'}$, then $J(m,n) \neq J(m', n')$; $J$ is injective as desired.
 	
 	Next, to show surjectivity, we observe that for every $y = J(m,n)$, we note that $y$ will be on some $k^{th}$ diagonal of the diagram. Then, we can do the following:
 	\begin{enumerate}
 		\item We let $m = y - \frac{1}{2}k(k+1)$.
 		\item We let $n = k - m$.
 	\end{enumerate}
 
 	Thus, we observe then that:
 	\begin{align*}
 		J(m,n) &= \frac{1}{2}\br{(m+n)^{2} + 3m + n} \\
 		&= \frac{1}{2}\br{(m + k - m)^{2} + 3m + (k-m)} \\
 		&= \frac{1}{2}\br{k^{2} + 2m + k} \\
 		&= \frac{1}{2}\br{k^{2} + 2\pr{y - \frac{1}{2}k(k+1)} + k} \\
 		&= \frac{1}{2}\br{k^{2} + 2y -k^{2} - k + k} \\
 		&= \frac{1}{2}\br{2y} \\
 		&= y
 	\end{align*}
 
 	Thus, we see that, indeed, for every $y \in \omega$, there exists some $\ang{m,n} \in \omega \times \omega$ such that $J(m,n) = y$.
 	
 	In other words, $J$ is surjective.
 	
 	Thus, we conclude that, indeed, $J(m,n)$ is a one-to-one correspondence between $\omega \times \omega$ and $\omega$ as desired.
\end{solution}

\begin{hw}{3}[0][0]
	Find a one-to-one correspondence between the open unit interval $(0,1)$ and $\RR$ that takes rationals to rationals and irrationals to irrationals.
\end{hw}
\begin{solution}
	We can construct a function as follows:
	\begin{equation*}
		f(x) = \begin{cases}
			\dfrac{1}{x} - 2 & 0 < x \leq \dfrac{1}{2} \\
			\dfrac{1}{x-1} + 2 & \dfrac{1}{2} < x < 1
		\end{cases}
	\end{equation*}

	We note then that every rational will get mapped to a rational, whereas every irrational will get mapped to an irrational, by this function.
	
	To check that it's a bijection, we see that if $f(x) = f(y)$, then:
	\begin{align*}
		\frac{1}{x} - 2 &= \frac{1}{y} - 2 \\
		y - 2xy &= x - 2xy \\
		y &= x
	\end{align*}

	Or, we have:
	\begin{align*}
		\frac{1}{x-1} + 2 &= \frac{1}{y-1} + 2 \\ (y-1) + 2(x-1)(y-1) &= (x-1) + 2(x-1)(y-1) \\
		y &= x
	\end{align*}

	In either cases, we have $x = y$, so $f$ is indeed injective.
	
	And to prove surjectivity, we see that if $y \geq 0$, then:
	\begin{align*}
		y &= \dfrac{1}{x} - 2 \\
		y + 2 &= \dfrac{1}{x} \\
		x &= \dfrac{1}{y+2}
	\end{align*}

	And we see that
	\begin{align*}
		f(x) &= \frac{1}{\dfrac{1}{y+2}} - 2 \\
		&= y + 2 - 2 \\
		&= y
	\end{align*}

	And we see that if $y < 0$, then we have:
	\begin{align*}
		y &= \frac{1}{x-1} + 2 \\
		y - 2 &= \dfrac{1}{x-1} \\
		x &= \dfrac{1}{y-2} + 1
	\end{align*}

	And so we see:
	\begin{align*}
		f(x) &= \frac{1}{\dfrac{1}{y-2} +1 - 1} + 2 \\
		&= \dfrac{1}{\dfrac{1}{y-2}} + 2 \\
		&= y - 2 + 2 \\
		&= y
	\end{align*}

	So, we see that for every $y \in \RR$, there exists some $x \in \pr{0,1}$ such that $f(x) = y$.
	
	Therefore, we see that $f$ is bijective as desired. 
\end{solution}

\begin{hw}{4}[0][0]
	Construct a one-to-one correspondence between the closed unit interval
	\begin{equation*}
		\br{0,1} = \brc{x \in \RR : 0 \leq x \leq 1},
	\end{equation*}
	and the open unit interval $(0,1)$.
\end{hw}
\begin{solution}
	We can construct the following function, where $n \in \omega$:
	\begin{equation*}
		f(x) = \begin{cases}
			\dfrac{1}{2} & x = 0 \\
			\dfrac{1}{2^{n+2}} & x = \dfrac{1}{2^{n}} \\
			x & \text{otherwise}
		\end{cases}
	\end{equation*}
	
	To check injectivity, we observe that if $f(x) = f(y)$, then, either both $f(x) = \frac{1}{2}$ and $f(y) = \frac{1}{2}$, in which case $x = y = 0$.

	Or:
	\begin{align*}
		\frac{1}{2^{n+2}} &= \frac{1}{2^{m+2}} \\
		2^{m+2} &= 2^{n+2} \\
		m+2 &= n+2 \\
		m &= n \\
	\end{align*}
	meaning that $x = \frac{1}{2^{m}} = \frac{1}{2^{n}} = y$.

	Or, $x = f(x) = f(y) = y$. 

	In all cases, we see that $x = y$.
	
	Then, to show surjectivity, we observe that if $y \in (0,1)$, if $y = \frac{1}{2}$, then we let $x = 0$ for $f(x) = y$.
	
	If $y$ is a negative power of two which is not $2^{-1}$, then we can simply let $x = 4y$.
	
	And if $y$ is otherwise, we let $x = y$.
	
	Thus, for every $y$ we see there exists an $x$ such that $f(x) = y$. Therefore, $f$ is surjective.

	Thus, we see that this is indeed bijective. 
\end{solution}

\begin{hw}{6}[0][0]
	Let $\kappa$ be a nonzero cardinal number. Show that there does not exist a set to which every set of cardinality $\kappa$ belongs.
\end{hw}
\begin{solution}
	We can let $\kappa = 1$. Then, suppose for the sake of contradiction that there exists a set to which every set of cardinality $\kappa = 1$ belongs to. 
	
	We note that a set with cardinality $\kappa = 1$ is a singleton. And we have proven in a previous homework that the set of all singletons cannot exist. In other words, a set to which every set of cardinality $\kappa = 1$ belongs to doesn't exist.
\end{solution}

\begin{hw}{7}[0][0]
	Assume that $A$ is finite and $f : A \rightarrow A$. Show that $f$ is one-to-one iff $\ran f = A$.
\end{hw}
\begin{solution}
	First, suppose that $f$ is one-to-one. Then, let $B = f\image{A}$. Then, we have $B \subseteq A$ and $B \approx A$.
	
	However, note that if $B \subset A$, then $\mathrm{card} B < \mathrm{card} A$. But, we have that $B \approx A$, so it follows that $B = A$. Therefore, we see that $f$ is indeed onto; i.e. $\ran f = A$.
	
	On the other hand, suppose that $\ran f = A$. Now, since $A$ is finite, it has a cardinality of $n$, for some $n \in \omega$.
	
	From here, let's suppose for the sake of contradiction that there exists some $y \in A$ such that for $x \neq x'$, we have $f(x) = f(x') = y$.
	
	Then, there's $n-2$ elements left in the domain and $n-1$ in the range that need to be paired with each other. However, since $f$ is a function, an element in $\dom f$ can't be mapped to two elements in $\ran f$. By the Pigeonhole Principle then, there's at least one element in $A$ which doesn't have a pre-image.
	
	Thus, we have a contradiction. So, we conclude that $f$ is indeed one-to-one.
\end{solution}

\begin{hw}{13}[0][0]
	Show that a finite union of finite sets is finite.
\end{hw}
\begin{solution}
	We can proceed by induction.
	
	First, we observe that for a set $A$ with cardinality $0$, we have then that $A = \emptyset$. Then, $\bigcup A = \emptyset$, so, indeed, we have that $\bigcup A$ is finite as well.
	
	Next, suppose that our claim holds for set $A$ with cardinality $n$.
	
	Now, we look at $A$ whose cardinality is $n^{+}$. Observe then that because $A$ is finite, it follows that there exists a bijection between $A$ and $n^{+}$, and thus some bijective function $f : n^{+} \rightarrow A$.
	
	Then, we have:
	\begin{align*}
		\bigcup A &= \bigcup_{k \in n^{+}} f(k) \\
		&= \bigcup_{k \in n} f(k) \cup f(n)
	\end{align*}

	By our induction hypothesis, we have that $\bigcup_{k \in n} f(k)$ is finite. And since $f(n)$ is also finite, we thus have that $\bigcup_{k \in n} f(k) \cup f(n)$ is finite too.
	
	Therefore, by induction, we conclude that the claim holds as desired.
\end{solution}

\begin{hw}{14}[0][0]
	Define a permutation of $K$ to be any one-to-one function from $K$
	onto $K$. We can then define the factorial operation on cardinal numbers by the equation
	\begin{equation*}
		\kappa ! = \mathrm{card} \brc{f : f \text{ is a permutation of $K$}}
	\end{equation*}
	where $K$ is any set of cardinality $\kappa$. Show that $\kappa !$ is well defined.
\end{hw}
\begin{solution}
	Suppose we have sets $K_0$ and $K_1$. Let $\card K_0 = \kappa = \card K_1$. 
	
	Then, because the cardinalities of $K_0$ and $K_1$ are the same, we can thus construct a bijection between them.
	
	To show that $\kappa!$ is well-defined, we will have to show then that there exists a bijection between the set of permutations of $K_0$ (which we will denote at $K_0'$) and permutations of $K_1$ (which we denote as $K_1$').
	
	Then to do this, we recall that there exists a bijection $g$ between $K_0$ and $K_1$. So, for each permutation $f$ of $K_1$, we can first send this permutation to $K_0$ with our bijection $g$, which we then permute. After, we can send the permutation of $K_0$ back to $K_1$ using $g^{-1}$.
	
	Thus, we observe then that there exists a bijection between the set of permutations $f$ of $K_0$ and of $K_1$; i.e., we have shown that $\kappa!$ is well-defined.
\end{solution}

\end{document}