\documentclass{article}
\usepackage{homework}
\usepackage{macros}

%% LIST OF PROBLEMS SETUP %%
\renewcommand\thmtformatoptarg[1]{:\enspace#1}
\makeatletter
\def\ll@homework{
	\thmt@thmname~
	\protect\numberline{\csname the\thmt@envname\endcsname}%
	\ifx\@empty
	\thmt@shortoptarg
	\else
	\protect\thmtformatoptarg{\thmt@shortoptarg}
	\fi
}
\makeatother

\makeatletter
\renewcommand*{\numberline}[1]{\hb@xt@3em{#1}}
\makeatother	

%% RENEW TITLE PAGE %%
\renewcommand{\mytitle}[2]{%
	\title{#1}
	\author{Michael Pham}
	\date{#2}
	\maketitle
	\newpage
	\listoftheorems
	\newpage
}

\begin{document}
\mytitle{Math 135: Homework 12}{Spring 2024}

\setcounter{section}{6}
\section{Orderings and Ordinals}
\begin{hw}{26}[0][0]
	Show that every ordinal $\alpha$ is grounded, and that $\rank \alpha = \alpha$.
\end{hw}
\begin{solution}
	To begin with, we note by the Axiom of Regularity that every set is grounded. Then, it follows that since $\alpha$ is a set, it must thus be grounded as well.
	
	Now, we know then that for every ordinal $\alpha$, we have $\alpha \subseteq V_{\rank\pr{\alpha}}$ and $\alpha \in V_{ \pr{\rank \alpha}^{+}}$. We now want to show that $\rank \alpha = \alpha$. To do this, we will proceed by transfinite induction.
	
	Suppose that for ordinal $\alpha$, we have that $\rank \beta = \beta$ for all $\beta < \alpha$.
	
	First, we observe that for $\alpha = 0$, we have that $\alpha = 0 = \emptyset \subseteq \emptyset = V_0$. Also, we note that $V_{0^{+}} = \mathscr{P} V_0 = \brc{\emptyset}$. So, we have $\alpha = 0 \in \brc{\emptyset} = \mathscr{P} V_0 = V_{0^{+}}$. And since there exists no $\beta$ such that $\beta \in 0 = \emptyset$, we see that, indeed, $\alpha = 0$ is the least ordinal such that $\alpha \subseteq V_\alpha$. In other words, $\alpha = \rank \alpha$ for $\alpha = 0$.
	
	Now, we look at the case of $\alpha$ is the successor ordinal. That is, $\alpha = \beta^{+}$ for some $\beta < \alpha$. From here, we see that $\alpha = \beta^{+} = \beta \cup \brc{\beta}$.	We note that if $\gamma \in \alpha$, then either $\gamma \in \beta$ or $\gamma = \beta$.
	
	Now, we observe that
	\begin{align*}
		\rank \alpha &= \bigcup \brc{(\rank \gamma)^{+} : \gamma \in \alpha} \\
		&= \bigcup \brc{\gamma^{+} : \gamma \in \alpha}
	\end{align*}

 	But we see then that $\rank \alpha$ is simply $\beta^{+} = \alpha$. In other words, $\rank \alpha = \alpha$.
	
	Now, if $\alpha$ is a limit ordinal, we note that $\alpha = \bigcup_{\beta < \alpha} \beta$. Then, from here, we have that
	\begin{align*}
		\rank \alpha &= \rank \bigcup_{\beta < \alpha} \beta \\
		&= \bigcup \brc{\pr{\rank \beta}^{+} : \beta < \alpha} \\
		&= \bigcup \brc{\beta^{+} : \beta < \alpha} \\
		&= \alpha
	\end{align*}

	Thus, we see that $\rank \alpha = \alpha$. So, by transfinite induction, we have proven our claim as desired.
\end{solution}

\begin{hw}{33}[0][0]
	Assume that $D$ is a transitive set. Let $B$ be a set with the property that for any $a$ in $D$,
	\begin{equation*}
		a \subseteq B \implies a \in B.
	\end{equation*}

	Show that $D \subseteq B$.
\end{hw}
\begin{solution}
	\begin{comment}
		Let us suppose for the sake of contradiction that $B \subset D$.
		
		Now, we note that since $D$ is transitive, it follows then that $\mathscr P D$ is also transitive.
		
		With this in mind, we observe that because $B \subset D$, it must be that $B \in \mathscr P D$.
		
		Then, we note that $B \subseteq \mathscr P D$ by transitivity of $\mathscr P D$.
	\end{comment}
	\begin{comment}
		First, we consider all $a \in D$ such that $a \subseteq B$. Then, for these $a$, we have that $a \in B$ as well by our assumption.
		
		Now, we note that since $D$ is a transitive set, we observe that for all $a' \in a \in D$, we have $a' \in D$ as well. And since $a' \in a$, we see then that as $a \subseteq B$, we have $a' \in a \subseteq B \implies a' \in B$.
		
		From here, let us consider the case where $a \not\subseteq B$. Then, it follows that $B \subset a$. We note then that for all $b \in B$, we have $b \in a \in D \implies b \in D$. In other words, we note that $B \subset D$.
		
		Now, we note that since $B \subset D$, this means then that for all $b \in B$, we have that $b \in D$.
		
		Furthermore, for all $b' \in b$, we have that $b' \in D$.
	\end{comment}
	\begin{comment}
	Suppose for the sake of contradiction that $B \subset D$. Then, we observe that $D \setminus B \neq \emptyset$. Then, by Regularity, we know then that there exists some $a \in (D \setminus B)$ such that $a \cap (D \setminus B) = \emptyset$.
	
	Now, with this in mind, we note that $a \in D$, and thus we have that $a \subseteq D$ by transitivity of $D$.
	
	Furthermore, we note that we must also have that $a \subseteq B$ so that $a \cap (D \setminus B) = \emptyset$.
	
	However, we note that by our assumption, we have then that $a \subseteq B \implies a \in B$. But we assumed that $a \in (D \setminus B)$, meaning that $a \not\in B$. Thus, we have a contradiction.
	
	So, we conclude that we must have $D \subseteq B$ as desired.
	\end{comment}
	Suppose that $D \subseteq B$ is false. Then, that means that there exists some element $a \in D$ such that $a \not\in B$.
	
	Now, because $a \in D$, then by transitivity, we note that for all $a' \in a$, we have that $a' \in D$. We note here that there exists some $a' \in a \in D \implies a' \in D$ such that $a' \not\in B$, or else we'd have that $a \subseteq B \implies a \in B$.
	
	So, let us define $D'$ to be the set of $a \in D$ such that $a \not\in B$. By definition, we observe then that $D' \setminus B \neq \emptyset$.
	
	\begin{comment}
		Furthermore, we note that $D'$ is in fact transitive: suppose $a \in D'$ and $a' \in a$. Since $a \in D'$, we note then that $a \in D$ and $a \not\in B$. Then, for $a \not\subseteq B$, there exists some $a' \in a$ such that $a' \not\in B$. We note that $a' \in D$ as well by transitivity of $D$, and thus we see that $a' \in D'$ as well.
	\end{comment}
	So, we note that there exists some $a \in D' \setminus B$ such that $a \cap D' \setminus B = \emptyset$. Then, we have the following:
	\begin{align*}
		a \cap \pr{D' \setminus B} &= (a \cap D') \setminus (a \cap B).
	\end{align*}

	From here, we note that for $a \in D'$, there exists some $a' \in a$ such that $a' \in D'$ as well by our observation from earlier. So, we note that $a \cap D' \neq \emptyset$, and contains all $a' \in a$ such that $a' \not\in B$.
	
	However, we note that for $(a \cap D') \setminus (a \cap B) = \emptyset$, we require for $(a \cap D') \subseteq (a \cap B)$.
	
	But this means then that for all $a' \in (a \cap D')$, we have that $a' \in (a \cap B) \implies (a' \in a \land a' \in B)$. But this is a contradiction.
	
	Therefore, we conclude that $D \subseteq B$ as desired.
\end{solution}

\begin{hw}{34}[0][0]
	Assume that
	\begin{equation*}
		\brc{x, \brc{x, y}} = \brc{u, \brc{u, v}}.
	\end{equation*}

	Show that $x = u$ and $y = v$.
\end{hw}
\begin{solution}
	We note that in order for $\brc{x, \brc{x, y}} = \brc{u, \brc{u, v}}$, we observe that for all $z \in \brc{x, \brc{x, y}}$, we must have $z \in \brc{u, \brc{u, v}}$ as well.
	
	Now, we note that since $x \in \brc{x, \brc{x, y}}$, we must have that $x \in \brc{u, \brc{u, v}}$ too.
	
	Then, either $x = u$ or $x = \brc{u, v}$. Suppose first that $x = \brc{u, v}$. Now, we note that since $\brc{x, y} \in \brc{x, \brc{x, y}}$, then we have that $\brc{x, y} \in \brc{u, \brc{u, v}}$.
	
	Again, we have two cases:
	\begin{enumerate}
		\item $\brc{x, y} = u$, or
		\item $\brc{x, y} = \brc{u, v}$.
	\end{enumerate}

	In the first case, if $\brc{x, y} = u$, we note then that $x \in \brc{x, y} = u \in \brc{u, v} = x$. However, this contradicts Regularity, and thus we conclude that we can't have $\brc{x, y} = u$.
	
	Now, if $\brc{x, y} = \brc{u, v}$, we observe then that $x = \brc{u, v} \in \brc{x, y} = \brc{u, v}$. And since we have $\brc{u, v} \in \brc{u, v}$, we see that this again contradicts Regularity.
	
	So, we conclude that we can't have $x = \brc{u, v}$.
	
	Then, we have $x = u$. Again, in this case, we have two possibilities for $\brc{x, y}$:
	\begin{enumerate}
		\item $\brc{x, y} = u$, or
		\item $\brc{x, y} = \brc{u, v}$.
	\end{enumerate}

	In the case where $\brc{x, y} = u$, we have then that $x \in \brc{x, y} = u = x \implies x \in x$. This again contradicts Regularity, and thus we must have $\brc{x, y} = \brc{u, v}$.
	
	Now, we see then that $x = u$ and $\brc{x, y} = \brc{u, v}$.
	
	Then, we note that $\brc{y} = \brc{x, y} \setminus \brc{x} = \brc{u, v} \setminus \brc{u} = \brc{v}$. So, we have $\brc{y} = \brc{v} \implies y = v$.
\end{solution}

\begin{hw}{39}[0][0]
	Prove that a set is an ordinal number iff it is a transitive set of transitive sets.
\end{hw}
\begin{solution}
	First, we will prove the forward direction.
	\begin{innerproof}
		Let us suppose that $\alpha$ is an ordinal number. Then, we note that for all $\beta \in \alpha$, $\beta$ is an ordinal.
		
		From here, we note that since ordinals are transitive, it follows then that $\alpha$ is a transitive set of transitive sets.
	\end{innerproof}

	For the other direction, we proceed as follows:
	\begin{innerproof}
		Denote $x$ to be a transitive set of transitive sets.
		
		Now, we let $\alpha$ be the least ordinal such that $\alpha \not\in x$; we know this exists due to the fact that there exists no set of all ordinals.
		
		Now, we observe that in the case where $x \subseteq \alpha$, then $x$ is a transitive set of ordinals; in other words, $x$ is itself an ordinal.
		
		If $x \not\subseteq \alpha$, we then note that $x \setminus \alpha \neq \emptyset$. Then, by Regularity, there exists some $y \in (x \setminus \alpha)$ such that $y \cap (x \setminus \alpha) = \emptyset$.
		
		Now, we note that $y \in x$, and so $y \subseteq x$ by transitivity of $x$. Furthermore, note that $y$ is transitive as well by our assumption.
		
		We also note then that $y \subseteq \alpha$. However, since $y$ is a transitive set of ordinals, we note that $y$ is thus an ordinal.
		
		From here, we see that $y \in x \setminus \alpha$, so it follows that $y \not\in \alpha$. But this means then that either $\alpha = y$ or $\alpha \in y$.
		
		In either cases, we note then that we get $\alpha \in x$, and thus a contradiction that $\alpha \not\in x$.
		
		Therefore, we conclude that, indeed, $x$ is an ordinal.
	\end{innerproof}

	Thus, we have proven both directions as desired.
\end{solution}

\end{document}