\documentclass{article}
\usepackage{homework}
\usepackage{macros}

%% LIST OF PROBLEMS SETUP %%
\renewcommand\thmtformatoptarg[1]{:\enspace#1}
\makeatletter
\def\ll@homework{
	\thmt@thmname~
	\protect\numberline{\csname the\thmt@envname\endcsname}%
	\ifx\@empty
	\thmt@shortoptarg
	\else
	\protect\thmtformatoptarg{\thmt@shortoptarg}
	\fi
}
\makeatother

\makeatletter
\renewcommand*{\numberline}[1]{\hb@xt@3em{#1}}
\makeatother	

%% RENEW TITLE PAGE %%
\renewcommand{\mytitle}[2]{%
	\title{#1}
	\author{Michael Pham}
	\date{#2}
	\maketitle
	\newpage
	\listoftheorems
	\newpage
}

\begin{document}
\mytitle{Math 135: Homework 10}{Spring 2024}

\setcounter{section}{6}
\section{Orderings and Ordinals}
\begin{hw}{1}[1][0]
	Assume that $<_A$ and $<_B$ are partial orderings on $A$ and $B$ respectively, and that $f$ is a function from $A$ into $B$ satisfying
	\begin{equation*}
		x <_A y \implies f(x) <_B f(y)
	\end{equation*}
	for all $x \in A$. 
	
	Can we conclude that $f$ is one-to-one?
\end{hw}
\begin{solution}
	$f$ isn't one-to-one.
	
	We shall construct a counterexample. First, let us define $A = \brc{0, 1, 2}$ and $\brc{0, 1} = B$. Then, we have the linear ordering $<_A$ to be $1 <_A 2$. On other hand, let $<_B$ be the usual ordering on $B$.
	
	Then, from here, let us define some function $F$ as follows:
	\begin{itemize}
		\item $F(0) = 0$
		\item $F(1) = 0$
		\item $F(2) = 1$.
	\end{itemize}

	We note then that $F$ indeed follows the constraints given in the problem statement.
	
	Then, because of this, we see then that $0 \neq_A 1$ but $F(0) = F(1)$. Thus, we see that injectivity doesn't hold.
\end{solution}

\begin{hw}{1}[2][0]
	Can we conclude that
	\begin{equation*}
		x <_A y \iff f(x) <_B f(y)?
	\end{equation*}
\end{hw}
\begin{solution}
	This is false.
	
	We note that $<_A$ and $<_B$ are partial orderings, meaning that \textit{at most} one of the three holds:
	\begin{enumerate}
		\item $x <_A y$, or
		\item $x =_A y$, or
		\item $y <_A x$.
	\end{enumerate}

	However, it is possible for elements in $A$ to not be related to one another. For example, let us consider the sets $A = \brc{0, 1, 2} = B$, where $<_A$ is $1 <_A 2$, but $0$ isn't related to the other elements. Also, let $<_B$ be the usual ordering.
	
	Then, from here, let us define some function $F$ as follows:
	\begin{itemize}
		\item $F(0) = 1$
		\item $F(1) = 1$
		\item $F(2) = 2$.
	\end{itemize}

	Then, we observe that while indeed $1 <_A 2$ and $F(1) <_B F(2)$, we note that while $F(0) <_B F(2)$, we don't have that $0 <_A 2$. Thus, we have found a counterexample.
\end{solution}

\begin{hw}{5}[0][0]
	Assume that $<$ is a well-ordering on $A$, and that $f : A \rightarrow A$ satisfies the following condition:
	\begin{equation*}
		x < y \implies f(x) < f(y)
	\end{equation*}
	for all $x$ and $y$ in $A$. Show that $x \leq f(x)$ for all $x \in A$.
\end{hw}
\begin{solution}
	Suppose for the sake of contradiction that there exists some $x \in A$ such that $f(x) < x$. In other words, let us suppose that the following set is nonempty:
	\begin{equation*}
		A' \coloneq \brc{a \in A : f(a) < a}.
	\end{equation*}

	Now, we note that $A' \subseteq A$. Furthermore, we note by well-ordering of $<$ and the fact that $A'$ is a non-empty subset of $A$, we know then that there exists some least element $x \in A'$.
	
	Then, we observe that $f(x) < x \implies f(f(x)) < f(x) < x$. In other words, we observe that $f(x) \in A'$ and that $f(x) < x$. However, this is a contradiction with the minimality of $x$.
	
	Thus, we conclude that for all $x \in A$, we have that $x \leq f(x)$ as desired.
\end{solution}

\begin{hw}{6}[0][0]
	Assume that $S$ is a subset of the real numbers that is well-ordered. Show that $S$ is countable.
\end{hw}
\begin{solution}
	Suppose that $S \subseteq \RR$ is well-ordered. In order to show that $S$ is countable, we will want to create an injection $f : S \rightarrow A$, where $A$ is some countable set.
	
	Then, let us consider the set $A \coloneq \brc{q_n : n \in \omega}$. Next, for any $s \in S$, we define its successor $s^{+}$ to be the next element in $s$. Note that if $s$ is a maximal element of $S$, we define $s^{+} = s + 1$.
	
	Now, with all of this in mind, note that since $s \neq s^{+}$, we observe that the set $\QQ \cap \pr{s, s^{+}}$ is non-empty, and thus there exists some least rational. So, for any $s \in S$, we let $q_s$ be the least rational in $A$ such that $s < q_s < s^{+}$. 
	
	Now, to show injectivity, let us suppose that for $q_a = q_b$, we have $a \neq b$. Without loss of generality, suppose then that $a < b$.
	
	Then, we observe that we have $a < q_a = q_b < a^{+}$, and $b < q_b = q_a$. Then, we note that combining these inequalities together, we have:
	\begin{equation*}
		a < b < q_a = q_b < a^{+}.
	\end{equation*}

	However, we note here that $a^{+}$ is defined to be the least element in $S$ such that $a < a^{+}$. But if we have $a < b$ and $b \in S$ as well, this contradicts with the minimality of $a^{+}$. In other words, it can't be that $a < b$. A similar argument follows to show why $b < a$ can't happen.
	
	Thus, we conclude that, indeed, $a = b$, and thus we have found an injection from $S$ to some countable set $A$. Thus, we conclude that $S$ is countable as well.
\end{solution}

\begin{hw}{7}[1][0]
	Let $C$ be some fixed set. Apply transfinite recursion to $\omega$, using for $\gamma(x, y)$ the formula
	\begin{equation*}
		y = C \cup \bigcup\bigcup \ran x.
	\end{equation*}

	Let $F$ be the $\gamma$-constructed function on $\omega$.
	
	Calculate $F(0)$, $F(1)$, and $F(2)$. Make a good guess as to what $F(n)$ is.
\end{hw}
\begin{solution}
	We observe that $F(0) = C \cup \bigcup\bigcup \ran (F\restriction \seg(0)) = C \cup \emptyset = C$.
	
	For $F(1) = C \cup \bigcup\bigcup \ran (F\restriction \seg(1)) = C \cup \bigcup\bigcup \brc{F(0)} = C \cup \bigcup\bigcup \brc{C} = C \cup \bigcup C$.
	
	And $F(2) = C \cup \bigcup\bigcup \ran(F \restriction \seg(2)) = C \cup \bigcup\bigcup \brc{F(0), F(1)} = C \cup \bigcup\bigcup \brc{C, C \cup \bigcup C} = C \cup \bigcup\pr{C \cup \bigcup C} = C \cup \bigcup C \cup \bigcup\bigcup C$.
	
	And we have $F(n) = C \cup \bigcup C \cup \cdots \cup \bigcup_1\bigcup_2\cdots\bigcup_n C$.
	
	And in fact, we note that this simply becomes $F(n) = C \cup \bigcup F(n-1)$.
	
	For a concrete formula for this, we can define $F$ as follows:
	\begin{align*}
		F(0) &= C \\
		F(n^{+}) &= C \cup \bigcup F(n).
	\end{align*}
\end{solution}

\begin{hw}{7}[2][0]
	Show that if $a \in F(n)$, then $a \subseteq F(n^{+})$.
\end{hw}
\begin{solution}
	We see that since $a \in F(n)$, then by definition of $F$, we note that $F(n^{+}) = C \cup \bigcup F(n)$. Then, by definition of union, we have that $a \in F(n)$ means that $a \subseteq \bigcup F(n)$. In other words, $a \subseteq F(n^{+})$ as desired.
\end{solution}

\begin{hw}{7}[3][0]
	Let $\overline{C} = \bigcup \ran F$. Show that $\overline C$ is a transitive set, and that $C \subseteq \overline C$.
\end{hw}
\begin{solution}
	First, we will show that $C \subseteq \overline C$.
	\begin{innerproof}
		To do this, we note that $\ran F = \brc{F(n) : n \in \omega}$. Then, we have that $\bigcup \ran F = \bigcup \brc{F(n) : n \in \omega}$. 
		
		From here, we note that $F(0) = C$. Then, we see that $C \in \ran F$. Then, by definition, we see then that $F(0) = C \subseteq \bigcup \ran F = \overline C$.
	\end{innerproof}
	
	To prove transitivity of $\overline C$, we want to show that for some element $c \in \overline C$ and some $c' \in c$, we have that $c' \in \overline C$.
	\begin{innerproof}
		To do this, let us take some element $c \in \overline C$ and some element $c' \in c$. We note that for $c \in \overline C$, this means that $c \in \bigcup \ran F$. In other words, there exists some $n \in \omega$ such that $c \in F(n) \in \ran F$.
		
		Now, we note that since $c \in F(n)$, then by definition of $F$, we have that $c \subseteq F(n^{+})$. But since $c' \in c$, we note then that $c' \in F(n^{+}) \in \ran F$. And this means then that $c' \in \bigcup \ran F$. Thus, we have shown that, indeed, $\overline C$ is transitive as desired.
	\end{innerproof}
\end{solution}

\begin{hw}{8}[0][0]
	Show that the subset axioms are provable from the other axioms.
\end{hw}
\begin{solution}
	We want to show that for some set $A$ and formula $\psi(x)$, $\brc{x \in A : \psi(x)}$ is indeed a set.
	
	To do this with replacement, we first define $\psi(x)$ be some formula, and then we define $\varphi(x, y)$ to be: 
	\begin{equation*}
		\varphi(x, y)\coloneq x = y \land \psi(x).
	\end{equation*}

	Then, we observe that the condition $x = y$ ensures that $\varphi$ behaves like a function, and thus we can apply the replacement schema. So, we have that for any set $A$, there exists some set $B$ such that $\forall y(y \in B \iff (\exists x \in A)\varphi(x, y))$.
	
	In other words, we have:
	\begin{equation*}
		\brc{y : (\exists x \in A)(x = y \land \psi(x))}.
	\end{equation*}

	Or, we can rewrite this as:
	\begin{equation*}
		\brc{x \in A : \psi(x)}.
	\end{equation*}

	And by the replacement axioms, we see that this is indeed a set. Thus, we have proven the desired result.
\end{solution}

\end{document}