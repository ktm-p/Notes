\documentclass{article}
%% PREAMBLE %%
%BEGIN_FOLD
%% PACKAGES
\usepackage{amsmath}
\usepackage{amssymb}
\usepackage{amsthm}
\usepackage{cabin} % section title font
\usepackage[default]{cantarell} % default font
\usepackage[shortlabels]{enumitem}
\usepackage{fancyhdr}
\usepackage{graphicx}
\usepackage{hyperref}
\usepackage{mathtools}
\usepackage[framemethod=TikZ]{mdframed}
\usepackage[scr]{rsfso} % power set symbol
\usepackage{stmaryrd}
\usepackage{tasks} % vaguely remember this being important for something...?
\usepackage{tikz} % diagrams
\usepackage{titlesec}
\usepackage{thmtools}
\usepackage{varwidth}
\usepackage{verbatim} % longer comments
\usepackage{xcolor}
\usepackage{xparse}
%%

%% COLOURS
\definecolor{darkgreen}{HTML}{19A514}
\definecolor{lightgreen}{HTML}{9DFF9A}
\definecolor{darkblue}{HTML}{3E5FE4}
\definecolor{lightblue}{HTML}{BCDEFF}
\definecolor{darkred}{HTML}{CC3333}
\definecolor{lightred}{HTML}{FFA9A9}
\definecolor{darkpurple}{HTML}{A933CD}
\definecolor{lightpurple}{HTML}{F0BAFF}
\definecolor{darkyellow}{HTML}{D2D22A}
\definecolor{lightyellow}{HTML}{FFFFAE}
\definecolor{hyperlinkblue}{HTML}{3366CC}
%%

%% PAGE SETUP
% BASIC %
\setlength\parindent{0pt} % paragraph indentation
\setlength{\parskip}{5pt} % spacing between paragraphs
\usepackage[margin=1in]{geometry} % margin size

% HEADER/FOOTER %
\pagestyle{fancy}
\fancyhf{}
\fancyfoot[R]{\thepage} % page number on bottom right
\fancyhead[R]{\textit{\leftmark}} % section title
\renewcommand{\headrulewidth}{0pt} % removing horizontal line at the top

% HYPERLINK FORMATTING %
\hypersetup{
	colorlinks,    
	linkcolor=hyperlinkblue,
	urlcolor=hyperlinkblue,
	pdftitle={...},
	pdfauthor={Michael Pham},
}
%%

%% ENVIRONMENTS STYLES
% SOLUTION ENVIRONMENT %
\newenvironment{solution}{\begin{proof}[Solution]}{\end{proof}}

% PURPLE BOX %
\declaretheoremstyle[
mdframed={
	backgroundcolor=lightpurple,
	linecolor=darkpurple,
	rightline=false,
	topline=false,
	bottomline=false,
	linewidth=2pt,
	innertopmargin=8pt,
	innerbottommargin=8pt,
	innerleftmargin=8pt,
	leftmargin=-2pt,
	skipbelow=2pt,
	nobreak
},
headfont=\normalfont\bfseries\color{darkpurple}
]{purplebox}

% GREEN BOX %
\declaretheoremstyle[
mdframed={
	backgroundcolor=lightgreen,
	linecolor=darkgreen,
	rightline=false,
	topline=false,
	bottomline=false,
	linewidth=2pt,
	innertopmargin=8pt,
	innerbottommargin=8pt,
	innerleftmargin=8pt,
	leftmargin=-2pt,
	skipbelow=2pt,
	nobreak
},
headfont=\normalfont\bfseries\color{darkgreen}
]{greenbox}

% YELLOW BOX %
\declaretheoremstyle[
mdframed={
	backgroundcolor=lightyellow,
	linecolor=darkyellow,
	rightline=false,
	topline=false,
	bottomline=false,
	linewidth=2pt,
	innertopmargin=8pt,
	innerbottommargin=8pt,
	innerleftmargin=8pt,
	leftmargin=-2pt,
	skipbelow=2pt,
	nobreak
},
headfont=\normalfont\bfseries\color{darkyellow}
]{yellowbox}

% BLUE BOX %
\declaretheoremstyle[
mdframed={
	backgroundcolor=lightblue,
	linecolor=darkblue,
	rightline=false,
	topline=false,
	bottomline=false,
	linewidth=2pt,
	innertopmargin=8pt,
	innerbottommargin=8pt,
	innerleftmargin=8pt,
	leftmargin=-2pt,
	skipbelow=2pt,
	nobreak
},
headfont=\normalfont\bfseries\color{darkblue}
]{bluebox}

% RED BOX %
\declaretheoremstyle[
mdframed={
	backgroundcolor=lightred,
	linecolor=darkred,
	rightline=false,
	topline=false,
	bottomline=false,
	linewidth=2pt,
	innertopmargin=8pt,
	innerbottommargin=8pt,
	innerleftmargin=8pt,
	leftmargin=-2pt,
	skipbelow=2pt,
	nobreak
},
headfont=\normalfont\bfseries\color{darkred}
]{redbox}
%%

%% ENVIRONMENTS
% PURPLE BOXES (theorems, propositions, lemmas, and corollaries) %
\declaretheorem[style=purplebox,name=Theorem,within=section]{thm}
\declaretheorem[style=purplebox,name=Theorem,sibling=thm]{theorem}
\declaretheorem[style=purplebox,name=Theorem,numbered=no]{thm*, theorem*}
\declaretheorem[style=purplebox,name=Proposition,sibling=thm]{prop, proposition}
\declaretheorem[style=purplebox,name=Proposition,numbered=no]{prop*, proposition*}
\declaretheorem[style=purplebox,name=Lemma,sibling=thm]{lem, lemma}
\declaretheorem[style=purplebox,name=Lemma,numbered=no]{lem*, lemma*}
\declaretheorem[style=purplebox,name=Corollary,sibling=thm]{cor, corollary}
\declaretheorem[style=purplebox,name=Corollary,numbered=no]{cor*, corollary*}

% GREEN BOXES (definitions) %
\declaretheorem[style=greenbox,name=Definition,sibling=thm]{definition, defn}
\declaretheorem[style=greenbox,name=Definition,numbered=no]{definition*, defn*}

% BLUE BOXES (problems) %
\declaretheorem[style=bluebox,name=Problem,numberwithin=section]{homework}
\declaretheorem[style=bluebox,name=Problem,numbered=no]{homework*, hw*}

% RED BOXES %
\declaretheorem[style=redbox,name=Remark,sibling=thm]{remark, rmk}
\declaretheorem[style=redbox,name=Remark, numbered=no]{remark*, rmk*}
\declaretheorem[style=yellowbox,name=Warning,sibling=thm]{warn}
\declaretheorem[style=yellowbox,name=Warning,numbered=no]{warn*}
%%

%% PROOF FORMATTING
\renewcommand\qedsymbol{$\blacksquare$}
\newenvironment{innerproof}{\renewcommand{\qedsymbol}{$\square$}\proof}{\endproof}
%%

% CUSTOM COMMANDS
% basic %
\newcommand{\Mod}[1]{\ (\mathrm{mod}\ #1)}
\newcommand{\floor}[1]{\left\lfloor{#1}\right\rfloor}
\newcommand{\ceil}[1]{\left\lceil{#1}\right\rceil}
\newcommand{\norm}[1]{\left\lVert{#1}\right\rVert}

% logic %
\newcommand*\xor{\oplus}
\newcommand{\all}{\forall}
\newcommand{\bland}{\bigwedge}
\newcommand{\blor}{\bigvee}

% matrices %
\newcommand\aug{\fboxsep=- \fboxrule\!\!\!\fbox{\strut}\!\!\!}\makeatletter 

% sets %
\newcommand{\CC}{\mathbb{C}}
\newcommand{\NN}{\mathbb{N}}
\newcommand{\QQ}{\mathbb{Q}}
\newcommand{\RR}{\mathbb{R}}
\newcommand{\ZZ}{\mathbb{Z}}

% probability stuff %
\newcommand{\E}{\mathbb{E}}
\newcommand{\Var}{\mathrm{Var}}
\newcommand{\var}{\mathrm{Var}}
\newcommand{\cov}{\mathrm{cov}}
\newcommand{\corr}{\mathrm{Corr}}

% linalg stuff %
\DeclareMathOperator*{\Span}{\mathrm{Span}}
\DeclareMathOperator*{\Null}{\mathrm{Null}}
\DeclareMathOperator*{\Range}{\mathrm{Range}}
\DeclareMathOperator*{\vspan}{\mathrm{span}}
\DeclareMathOperator*{\vnull}{\mathrm{null}}
\DeclareMathOperator*{\vrange}{\mathrm{range}}
\newcommand{\innerproduct}[2]{\left\langle{#1}, {#2}\right\rangle}
\DeclareMathOperator*{\proj}{\mathrm{proj}}

% set theory stuff %
\DeclareMathOperator*{\codom}{\mathrm{codom}}
\DeclareMathOperator*{\dom}{\mathrm{dom}}
\DeclareMathOperator*{\ran}{\mathrm{ran}}
\DeclareMathOperator*{\fld}{\mathrm{fld}}
\newcommand{\image}[1]{\left\llbracket {#1} \right\rrbracket}

% bracketing %
\newcommand{\pr}[1]{\left( {#1} \right)}
\newcommand{\br}[1]{\left[ {#1} \right]}
\newcommand{\brc}[1]{ \left\{  {#1} \right\}}
\newcommand{\ang}[1]{\langle {#1} \rangle}
%%

%% LIST OF PROBLEM SETUP
%solution by mrclrchtr: https://tex.stackexchange.com/questions/475799/how-can-i-change-the-format-of-an-entry-in-listoftheorems-of-thmtools

% reformats (Name) to ''Name''
\renewcommand\thmtformatoptarg[1]{ ''#1''}

% swaps number with theorem name
% swaps hw
\makeatletter
\def\ll@hw{%
	\thmt@thmname~
	\protect\numberline{\csname the\thmt@envname\endcsname}%
	\ifx\@empty
	\thmt@shortoptarg
	\else
	\protect\thmtformatoptarg{\thmt@shortoptarg}
	\fi
}
\makeatother

% changes spacing
\makeatletter
\renewcommand*{\numberline}[1]{\hb@xt@3em{#1}}
\makeatother

%%

%% CUSTOM HOMEWORK
\NewDocumentEnvironment{hw}{m o}{\IfNoValueTF{#2}{
		% if hw:
		\renewcommand*{\thehomework}{\thesubsection}\setcounter{subsection}{#1}\begin{homework}}
		{% else, ahw:
			\renewcommand*{\thehomework}{\thesubsection\alph{homework}}\setcounter{subsection}{#1}\setcounter{homework}{\inteval{#2-1}}\begin{homework}}}
		{\end{homework}}
%%

%% MISC. RENAMING
\renewcommand*{\thehomework}{\thesubsection}
\renewcommand{\listtheoremname}{Problems}
%%

%% TITLE PAGE
\newcommand{\mytitle}[2]{%
	\title{#1}
	\author{Michael Pham}
	\date{#2}
	\maketitle
	\newpage
	\listoftheorems
	\newpage
}
%%

%%
%END_FOLD
%%

\begin{document}
	\mytitle{Math 135: Homework 3}{Spring 2024}
	
	\setcounter{section}{2}
	\section{Relations and Functions}
	\begin{hw}{12}
		Assume that $f$ and $g$ are functions and show that
		\begin{equation*}
			f \subseteq g \iff \pr{\dom f \subseteq \dom g} \land \pr{\forall x \in \dom f}f(x) = g(x).
		\end{equation*}
	\end{hw}
	\begin{solution}
		Let us suppose that $f$ and $g$ are functions
		
		First, we will prove the forward direction.
		\begin{innerproof}
			Suppose that $f \subseteq g$.
			
			From here, let us take some $x \in \dom f$. Then, by definition, there exists some $f(x)$ such that $\ang{x, f(x)} \in f$. Then, since $f \subseteq g$, it follows that $\ang{x, f(x)} \in g$ as well. Thus, we see that $x \in \dom g$ by definition. Therefore, we have that $\dom f \subseteq \dom g$.
			
			Next, we want to show that $\forall x \in \dom f$, we have $f(x) = g(x)$.
			
			To do this, we observe from the last step that by definition of $x \in \dom g$, there exists some $g(x)$ such that $\ang{x, g(x)} \in g$. However, we note that $\ang{x, f(x)} \in g$ as well. Then, as $g$ is a function and we have that $\ang{x, f(x)}$ and $\ang{x, g(x)}$ are both in $g$, it must be that $f(x) = g(x)$ for all $x \in \dom f$.
		\end{innerproof}
	
		Now, we will prove the backwards direction.
		\begin{innerproof}
			Let us suppose that $\dom f \subseteq \dom g$, and that $(\forall x \in \dom f) f(x) = g(x)$. We want to show that $f \subseteq g$.
			
			To do this, let us take some $\ang{x, f(x)} \in f$. Then, $x \in \dom f$, and by our assumption, $x \in \dom g$ as well. Then, this means that there exists some $g(x)$ such that $\ang{x, g(x)} \in g$.
			
			However, we note that by our assumption, $f(x) = g(x)$ for all $x \in \dom f$, so in fact we have $\ang{x, g(x)} = \ang{x, f(x)} \in g$.
			
			Thus, $f \subseteq g$ as desired.
		\end{innerproof}
	\end{solution}

	\begin{hw}{14}[1]
		Assume that $f$ and $g$ are functions. Show that $f \cap g$ is a function.
	\end{hw}
	\begin{solution}
		Let us suppose that $f$ and $g$ are functions.
		
		Now, by definition, we have that $f \cap g$ consists of all ordered pairs in both $f$ and $g$.
		
		So, let us take some $\ang{a,b} \in f \cap g$ and $\ang{a,c} \in f \cap g$. Note then that $\ang{a,b}$ and $\ang{a,c}$ must both be in $f$. And since $f$ is a function, by definition, it must be that $b = c$, and thus $f \cap g$ is a function as well.
	\end{solution}
	
	\begin{hw}{14}[2]
		Assume that $f$ and $g$ are functions. Show that $f \cup g$ is a function if and only if $f(x) = g(x)$ for every $x$ in $(\dom f) \cap (\dom g)$.
	\end{hw}
	\begin{solution}
		Suppose that $f$ and $g$ are functions.
		
		Now, we will prove the forward direction.
		\begin{innerproof}
			Suppose that $f \cup g$ is a function. Then, let us take some $x \in (\dom f) \cap (\dom g)$. Then, this means that there exists some $\ang{x,f(x)} \in f$ and $\ang{x,g(x)} \in g$.
			
			Now, since $f \cup g$ consists of all ordered pairs in $f$ or $g$, we note then that $\ang{x,f(x)}$ and $\ang{x,g(x)}$ are both in $f \cup g$. However. since $f \cup g$ is a function, then by definition, we have that $f(x) = g(x)$.
		\end{innerproof}
	
		Next, we'll show the backwards direction.
		\begin{innerproof}
			Suppose that $f(x) = g(x)$ for every $x \in (\dom f) \cap (\dom g)$.
			
			Then, let us take some $\ang{x,y} \in f \cup g$ and $\ang{x,z} \in f \cup g$. We have four cases to consider:
			\begin{itemize}
				\item First, we note that if both $\ang{x,y}$ and $\ang{x,z}$ are in $f$, then, since $f$ is a function, it follows that $y = z$.
				\item Similarly, if $\ang{x,y}$ and $\ang{x,z}$ are both in $g$, it follows that because $g$ is a function, $y = z$.
				\item Next, if $\ang{x,y} \in f$ and $\ang{x,z} \in g$, we first note then that $x \in (\dom f) \cap (\dom g)$. From here, by our assumption, we have that $f(x) = g(x)$ for all $x \in (\dom f) \cap (\dom g)$. Thus, we have that $\ang{x,y} = \ang{x, f(x)} = \ang{x,g(x)} = \brc{x,z}$. Thus, we have that $y = z$ as desired.
				\item Finally, if $\ang{x,y} \in g$ and $\ang{x,z} \in f$, we can follow a similar argument as above to conclude that $y = z$ as well.
			\end{itemize}
		
			Therefore, we conclude that $\forall x \forall y \forall z$, we have that if $\ang{x,y} \in f \cup g$ and $\ang{x,z} \in f \cup g$, then $y = z$, which is precisely what it means for $f \cup g$ to be a function.
		\end{innerproof}
	\end{solution}
	
	\begin{hw}{18}
		Let $R$ be the set
		\begin{equation*}
			\brc{\ang{0,1}, \ang{0,2}, \ang{0,3}, \ang{1,2}, \ang{1,3}, \ang{2,3}}.
		\end{equation*}
	
		Evaluate the following: $R \circ R$, $R \restriction \brc{1}$, $R^{-1} \restriction \brc{1}$, $R\image{\brc{1}}$, and $R^{-1}\image{ \brc{1}}$.
	\end{hw}
	\begin{solution}
		First, we note that we have
		\begin{equation*}
			R^{-1} = \brc{\ang{1,0}, \ang{2,0}, \ang{3,0}, \ang{2,1}, \ang{3,1}, \ang{3,2}}.
		\end{equation*}
		
		We have the following:
		\begin{itemize}
			\item $R \circ R = \brc{\ang{0,2}, \ang{0,3}, \ang{1,3}}$.
			\item $R \restriction \brc{1} = \brc{\ang{1,2}, \ang{1,3}}$.
			\item $R^{-1} \restriction \brc{1} = \brc{\ang{1,0}}$.
			\item $R\image{ \brc{1}} = \brc{2,3}$.
			\item $R^{-1}\image{ \brc{1}} = \brc{0}$.
		\end{itemize}
	\end{solution}

	\begin{hw}{21}
		Show that $(R \circ S) \circ T = R \circ (S \circ T)$ for any set $R, S$, and $T$.
	\end{hw}
	\begin{solution}
		We shall proceed as follows:
		\begin{align*}
			\ang{w,z} \in (R \circ S) \circ T &\iff \exists x(\ang{w,x} \in T \land \ang{x,z} \in R \circ S) \\
			&\iff \exists x( \ang{w,x} \in T \land \exists y (\ang{x,y} \in S \land \ang{y,z} \in R)) \\
			&\iff \exists x\exists y( \ang{w,x} \in T \land (\ang{x,y} \in S \land \ang{y,z} \in R)) \\
			&\iff \exists y\exists x( (\ang{w,x} \in T \land \ang{x,y} \in S) \land \ang{y,z} \in R) \\
			&\iff \exists y (\exists x(\ang{w,x} \in T \land \ang{x,y} \in S) \land \ang{y,z} \in R) \\
			&\iff \exists y (\ang{w,y} \in (S \circ T) \land \ang{y,z} \in R) \\
			&\iff \ang{w,z} \in R \circ (S \circ T) 
		\end{align*}
	
		Thus, we see that, indeed, $(R \circ S) \circ T = R \circ (S \circ T)$ as desired.
	\end{solution}
	
	\begin{hw}{22}[1]
		For any set, show that $A \subseteq B \implies F \image{A} \subseteq F \image B$.
	\end{hw}
	\begin{solution}
		Suppose that we have some $y \in F\image{A}$. Then, by definition, this means that there exists some $x \in A$ such that $\ang{x,y} \in F$. 
	
		However, we note that since $A \subseteq B$, $x \in B$ as well. Thus, we see that $y \in F\image{B}$ by definition, and thus we have that $F\image{A} \subseteq F\image{B}$ as desired.
	\end{solution}
	
	\begin{hw}{22}[2]
		For any set, show that $(F \circ G)\image{A} = F\image{G\image{A}}$.
	\end{hw}
	\begin{solution}
		We will first show that $\pr{F \circ G}\image A \subseteq F\image{G\image{A}}$.
	
		Let us take some $z \in (F \circ G)\image A$. Then, by definition, we observe that there exists some $x \in A$ such that $\ang{x,z} \in F \circ G$.
		
		Then, by definition of composition, there exists some $y$ such that $\ang{x,y} \in G$ and $\ang{y,z} \in F$. So, we observe that $y \in G\image{A}$, and thus $z \in F \image{G\image{A}}$.
		
		Now, we will show the other inclusion.
		
		Suppose that $z \in F\image{G\image{A}}$. Then, by definition, we see that there exists some $y \in G\image{A}$ such that $\ang{y,z} \in F$. But then, this means that there exists some $x \in A$ such that $\ang{x,y} \in G$.
			
		Putting this all together, we have then that there exists some $x \in A$ and there exists some $y$ such that $\ang{x,y} \in G$ and $\ang{y,z} \in F$. In other words, we have $z \in (F \circ G)\image{A}$ as desired.
	\end{solution}

	\begin{hw}{22}[3]
		For any set, show that $Q \restriction(A \cup B) = (Q \restriction A) \cup (Q \restriction B)$.
	\end{hw}
	\begin{solution}
		We proceed as follows:
		\begin{align*}
			\ang{x,y} \in Q \restriction (A \cup B) &\iff \ang{x,y} \in \brc{\ang{u,v} : (\ang{u,v} \in Q) \land (u \in A \cup B)} \\
			&\iff \ang{x,y} \in \brc{\ang{u,v} : (\ang{u,v} \in Q) \land (u \in A \lor u \in B)} \\
			&\iff \ang{x,y} \in \brc{\ang{u,v} : (\ang{u,v} \in Q \land u \in A) \lor (\ang{u,v} \in Q \land u \in B)} \\
			&\iff \ang{x,y} \in \brc{\ang{u,v} : \ang{u,v} \in Q \land u \in A} \lor \ang{x,y} \in \brc{\ang{u,v} : \ang{u,v} \in Q \land u \in B} \\
			&\iff \ang{x,y} \in (Q \restriction A) \cup (Q \restriction B)
		\end{align*}
	
		And thus, we see that $Q \restriction (A \cup B) = (Q \restriction A) \cup (Q \restriction B)$ as desired.
	\end{solution}

	\begin{hw}{29}
		Assume that $f : A \rightarrow B$ and define a function $G : B \rightarrow \mathscr{P}A$ by 
		\begin{equation*}
			G(b) \coloneq \brc{x \in A : f(x) = b}.
		\end{equation*} 
	
		Show that if $f$ maps $A$ onto $B$, then $G$ is one-to-one. Does the converse hold?
	\end{hw}
	\begin{solution}
		First, we will show that if $f$ maps $A$ onto $B$, then $G$ is one-to-one.
		\begin{innerproof}
			Suppose that $f$ maps $A$ onto $B$. Then, by definition, for all $b \in B$, there exists some $a \in A$ such that $f(a) = b$.
			
			Then, we want to show that $G$ is one-to-one. Suppose that we had $G(b) = G(b')$ for $b, b' \in B$. Because $f$ is surjective, we note then that $G(b)$ and $G(b')$ are both non-empty. Now, we have the following:
			\begin{equation*}
				G(b) = \brc{x \in A : f(x) = b} = \brc{x \in A : f(x) = b'} = G(b')
			\end{equation*}
			
			However, we note here that for these two sets to be equal, their elements must be the same. In other words, $\forall x(x \in G(b) \iff x \in G(b'))$.
			
			Then, let us take $x \in G(b)$ such that $f(x) = b$. Since $G(b) = G(b')$, $x \in G(b')$ as well, meaning that $f(x) = b'$. Since $f$ is a function, it follows then that since we have $f(x) = b$ and $f(x) = b'$, it must be that $b = b'$. Thus, we see that if $G(b) = G(b')$, then $b = b'$. Thus, $G$ is one-to-one as desired.
		\end{innerproof}
	
		Now, we will examine whether the converse holds or not.
		\begin{innerproof}
			We claim that it doesn't.
			
			Let us consider the set $A = \brc{x}$, and $B = \brc{x,y}$. Then, we define $G(x) = \brc{x}$ and $G(y) = \emptyset$. By inspection, we see that $G$ is indeed injective.
			
			However, note here that while $f(x) = x$, there exists no $a \in A$ such that $f(a) = 2$; $f$ does not map $A$ onto $B$.
		\end{innerproof}
	\end{solution}

	\begin{hw}{30}[1]
		Assume that $F : \mathscr{P}A \rightarrow \mathscr{P}A$, and that $F$ has the monotonicity property:
		\begin{equation*}
			X \subseteq Y \subseteq A \implies F(X) \subseteq F(Y).
		\end{equation*}
		
		Define
		\begin{equation*}
			B = \bigcap \brc{X \subseteq A : F(X) \subseteq X} \qquad \text{and} \qquad C = \bigcup \brc{X \subseteq A : X \subseteq F(X)}.
		\end{equation*}
		Show that $F(B) = B$ and $F(C) = C$.
	\end{hw}
	\begin{solution}
		We will first show that $F(B) = B$.
		\begin{innerproof}
			First, we will show that $F(B) \subseteq B$. To do this, let us define $\mathcal B$ to be the following:
			\begin{equation*}
				\mathcal B = \brc{X \subseteq A : F(X) \subseteq X}.
			\end{equation*}
		
			Then, we see that $B = \bigcap \mathcal B$.
			
			Now, for all $X \in \mathcal B$, we observe that $B \subseteq X$ by definition of the intersection. Then, it follows that $F(B) \subseteq F(X)$ by monotonicity of $F$.
			
			From here, we note that for every $X \in \mathcal B$, we have that $F(B) \subseteq F(X) \subseteq X$. Then, we have that for all $X \in \mathcal B$, $F(B) \subseteq \bigcap \brc{X : X \in \mathcal B} = B$. Therefore, $F(B) \subseteq B$ as desired.
			
			For this other inclusion, observe that since $F(B) \subseteq B \subseteq X \subseteq A$, then $F(F(B)) \subseteq F(B)$ by monotonicity. However, this means then that $F(B) \in \mathcal B$ by definition. Then, by definition of the intersection, any $x \in \bigcap \mathcal B$ is also in $F(B)$. Thus, we have that $\bigcap \mathcal B = B \subseteq F(B)$ as desired.
			
			Therefore, we see that equality holds.
		\end{innerproof}
	
		Now we will prove that $F(C) = C$. Note that the proof is similar to the previous claim.
		\begin{innerproof}
			First, we will show $C \subseteq F(C)$. Let us define $\mathcal C$ to be
			\begin{equation*}
				\mathcal C = \brc{X \subseteq A : X \subseteq F(X)}.
			\end{equation*}
			
			Then, we have that $C = \bigcup \mathcal C$.
			
			Now, for all $X \in \mathcal C$, we observe that $X \subseteq C$, and by monotonicity of $F$, $F(X) \subseteq F(C)$. And since $\mathcal C$ contains all $X$ such that $X \subseteq F(X)$, we have that $X \subseteq F(X) \subseteq F(C)$.
			
			So, for any $X \in \mathcal C$, we have that for any $x \in X$, $x \in F(C)$ as well. From here, by definition of $\bigcup \mathcal C$, this means that any $x \in \bigcup\mathcal C$ is in some $X \in \mathcal C$, and thus in $F(C)$ as desired. So, we have that $\bigcup \mathcal C = C \subseteq F(C)$.
			
			For the other inclusion, we observe that since $C \subseteq F(C)$, then we have that $F(C) \subseteq F(F(C))$ by monotonicity of $F$. However, this means then that $F(C) \in \mathcal C$ by definition of $\mathcal C$.
			
			From here, by definition of $\bigcup \mathcal C$, we see then that $F(C) \subseteq \bigcup \mathcal C = C$.
			
			Therefore, we can conclude that equality holds.
		\end{innerproof}
	\end{solution}
	
	\begin{hw}{30}[2]
		Show that if $F(X) = X$, then $B \subseteq X \subseteq C$.
	\end{hw}
	\begin{solution}
		We observe that if $F(X) = X$, then it follows that $F(X) \subseteq X$ and $X \subseteq F(X)$.
		
		With this in mind, we observe that as $F(X) \subseteq X$, by definition of $B$, we have that $B \subseteq X$.
		
		Similarly, since $X \subseteq F(X)$, then by definition of $C$, we have that $X \subseteq C$.
		
		Putting it all together, we have that $B \subseteq X \subseteq C$ as desired.
	\end{solution}

	% AOC 1: For any relation $R$, there exists a function $H \subseteq R$ with $\dom H = \dom R$.
	% AOC 2: For any set I, and any function H with domain I, if H(i) isn't empty for all i in I, then X_{i in I} H(i) isn't empty.
	\begin{hw}{31}
		Show that the first form of the Axiom of Choice is equivalent to the second form.
	\end{hw}
	\begin{solution}
		Before we begin, we will state the two forms of the Axiom of Choice:
		\begin{enumerate}
			\item For any relation $R$, there exists a function $H \subseteq R$ with $\dom H = \dom R$.
			\item For any set $I$, and any function $H$ with domain $I$, if $H(i) \neq \emptyset$ for all $i \in I$, then $\bigtimes_{i \in I} H(i) \neq \emptyset$.
		\end{enumerate}
	
		Furthermore, we note that the definition of $\bigtimes_{i \in I} H(i)$ is:
		\begin{equation*}
			\bigtimes_{i \in I} H(i) = \brc{f : f\text{ is a function with domain $I$ and } (\forall i \in I)f(i) \in H(i)}.
		\end{equation*}
		
		First, we will prove the forward direction.
		\begin{innerproof}
			Let us assume the first form of the Axiom of Choice.
			
			Now, to begin with, for any set $I$ and any function $H$ whose domain is $I$ and $H(i) \neq \emptyset$ for all $i \in I$, suppose we had some relation $R \subseteq I \times \bigcup_{i \in I} H(i)$ with the following property:
			\begin{equation*}
				\ang{i, r} \in R \iff r \in H(i).
			\end{equation*}
			
			Then from here, we see that by the first form of the Axiom of Choice, there exists a function $G \subseteq R$ with $\dom G = \dom R = I$. Then, we observe that for all $i \in I$, we have that $\ang{i, G(i)} \in G$, and thus $\ang{i, G(i)} \in R$. 
			
			By definition of $R$ then, for all $i \in I$ we have $G(i) \in H(i)$, meaning that $G \in \bigtimes_{i \in I} H(i)$ by definition of $\bigtimes_{i \in I} H(i)$.
			
			And since this is the case, we see then that $\bigtimes_{i \in I} H(i)$ can't be empty. Thus, we have proven that the first version of the Axiom of Choice implies the second.
		\end{innerproof}
	
		Now, we will show the other direction.
		\begin{innerproof}
			Assume the second form of the Axiom of Choice. Then, let us define a relation $R$ such that its domain is $I$. Next, we define a function $H$ to be the following:
			\begin{equation*}
				H: I \rightarrow \mathscr{P}\pr{\ran R} \text{ where }i \mapsto H(i),
			\end{equation*}
			where $H(i)$ is defined to be the following:
			\begin{equation*}
				H(i) = \brc{r \in \ran{R} : \ang{i, r} \in R}.
			\end{equation*}
		
			From here, we see that the domain of $H$ is $I$. Furthermore, $H(i) \neq \emptyset$ for all $i \in I$. Then, by the second form of the Axiom of Choice, we see that $\bigtimes_{i \in I} H(i) \neq \emptyset$.
			
			With this in mind, we can choose some function $G \in \bigtimes_{i \in I} H(i)$. Note then that $\dom G = I = \dom R$.
			
			Then, we want to show that $G \subseteq R$. To do this, let us take some $\ang{i, G(i)} \in G$. By definition, we note that $G(i) \in H(i)$ for all $i \in I$, meaning that $G(i) \in \ran R$. And since we have that $i \in \dom R$, we see that $\ang{i, G(i)} \in R$ for all $i \in I$. Thus, we can conclude then that $G \subseteq R$ with $\dom G = \dom R$ as desired.
		\end{innerproof}
	
		Therefore, we can conclude that these two forms of the Axiom of Choice are indeed equivalent.
	\end{solution}
\end{document}