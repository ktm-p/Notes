\documentclass{article}
%%%% PREAMBLE %%%%
%BEGIN_FOLD
%%% PACKAGES
\usepackage{amsmath}
\usepackage{amssymb}
\usepackage{amsthm}
\usepackage{cabin} % section title font
\usepackage[default]{cantarell} % default font
\usepackage[shortlabels]{enumitem}
\usepackage{fancyhdr}
\usepackage{graphicx}
\usepackage{hyperref}
\usepackage{mathtools}
\usepackage[framemethod=TikZ]{mdframed}
\usepackage[scr]{rsfso} % power set symbol
\usepackage{tasks} % vaguely remember this being important for something...?
\usepackage{tikz} % diagrams
\usepackage{titlesec}
\usepackage{thmtools}
\usepackage{varwidth}
\usepackage{verbatim} % longer comments
\usepackage{xcolor}
%%%

%%% COLOURS
\definecolor{darkgreen}{HTML}{19A514}
\definecolor{lightgreen}{HTML}{9DFF9A}
\definecolor{darkblue}{HTML}{3E5FE4}
\definecolor{lightblue}{HTML}{BCDEFF}
\definecolor{darkred}{HTML}{CC3333}
\definecolor{lightred}{HTML}{FFA9A9}
\definecolor{darkpurple}{HTML}{A933CD}
\definecolor{lightpurple}{HTML}{F0BAFF}
\definecolor{darkyellow}{HTML}{D2D22A}
\definecolor{lightyellow}{HTML}{FFFFAE}
\definecolor{hyperlinkblue}{HTML}{3366CC}
%%%

%%% PAGE SETUP
% BASIC %
\setlength\parindent{0pt} % paragraph indentation
\setlength{\parskip}{5pt} % spacing between paragraphs
\usepackage[margin=1in]{geometry} % margin size

% HEADER/FOOTER %
\pagestyle{fancy}
\fancyhf{}
\fancyfoot[R]{\thepage} % page number on bottom right
\fancyhead[R]{\textit{\leftmark}} % section title
\renewcommand{\headrulewidth}{0pt} % removing horizontal line at the top

% HYPERLINK FORMATTING %
\hypersetup{
	colorlinks,    
	linkcolor=hyperlinkblue,
	urlcolor=hyperlinkblue,
	pdftitle={...},
	pdfauthor={Michael Pham},
}
%%%

%%% ENVIRONMENTS STYLES
% SOLUTION ENVIRONMENT %
\newenvironment{solution}{\begin{proof}[Solution]}{\end{proof}}

% PURPLE BOX %
\declaretheoremstyle[
mdframed={
	backgroundcolor=lightpurple,
	linecolor=darkpurple,
	rightline=false,
	topline=false,
	bottomline=false,
	linewidth=2pt,
	innertopmargin=8pt,
	innerbottommargin=8pt,
	innerleftmargin=8pt,
	leftmargin=-2pt,
	skipbelow=2pt,
	nobreak
},
headfont=\normalfont\bfseries\color{darkpurple}
]{purplebox}

% GREEN BOX %
\declaretheoremstyle[
mdframed={
	backgroundcolor=lightgreen,
	linecolor=darkgreen,
	rightline=false,
	topline=false,
	bottomline=false,
	linewidth=2pt,
	innertopmargin=8pt,
	innerbottommargin=8pt,
	innerleftmargin=8pt,
	leftmargin=-2pt,
	skipbelow=2pt,
	nobreak
},
headfont=\normalfont\bfseries\color{darkgreen}
]{greenbox}

% YELLOW BOX %
\declaretheoremstyle[
mdframed={
	backgroundcolor=lightyellow,
	linecolor=darkyellow,
	rightline=false,
	topline=false,
	bottomline=false,
	linewidth=2pt,
	innertopmargin=8pt,
	innerbottommargin=8pt,
	innerleftmargin=8pt,
	leftmargin=-2pt,
	skipbelow=2pt,
	nobreak
},
headfont=\normalfont\bfseries\color{darkyellow}
]{yellowbox}

% BLUE BOX %
\declaretheoremstyle[
mdframed={
	backgroundcolor=lightblue,
	linecolor=darkblue,
	rightline=false,
	topline=false,
	bottomline=false,
	linewidth=2pt,
	innertopmargin=8pt,
	innerbottommargin=8pt,
	innerleftmargin=8pt,
	leftmargin=-2pt,
	skipbelow=2pt,
	nobreak
},
headfont=\normalfont\bfseries\color{darkblue}
]{bluebox}

% RED BOX %
\declaretheoremstyle[
mdframed={
	backgroundcolor=lightred,
	linecolor=darkred,
	rightline=false,
	topline=false,
	bottomline=false,
	linewidth=2pt,
	innertopmargin=8pt,
	innerbottommargin=8pt,
	innerleftmargin=8pt,
	leftmargin=-2pt,
	skipbelow=2pt,
	nobreak
},
headfont=\normalfont\bfseries\color{darkred}
]{redbox}
%%%

%%% ENVIRONMENTS
% PURPLE BOXES (theorems, propositions, lemmas, and corollaries) %
\declaretheorem[style=purplebox,name=Theorem,within=section]{thm}
\declaretheorem[style=purplebox,name=Theorem,sibling=thm]{theorem}
\declaretheorem[style=purplebox,name=Theorem,numbered=no]{thm*, theorem*}
\declaretheorem[style=purplebox,name=Proposition,sibling=thm]{prop, proposition}
\declaretheorem[style=purplebox,name=Proposition,numbered=no]{prop*, proposition*}
\declaretheorem[style=purplebox,name=Lemma,sibling=thm]{lem, lemma}
\declaretheorem[style=purplebox,name=Lemma,numbered=no]{lem*, lemma*}
\declaretheorem[style=purplebox,name=Corollary,sibling=thm]{cor, corollary}
\declaretheorem[style=purplebox,name=Corollary,numbered=no]{cor*, corollary*}

% GREEN BOXES (definitions) %
\declaretheorem[style=greenbox,name=Definition,sibling=thm]{definition, defn}
\declaretheorem[style=greenbox,name=Definition,numbered=no]{definition*, defn*}

% BLUE BOXES (problems) %
\declaretheorem[style=bluebox,name=Problem,numberwithin=section]{homework, hw}
\declaretheorem[style=bluebox,name=Problem,numberwithin=section]{alphahomework, ahw}
\declaretheorem[style=bluebox,name=Problem,numbered=no]{homework*, hw*}

% RED BOXES %
\declaretheorem[style=redbox,name=Remark,sibling=thm]{remark, rmk}
\declaretheorem[style=redbox,name=Remark, numbered=no]{remark*, rmk*}
\declaretheorem[style=yellowbox,name=Warning,sibling=thm]{warn}
\declaretheorem[style=yellowbox,name=Warning,numbered=no]{warn*}
%%%

%%% PROOF FORMATTING
\renewcommand\qedsymbol{$\blacksquare$}
\newenvironment{innerproof}{\renewcommand{\qedsymbol}{$\square$}\proof}{\endproof}
%%%

%% CUSTOM COMMANDS
% basic %
\newcommand{\Mod}[1]{\ (\mathrm{mod}\ #1)}
\newcommand{\floor}[1]{\left\lfloor{#1}\right\rfloor}
\newcommand{\ceil}[1]{\left\lceil{#1}\right\rceil}
\newcommand{\norm}[1]{\left\lVert{#1}\right\rVert}

% logic %
\newcommand*\xor{\oplus}
\newcommand{\all}{\forall}
\newcommand{\bland}{\bigwedge}
\newcommand{\blor}{\bigvee}

% matrices %
\newcommand\aug{\fboxsep=- \fboxrule\!\!\!\fbox{\strut}\!\!\!}\makeatletter 

% sets %
\newcommand{\CC}{\mathbb{C}}
\newcommand{\NN}{\mathbb{N}}
\newcommand{\QQ}{\mathbb{Q}}
\newcommand{\RR}{\mathbb{R}}
\newcommand{\ZZ}{\mathbb{Z}}

% probability stuff %
\newcommand{\E}{\mathbb{E}}
\newcommand{\Var}{\mathrm{Var}}
\newcommand{\var}{\mathrm{Var}}
\newcommand{\cov}{\mathrm{cov}}
\newcommand{\corr}{\mathrm{Corr}}

% linalg stuff %
\DeclareMathOperator*{\Span}{\mathrm{Span}}
\DeclareMathOperator*{\Null}{\mathrm{Null}}
\DeclareMathOperator*{\Range}{\mathrm{Range}}
\DeclareMathOperator*{\vspan}{\mathrm{span}}
\DeclareMathOperator*{\vnull}{\mathrm{null}}
\DeclareMathOperator*{\vrange}{\mathrm{range}}
\newcommand{\innerproduct}[2]{\left\langle{#1}, {#2}\right\rangle}
\DeclareMathOperator*{\proj}{\mathrm{proj}}

% set theory stuff %
\DeclareMathOperator*{\dom}{\mathrm{dom}}
\DeclareMathOperator*{\ran}{\mathrm{ran}}
\DeclareMathOperator*{\fld}{\mathrm{fld}}

% bracketing %
\newcommand{\pr}[1]{\left( {#1} \right)}
\newcommand{\br}[1]{\left[ {#1} \right]}
\newcommand{\brc}[1]{ \left\{  {#1} \right\}}
\newcommand{\ang}[1]{\langle {#1} \rangle}
%%%

%%% LIST OF PROBLEM SETUP
%solution by mrclrchtr: https://tex.stackexchange.com/questions/475799/how-can-i-change-the-format-of-an-entry-in-listoftheorems-of-thmtools

% reformats (Name) to ''Name''
\renewcommand\thmtformatoptarg[1]{ ''#1''}

% swaps number with theorem name
% swaps hw
\makeatletter
\def\ll@hw{%
	\thmt@thmname~
	\protect\numberline{\csname the\thmt@envname\endcsname}%
	\ifx\@empty
	\thmt@shortoptarg
	\else
	\protect\thmtformatoptarg{\thmt@shortoptarg}
	\fi
}
\makeatother

% swaps ahw
\makeatletter
\def\ll@ahw{%
	\thmt@thmname~
	\protect\numberline{\csname the\thmt@envname\endcsname}%
	\ifx\@empty
	\thmt@shortoptarg
	\else
	\protect\thmtformatoptarg{\thmt@shortoptarg}
	\fi
}
\makeatother
%%%

%%% MISC. RENAMING
\renewcommand*{\thehw}{\thesubsection}
\renewcommand*{\theahw}{\thesubsection\alph{ahw}}
\renewcommand{\listtheoremname}{Problems}
%\newcommand{prb}[3]{def}
%%%

%%% TITLE PAGE
\newcommand{\mytitle}[2]{%
	\title{#1}
	\author{Michael Pham}
	\date{#2}
	\maketitle
	\newpage
	\listoftheorems
	\newpage
}
%%%

%%%
%END_FOLD
%%%

\begin{document}
	\mytitle{Math 135: Homework 2}{Spring 2024}
	
	\setcounter{section}{2}
	
	\section{Relations and Functions}
	\setcounter{subsection}{1}
	\begin{hw}
		Suppose that we attempted to generalize the Kuratowski definitions of ordered pairs to ordered triples with the following definition:
		\begin{equation*}
			\ang{x,y,z}^{*} = \brc{ \brc{x}, \brc{x,y}, \brc{x,y,z}}.
		\end{equation*}
		
		Show that the definition is unsuccessful by giving examples of objects $u,v,w,x,y,z$ with $\ang{x,y,z}^{*} = \ang{u,v,w}^{*}$ but with either $y \neq v$ or $z \neq w$ (or both)
	\end{hw}
	\begin{solution}
		We observe that if we let $x = x, y = y, z = x$ and $u = x, v = y, w = y$, we have the following:
		\begin{align*}
			\ang{x,y,z}^{*} &= \brc{ \brc{x}, \brc{x,y}, \brc{x,y,x}} \\
			&= \brc{ \brc{x}, \brc{x,y}, \brc{x,y}} \\
			&= \brc{ \brc{x}, \brc{x,y}} \\
			\ang{u,v,w}^{*} &= \brc{ \brc{x}, \brc{x,y}, \brc{x,y,y}} \\
			&= \brc{ \brc{x}, \brc{x,y}, \brc{x,y}} \\
			&= \brc{ \brc{x}, \brc{x,y}} \\
		\end{align*}
		
		So, we see that $\ang{x,y,z}^{*} = \ang{u,v,w}^{*}$, but that $z \neq w$. Thus, the definition is unsuccessful.
	\end{solution}
	
	\setcounter{subsection}{2}
	\begin{ahw}
		Show that $A \times \pr{B \cup C} = \pr{A \times B} \cup \pr{A \times C}$.
	\end{ahw}
	\begin{solution}
		\begin{comment}
			Let us denote $X = A$ and $Y = B \cup C$ so that it's clearer.
			
			Then, we observe the following:
			\begin{align*}
				x \in A \times \pr{B \cup C} &\iff x \in \brc{t : \exists A \exists \pr{B \cup C}\pr{a \in A \land b \in \pr{B \cup C} \land t \in \ang{a,b}}} \\
				&\iff x \in \brc{t : \exists A \exists B \exists C \pr{a \in A \land (b \in B \lor c \in C) \land t \in \ang{a,b}}} \\
				&\iff x \in \brc{t : \exists A \exists B \exists C \pr{ \pr{a \in A \land b \in B \land t \in \ang{a,b}} \lor \pr{a \in A \land c \in C \land t \in \ang{a,c}}}} \\
				&\iff x \in \brc{t : \exists A \exists B (a \in A \land b \in B \land t \in \ang{a,b})} \lor x \in \brc{t : \exists A \exists C (a \in A \land c \in C \land t \in \ang{a,c})}
				&\iff x \in \pr{A \times B} \cup \pr{A \times C}
			\end{align*}
		\end{comment}
		We shall proceed as follows:
		\begin{align*}
			\ang{x,y} \in A \times (B \cup C) &\iff (x \in A) \land (y \in \pr{B \cup C}) \\
			&\iff (x \in A) \land \pr{(y \in B) \lor (y \in C)} \\
			&\iff \pr{(x \in A) \land (y \in B)} \lor \pr{(x \in A) \land (y \in C)} \\
			&\iff (\ang{x,y} \in A \times B) \lor \pr{\ang{x,y} \in A \times C} \\
			&\iff \ang{x,y} \in \pr{A \times B} \cup \pr{A \times C}
		\end{align*}
	\end{solution}
	
	\begin{ahw}
		Show that if $A \times B = A \times C$, and $A \neq \emptyset$, then $B = C$.
	\end{ahw}
	\begin{solution}
		We observe that $A \times B = A \times C$ means that we have the following:
		\begin{equation*}
			\ang{x,y} \in A \times B \iff \ang{x,y} \in A \times C
		\end{equation*}
		
		Then, since $A \neq \emptyset$, we observe that for all $y \in B$, there exists some $x \in A$ such that:
		\begin{align*}
			x \in A \land y \in B \iff \ang{x,y} \in A \times B \iff \ang{x,y} \in A \times C \iff x \in A \land y \in C
		\end{align*}
		
		Then, we see that $y \in C$ as well by going forward in the implications above. Similarly, if $y \in C$, then we can go backwards and thus get that $y \in B$.
		
		Since this holds for all $y$, we see that, indeed, $B = C$ as desired.
	\end{solution}
	
	\setcounter{subsection}{4}
	\begin{hw}
		Show that there exists no set to which every ordered pair belongs.
	\end{hw}
	\begin{solution}
		We first recall the definition of an ordered pair:
		\begin{equation*}
			\ang{x,y} \coloneq \brc{ \brc{x}, \brc{x,y}}.
		\end{equation*}
		
		Now, let us suppose for the sake of contradiction that the set to which every ordered pair belongs does exist. We denote this set by $S$.
		
		Then, with this in mind, we can have the following ordered pair:
		\begin{equation*}
			\ang{x,x} = \brc{ \brc{x}, \brc{x,x}} = \brc{ \brc{x}, \brc{x}} = \brc{ \brc{x}}.
		\end{equation*} 
		
		We note that this is a subset of $S$. From here, we note then that this means then that the set of all singletons $S'$ is also be a subset of $S$.
		
		However, recall from a previous homework problem that $S'$ cannot exist.
		
		Thus, we have a contradiction and conclude that such a set $S$ cannot exist.
	\end{solution}
	
	\setcounter{subsection}{5}
	\setcounter{ahw}{0}
	\begin{ahw}
		Assume that $A$ and $B$ are given sets, and show that there exists a set $C$ such that for any $y$,
		\begin{equation*}
			y \in C \iff y = \brc{x} \times B \text{ for some $x$ in $A$}.
		\end{equation*}
		
		In other words, show that $\brc{ \brc{x} \times B : x \in A}$ is a set.
	\end{ahw}
	\begin{solution}
		To begin with, we observe that $\brc{x} \subseteq A \implies \brc{x} \times B \subseteq A \times B$, for some $x \in A$.
		
		Then, we see that $\brc{x} \times A \in \mathscr{P}(A \times B)$. Now, we know that $A \times B$ is a set, and by the Power Set Axiom, so is $\mathscr{P}(A \times B)$. Furthermore, by definition of $A \times B$, we note then that any $t \in \mathscr{P}(A \times B)$ is in $\mathscr{P}\mathscr{P}\mathscr{P}(A \cup B)$.
		
		Then, that means that by the Subset Axiom, we can construct the following set:
		\begin{equation*}
			C \coloneq \brc{t \in \mathscr{P}\mathscr{P}\mathscr{P}(A \cup B) : \exists x(t = \brc{x} \times B) \land (x \in A)}
		\end{equation*}
	\end{solution}
	
	\begin{ahw}
		With $A, B, C$ as above, show that $A \times B = \bigcup C$.
	\end{ahw}
	\begin{solution}
		We will show that $A \times B = \bigcup C$ by first showing that $A \times B \subseteq \bigcup C$, then we will show that $\bigcup C \subseteq A \times B$.
		
		To begin with, we will show that $A \times B \subseteq \bigcup C$. Let us denote $z = \ang{x,y}$. Then, we see that if $z = \ang{x,y} \in A \times B$, then by definition we have that $x \in A$ and $y \in B$.
		
		More specifically, we note that $\ang{x,y} \in \brc{x} \times B$. Then, by definition of the union, this means that $\ang{x,y} = z \in \bigcup \brc{ \brc{x} \times B : x \in A} = \bigcup C$ as desired. Thus, we have shown that $A \times B \subseteq \bigcup C$.
		
		On the other hand, let $z \in \bigcup C = \bigcup \brc{ \brc{x} \times B : x \in A}$. Then, we note that, by definition of union, we have that $z \in \brc{x} \times B$. Then, note that $\brc{x} \times B \subseteq A \times B$, meaning that $z \in A \times B$. Thus, $\bigcup C \subseteq A \times B$ as desired.
		
		Therefore, we conclude that $A \times B = \bigcup C$. 
	\end{solution}
	
	\setcounter{subsection}{6}
	\begin{hw}
		Show that a set $A$ is a relation iff $A \subseteq \dom A \times \ran A$.
	\end{hw}
	\begin{solution}
		We will first show the forward direction.
		\begin{innerproof}($\implies$)
			Let us suppose that $A$ is a relation. Then, by definition of a relation, any $a \in A$ is an ordered pair.
			
			Then, let us take $a = \ang{x,y} \in A$. We note then that since $\ang{x,y} \in A$, then we know that:
			\begin{enumerate}
				\item There exists some $y$ such that $\ang{x,y} \in A$. Thus, we see that $x \in \dom A$.
				\item There exists some $x$ such that $\ang{x,y} \in A$. Thus, we see that $y \in \ran A$. 
			\end{enumerate}
			
			Therefore, $a = \ang{x,y} \in \dom A \times \ran A$, and thus we see that $A \subseteq \dom A \times \ran A$.
		\end{innerproof}
		
		For the backwards direction, we proceed as follows:
		\begin{innerproof}($\impliedby$)
			Let us suppose that $A \subseteq \dom A \times \ran A$. Then, we note that every element in $\dom A \times \ran A$ is an ordered pair by definition, and thus, since $A \subseteq \dom A \times \ran A$, every element in $A$ must also be an ordered pair.
			
			Thus, we see that $A$ is a relation by definition.
		\end{innerproof}
		
		Thus, indeed, a set $A$ is a relation iff $A \subseteq \dom A \times \ran A$.
	\end{solution}
	
	\setcounter{subsection}{7}
	\begin{hw}
		Show that if $R$ is a relation, then $\fld R = \bigcup \bigcup R$.
	\end{hw}
	\begin{solution}
		Suppose that $R$ is a relation.
		
		%We will first show that $\fld R \subseteq \bigcup\bigcup R$.
		Note that, by definition, $\fld R = \dom R \cup \ran R$.
		
		Furthermore, we have:
		\begin{align*}
			\dom R &\coloneq \brc{t \in \bigcup\bigcup R : \exists b \pr{\ang{t,b} \in R}} \\
			\ran R &\coloneq \brc{t \in \bigcup\bigcup R : \exists a \pr{\ang{a,t} \in R}}
		\end{align*}
		
		Then, we observe the following:
		\begin{align*}
			x \in \fld R &\iff x \in \dom R \cup \ran R \\
			&\iff \pr{x \in \dom R} \lor \pr{x \in \ran R} \\
			&\iff \pr{x \in  \brc{t \in \bigcup\bigcup R : \exists b \pr{\ang{t,b} \in R}}} \lor \pr{x \in  \brc{t \in \bigcup\bigcup R : \exists a \pr{\ang{a,t} \in R}}} \\
			&\iff x \in \bigcup\bigcup R
		\end{align*}
		
		Thus, we see that $\fld R = \bigcup\bigcup R$.
		
		\begin{comment}
			Next, we will show that $\bigcup\bigcup R \subseteq \fld R$. Let us take some $x \in \bigcup \bigcup R$. Then, there exists a $y \in \bigcup R$ such that $x \in y \in \bigcup R$. Furthermore, there exists a $z \in R$ such that $x \in y \in z \in R$. Note then that since $R$ is a relation, all of its elements are an ordered pair, and thus $z$ is an ordered pair.
			
			Then, let us define $z = \ang{a,b} = \brc{ \brc{a}, \brc{a,b}}$. Note that, by definition, we have that $a \in \dom R$ and $b \in \ran R$.
			
			Now, we note that $y \in z$, so we have two cases:
			\begin{enumerate}
				\item $y = \brc{a}$, or
				\item $y = \brc{a,b}$.
			\end{enumerate}
			
			In the first case, we have that $y = \brc{a}$, and since $x \in y$, we see that $x = a \in \dom R$. Thus, $x \in \dom R \implies x \in \dom R \cup \ran R$ as desired.
			
			In the second case, if $y = \brc{a,b}$, and $x \in y$, then that means that we have two more cases:
			\begin{enumerate}
				\item $x = a$, or
				\item $x = b$.
			\end{enumerate}
			
			In the first case, $x = a \in \dom R$. In the second case, $x = b \in \ran R$. Thus, we have $x \in \dom R$ or $x \in \ran R$. In other words, $x \in \dom R \cup \ran R$.
			
			Therefore, in all cases, we see that $x \in \dom R \cup \ran R = \fld R$.
			
			Thus, we have shown that $\fld R = \bigcup\bigcup R$ as desired.
		\end{comment}
	\end{solution}
	
	\setcounter{subsection}{8}
	\begin{hw}
		Show that for any set $A$, we have:
		\begin{align*}
			\dom \bigcup A &= \bigcup \brc{\dom R : R \in A} \\
			\ran \bigcup A &= \bigcup \brc{\ran R : R \in A}.
		\end{align*}
	\end{hw}
	\begin{solution}
		To begin with, we will prove the first statement.
		\begin{innerproof}
			Suppose that $x \in \dom \bigcup A$. Then, by definition of $\dom \bigcup A$, there exists some $b$ such that $\ang{x,b} \in \bigcup A$.
			
			Then, this means that there exists some $R \in A$ such that $\ang{x,b} \in R$. By definition then, we see that $x \in \dom R$, and thus $x \in \bigcup \brc{\dom R : R \in A}$. Therefore, $\dom \bigcup A \subseteq \bigcup \brc{\dom R : R \in A}$.
			
			On the other hand, suppose that $x \in \bigcup \brc{\dom R : R \in A}$. Then, this means that for some $R \in A$, we have that $x \in \dom R$. And by definition, this means that there exists some $b$ such that $\ang{x,b} \in R$.
			
			From here, we note then that $R \subseteq \bigcup A$, so we have that $\ang{x,b} \in \bigcup A$. This means then that there exists a $b$ such that $\ang{x,b} \in \bigcup A$. However, this is precisely the definition of $x \in \dom \bigcup A$ as desired. Thus, we see that $\bigcup \brc{\dom R : R \in A} \subseteq \dom \bigcup A$.
			
			Thus, we can conclude that $\dom \bigcup A = \bigcup \brc{\dom R : R \in A}$.
			
			\begin{comment}
				We will proceed as follows:
				\begin{align*}
					x \in \dom \bigcup A &\iff x \in \brc{t : \exists b\pr{\ang{t,b} \in \bigcup A}} \\
					&\iff x \in \brc{t : \exists b\exists R\pr{\pr{\ang{t,b} \in R} \land \pr{R \in A}}} \\
					&\iff \exists R \pr{ \pr{x \in \dom R} \land \pr{R \in A}} \\
					&\iff x \in \bigcup \brc{\dom R : R \in A}.
				\end{align*}
			\end{comment}
		\end{innerproof}
		
		Next, we will prove the second statement.
		\begin{innerproof}
			\begin{comment}
				We note that this proof is similar to the previous one. Suppose that $y \in \ran \bigcup A = \bigcup \brc{\ran R : R \in A}$. Then, there exists some $R \in \bigcup A$ such that $\ang{a,y} \in R$. Then, by definition, we see that $y \in \ran R$, and thus $y \in \bigcup \brc{\ran R: R \in A}$. Thus, $\ran \bigcup A \subseteq \bigcup \brc{\ran R : R \in A}$.
				
				On the other hand, 
				
				Then, 
			\end{comment}
			Suppose that $y \in \ran \bigcup A$. Then, by definition of $\ran \bigcup A$, there exists some $a$ such that $\ang{a,y} \in \bigcup A$.
			
			Then, this means that there exists some $R \in A$ such that $\ang{a,y} \in R$. By definition then, we see that $y \in \ran R$, and thus $y \in \bigcup \brc{\ran R : R \in A}$. Therefore, $\ran \bigcup A \subseteq \bigcup \brc{\ran R : R \in A}$.
			
			On the other hand, suppose that $y \in \bigcup \brc{\ran R : R \in A}$. Then, this means that for some $R \in A$, we have that $y \in \ran R$. And by definition, this means that there exists some $a$ such that $\ang{a,y} \in R$.
			
			From here, we note then that $R \subseteq \bigcup A$, so we have that $\ang{a,y} \in \bigcup A$. This means then that there exists a $a$ such that $\ang{a,y} \in \bigcup A$. However, this is precisely the definition of $y \in \ran \bigcup A$ as desired. Thus, we see that $\bigcup \brc{\ran R : R \in A} \subseteq \ran \bigcup A$.
			\begin{comment}
				\begin{align*}
					y \in \ran \bigcup A &\iff y \in \brc{t : \exists a\pr{\ang{a,t} \in \bigcup A}} \\
					&\iff y \in \brc{t : \exists a\exists R\pr{\pr{\ang{a,t} \in R} \land \pr{R \in A}}} \\
					&\iff \exists R\pr{\pr{y \in \ran R} \land \pr{R \in A}} \\
					&\iff y \in \bigcup \brc{\ran R : R \in A}.
				\end{align*}
			\end{comment}
			
			And thus, we see that $\ran \bigcup A = \bigcup \brc{\ran R : R \in A}$ as desired.
		\end{innerproof}
	\end{solution}
\end{document}
