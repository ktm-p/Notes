\documentclass{article}
\usepackage{homework}
\usepackage{macros}

%% LIST OF PROBLEMS SETUP %%
\renewcommand\thmtformatoptarg[1]{:\enspace#1}
\makeatletter
\def\ll@homework{
	\thmt@thmname~
	\protect\numberline{\csname the\thmt@envname\endcsname}%
	\ifx\@empty
	\thmt@shortoptarg
	\else
	\protect\thmtformatoptarg{\thmt@shortoptarg}
	\fi
}
\makeatother

\makeatletter
\renewcommand*{\numberline}[1]{\hb@xt@3em{#1}}
\makeatother	

%% RENEW TITLE PAGE %%
\renewcommand{\mytitle}[2]{%
	\title{#1}
	\author{Michael Pham}
	\date{#2}
	\maketitle
	\newpage
	\listoftheorems
	\newpage
}

\begin{document}
\mytitle{Math 135: Homework 8}{Spring 2024}

\setcounter{section}{5}
\section{Cardinal Numbers and the Axiom of Choice}
\begin{hw}{13}[0][0]
	Show that a finite union of finite sets is finite.
\end{hw}
\begin{solution}
	We can proceed by induction.
	
	First, we observe that for a set $A$ with cardinality $0$, we have then that $A = \emptyset$. Then, $\bigcup A = \emptyset$, so, indeed, we have that $\bigcup A$ is finite as well.
	
	Next, suppose that our claim holds for set $A$ with cardinality $n$.
	
	Now, we look at $A$ whose cardinality is $n^{+}$. Observe then that because $A$ is finite, it follows that there exists a bijection between $A$ and $n^{+}$, and thus some bijective function $f : n^{+} \rightarrow A$.
	
	Then, we have:
	\begin{align*}
		\bigcup A &= \bigcup_{k \in n^{+}} f(k) \\
		&= \bigcup_{k \in n} f(k) \cup f(n)
	\end{align*}
	
	By our induction hypothesis, we have that $\bigcup_{k \in n} f(k)$ is finite. And since $f(n)$ is also finite, we thus have that $\bigcup_{k \in n} f(k) \cup f(n)$ is finite too.
	
	Therefore, by induction, we conclude that the claim holds as desired.
\end{solution}

\begin{hw}{14}[0][0]
	Define a permutation of $K$ to be any one-to-one function from $K$
	onto $K$. We can then define the factorial operation on cardinal numbers by the equation
	\begin{equation*}
		\kappa ! = \mathrm{card} \brc{f : f \text{ is a permutation of $K$}}
	\end{equation*}
	where $K$ is any set of cardinality $\kappa$. Show that $\kappa !$ is well defined.
\end{hw}
\begin{solution}
	Suppose we have sets $K_0$ and $K_1$. Let $\card K_0 = \kappa = \card K_1$. 
	
	Then, because the cardinalities of $K_0$ and $K_1$ are the same, we can thus construct a bijection between them.
	
	To show that $\kappa!$ is well-defined, we will have to show then that there exists a bijection between the set of permutations of $K_0$ (which we will denote at $K_0'$) and permutations of $K_1$ (which we denote as $K_1$').
	
	Then to do this, we recall that there exists a bijection $g$ between $K_0$ and $K_1$. So, for each permutation $f$ of $K_1$, we can first send this permutation to $K_0$ with our bijection $g$, which we then permute. After, we can send the permutation of $K_0$ back to $K_1$ using $g^{-1}$.
	
	Thus, we observe then that there exists a bijection between the set of permutations $f$ of $K_0$ and of $K_1$; i.e., we have shown that $\kappa!$ is well-defined.
\end{solution}

\begin{hw}{15}[0][0]
	Show that there is no set $A$ with the property that for every set there is some member of $A$ that dominates it.
\end{hw}
\begin{solution}
	Suppose for the sake of contradiction that such a set $A$ did exist.
	
	Then, let us consider the power set of the union of $A$. We denote this by $S$.
	
	Then, we observe that $A \subset \mathscr{P}\bigcup A = S$. However, this means then that $A \prec S$ which is a contradiction, as then no element of $A$ dominates $S$.
\end{solution}

\begin{hw}{16}[0][0]
	Show that for any set $S$ we have $S \preccurlyeq \prescript{S}{}{2}$, but $S \not\approx \prescript{S}{}{2}$.
\end{hw}
\begin{solution}
	To begin with, denote $\card S = \lambda$. We note that $\card \prescript{S}{}{2} = 2^{\lambda}$. Meanwhile, $\card S = \lambda$. So, we have that $\card S \leq \card \prescript{S}{}{2}$; in other words, $S \preccurlyeq\prescript{S}{}{2}$.
	
	Now, we will show that $S \not\approx \prescript{S}{}{2}$. To do this, let us consider the following functions $F: S \rightarrow \prescript{S}{}{2}$ and $g(x) = 1 - F(x)(x)$. We note here that $g : S \rightarrow 2$.
	
	Now, we observe that for some function $F(x)$ such that $F(x) = g$ for all $x \in S$, then we have the following:
	\begin{align*}
		F(x) = g &\implies F(x)(x) = 1 - F(x)(x) \\
		&\implies 2F(x)(x) = 1 \\
		&\implies F(x)(x) = \dfrac{1}{2}
	\end{align*}

	However, we note that since $F(x)$ should be a function from $S$ to $2$, it must be then that $F(x)$ has output of either $0$ and $1$; therefore, because $g$ isn't in the image of $F$, $S \not\approx\prescript{S}{}{2}$.
\end{solution}

\end{document}