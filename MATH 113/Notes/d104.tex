\documentclass[openany]{book}
% !TeX TXS-program:compile = txs:///pdflatex/[--shell-escape]
\usepackage{macros}
\usepackage{notes}

%% PICTURES DIRECTORY %%
\graphicspath{{C:/Users/Michael/Pictures/}}

%% RENEW TITLE PAGE %%
\renewcommand{\mytitle}[2]{%
	\title{#1}
	\author{Michael Pham}
	\date{#2}
	\maketitle
	\newpage
	\mytoc
	\newpage
}

\begin{document}
\mytitle{Notes for Data 104}{Idk}

\chapter{The Last Week}
\section{Discussion -- 12/05/2024}
\subsection{Moral Frameworks and MIT Moral Machine}
First, we remind ourselves of the traditional western moral philosophical frameworks:
\begin{itemize}
	\item Virtue Ethics: This focuses more on the person themselves, looking at the character habits/qualities.
	\begin{itemize}
		\item It emphasizes a way of ``being" rather than ``doing."
	\end{itemize}
	\item Deontology: 
	\item Consequentialism:
\end{itemize}

\subsection{Value Alignment}
The reason we are discussing value alignment is because it's one of the main examples of how concepts from this course concretely manifests in a workplace setting.

It's also a powerful way of thinking, aligning machines to human ways of thinking.

A similarity between Effective Alturism and Value Alignment can be seen in how ...

\begin{fancyquotes}
	The moral good in any democracy consist of providing people with what they want, and the challenge facing AI engineers consists of developing AI that understands what people want.
	
	\begin{flushright}
		\emph{citation}
	\end{flushright}
\end{fancyquotes}
\end{document}