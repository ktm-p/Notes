\documentclass[openany]{book}
% !TeX TXS-program:compile = txs:///pdflatex/[--shell-escape]
\usepackage{macros}
\usepackage{notes}

%% PICTURES DIRECTORY %%
\graphicspath{{C:/Users/Michael/Pictures/}}

%% RENEW TITLE PAGE %%
\renewcommand{\mytitle}[2]{%
	\title{#1}
	\author{Michael Pham}
	\date{#2}
	\maketitle
	\newpage
	\mytoc
	\newpage
}

\begin{document}
\mytitle{Temporary}{12/6/2024}

\chapter{The Final Week}
\section{Lecture -- 12/06/2024}
\subsection{Warm-Up}
\begin{hw}
	Why is $x^{3} - 2$ an irreducible polynomial over $\QQ$?
\end{hw}
\begin{solution}
		Recall in Homework 11, we were asked to prove that $\QQ[\sqrt[3] 2, \sqrt[3] 4]$ is a field. The biggest problem was to show invertibility. That is, why $a + b\sqrt[3] 2 + c\sqrt[3] 4$ has an inverse for $a,b,c \in \QQ$.
	
	Now, looking at $\sqrt[3] 2$ and $\sqrt[3] 4$, we note that the latter is just the square root of the former. So, let us denote $x = \sqrt[3] 2$ and $\sqrt[3] 4 = x^{2}$.
	
	Now, right now, we know that $x^{3} = 2$. This is the same as saying that $x^{3} - 2 = 0$.
	
	Now, we note that $\QQ[x, x^{2}] = \QQ[\sqrt[3] 2, \sqrt[3] 4]$ is a field. Then, using the following proposition, we see then that $x^{3} - 2$ is in fact irreducible over $\QQ$:
	\begin{prop}
		We have the following isomorphism (as a ring):
		\begin{equation*}
			\QQ[\sqrt[3] 2, \sqrt[3] 4] \cong \QQ[x] / (x^{3} - 2)
		\end{equation*}
		
		Where we note that $\QQ[x]$ is a polynomial ring in $x$ with $\QQ$-coefficients, and $x^{3} - 2$ is an ideal in $\QQ[x]$ generated by $x^{3} - 2$.
		
		Note that we sometimes denote $(x^{3} - 2)$ as $\ang{x^{3} - 2} = \brc{q(x) \cdot (x^{3} - 2) : q(x) \in \QQ[x]}$.
	\end{prop}
\end{solution}

\subsection{Why is it a field?}
For this lecture, our primary motivation is to show that $\QQ[\sqrt[3] 2, \sqrt[3] 4]$ is a field by going into the land of rings.

That is, we want to show that $\QQ[x] / (x^{3} - 2)$ is a field. To answer this question, we note that this is only true if and only if $(x^{3} - 2)$ is maximal in $\QQ[x]$.

From here then, we see that we can ignore $a + b\sqrt[3] 2 + c\sqrt[3] 4$, and instead look at this in a purely algebraic way.

\begin{rmk}
	Recall previously, we showed that $\ZZ/(3)$ was a field iff $(3) \leq \ZZ$ is maximal. 
	
	To do this, supposed that $(3)$ wasn't maximal, meaning that there exists some ideal $J$ such that $(3) \lneq J \lneq \ZZ$. Then, since they're not equal to each other, we note that $j \not\in (3)$. Then, note that $\gcd(3, j) = 1$; this means then that $3a + jb = 1$. But this means then that ...
\end{rmk}

Then, to show maximality of the ideal, we see that:
\begin{proof}
	Suppose for contradiction that $(x^{3} - 2) \lneq J \lneq \QQ[x]$. Then, we know that $j(x) \in J$ isn't in $(x^{3} - 2)$.
	
	Now, we note that since $x^{3} - 2$ is irreducible in $\QQ[x]$, it follows then that it can't be factored down into any other polynomials in $\QQ[x]$. Next, we note that because $j(x) \not\in (x^{3} - 2)$, we know then that it can't be a multiple of $x^{3} - 2$, meaning that it can't have $x^{3} - 2$ as a factor.
	
	Thus, we note that $\gcd(j(x), x^{3} - 2) = 1$.
	
	Then, with that in mind, we see then that by Euclidean Algorithm for $\QQ[x]$, we can show that there exists polynomials $a(x), b(x)$ $(x^{3} - 2)a(x) + j(x)b(x) = 1$.
	
	Now, we note that because $1$ can be written as a linear combination $x^{3} - 2$ and $j(x)$, both of which sits inside of $J$. Recall from the definition of ideal, we know that $a(x)(x^{3} - 2)$ and $b(x)j(x)$ are both part of $J$. Then, since ideals are additive subgroup, it follows then that $1 \in J$.
	
	And since $1 \in J$, we know then that $J = \QQ[x]$. But this contradicts with $J$ not being the whole ring.
	
	Therefore, we conclude that $(x^{3} - 2)$ must in fact be maximal.
\end{proof}

And since $(x^{3} - 2)$ is a maximal ideal, we see then that $\QQ[x] / (x^{3} - 2)$ is thus a field. And therefore, we know then that $\QQ[\sqrt[3] 2, \sqrt[3] 4]$ is a field.

When solving a problem, proper mathematicians often try to generalize the problem as much as possible. This leads us to the final theorem in this course:
\begin{thm}
	Let $F$ be a field, and $f(x) \in F[x]$ is irreducible. Then, $F[x] / (f(x))$ is also a field.
\end{thm}
\begin{proof}
	To prove this, we simply want to prove that the ideal $(f(x))$ is maximal. The prove will follow as previously with more concrete examples.
\end{proof}

\begin{rmk}
	As a final remark, we note that from this, we can instead start with any finite field $F$, and any irreducible polynomial, we can create new finite fields of greater sizes.
\end{rmk}
\end{document}