\documentclass{article}
% !TeX TXS-program:compile = txs:///pdflatex/[--shell-escape]
\usepackage{homework}
\usepackage{macros}

%% PICTURES DIRECTORY %%
\graphicspath{{C:/Users/Michael/Pictures/}}

%% LIST OF PROBLEMS SETUP %%
\renewcommand\thmtformatoptarg[1]{:\enspace#1}
\makeatletter
\def\ll@homework{
	\thmt@thmname~
	\protect\numberline{\csname the\thmt@envname\endcsname}%
	\ifx\@empty
	\thmt@shortoptarg
	\else
	\protect\thmtformatoptarg{\thmt@shortoptarg}
	\fi
}
\makeatother

\makeatletter
\renewcommand*{\numberline}[1]{\hb@xt@3em{#1}}
\makeatother	

%% RENEW TITLE PAGE %%
\renewcommand{\mytitle}[2]{%
	\title{#1}
	\author{Michael Pham}
	\date{#2}
	\maketitle
	\newpage
	\listoftheorems
	\newpage
}

\begin{document}
\mytitle{Math 128A: Programming Assignment 2}{Summer 2024}

\setcounter{section}{1}

\begin{hw}{1}[0][0]
	Approximate the curve by fitting natural cubic splines to the given data, independently for $x(t)$ and $y(t)$. Plot the curve in MATLAB by
	\texttt{plot(xx,yy, x,y,'o'), axis equal, grid on},
	where \texttt{x,y} contain the given values and \texttt{xx,yy} contain the spline data evaluated for a large number of parameter values between $0$ and $5$.
\end{hw}
\begin{solution}
Below, we include the commands used to construct the approximation of the curve using natural cubic splines:
\begin{code}{MATLAB}{MATLAB Commands}
>>> ax = [1, 1.5, 2, 2, 2.5, 2.5];
ay = [1, 0.5, 1, 1.5, 1.5, 1];
t = [0, 1, 2, 3, 4, 5];
[bx, cx, dx] = ncspline(t, ax);
[by, cy, dy] = ncspline(t, ay);
s = 0:0.01:5;
xx = splineeval(t, ax, bx, cx, dx, s);
yy = splineeval(t, ay, by, cy, dy, s);
plot(xx, yy, ax, ay, 'o'), axis equal, grid on
\end{code}

The following is the image of the plot created:
\begin{center}
\includegraphics{ncs_parametric_curve}
\end{center}

We also provide the values coefficients for both $x(t)$ and $y(t)$ in the provided order:
\begin{code}{MATLAB}{Values for $x(t)$}
>> ax

ax =

Columns 1 through 5

1.000000000000000   1.500000000000000   2.000000000000000   2.000000000000000   2.500000000000000

Column 6

2.500000000000000

>> bx

bx =

0.452153110047847   0.595693779904306   0.165071770334928   0.244019138755981   0.358851674641148

>> cx

cx =

0   0.143540669856459  -0.574162679425837   0.653110047846890  -0.538277511961722

>> dx

dx =

0.047846889952153  -0.239234449760766   0.409090909090909  -0.397129186602871   0.179425837320574
\end{code}

\begin{code}{MATLAB}{Values for $y(t)$}
>> ay

ay =

Columns 1 through 5

1.000000000000000   0.500000000000000   1.000000000000000   1.500000000000000   1.500000000000000

Column 6

1.000000000000000

>> by

by =

-0.760765550239234   0.021531100478469   0.674641148325359   0.279904306220096  -0.294258373205742

>> cy

cy =

0   0.782296650717703  -0.129186602870813  -0.265550239234450  -0.308612440191388

>> dy

dy =

0.260765550239234  -0.303827751196172  -0.045454545454545  -0.014354066985646   0.102870813397129
\end{code}
\end{solution}

\begin{hw}{2}[0][0]
	Use Newton's method to find the parameter values $t_1$ and $t_2$ where the curve intersects the line
	$y = 1.2$. Use the provided MATLAB functions \texttt{splineeval.m} and \texttt{diffsplineeval.m}. Give a
	short justification for your choice of initial guess for each of the two intersections.
\end{hw}
\begin{solution}
	First, we provide the MATLAB commands below:
\begin{code}{MATLAB}{MATLAB Commands}
>> f = @(p0) splineeval(t, ay, by, cy, dy, p0) - 1.2;
df = @(p0) diffsplineeval(t, ay, by, cy, dy, p0);
t1 = newton(f, df, 2.5, 0.000000001);
t2 = newton(f, df, 4.5, 0.000000001);

>> fprintf("%.8f", t1)
2.31798217

>> fprintf("%.8f", t2)
4.66164416
\end{code}

We chose the initial points $p_0 = 2.5$ and $q_0 = 4.5$. Looking at the data, we see that $y(t)$ goes from $1.0$ to $1.5$ between $t = 2$ and $t = 3$. So, we used the average of these endpoints.

Similarly, we used $q_0 = 4.5$ since at $t = 4$, $y(t) = 1.5$, and at $t = 5$, we have $y(t) = 1.0$. So, since the curve is continuous, we know that at some point between those two times, $y(t) = 1.2$ by the Mean Value Theorem.    
\end{solution}

\begin{hw}{3}[0][0]
Compute the length of the segment of the curve above $y = 1.2$, by numerically evaluating the
integral
\begin{equation*}
	L = \int_{t_1}^{t_2} \sqrt[]{x'(t)^{2} + y'(t)^{2}} \mathrm{dt}
\end{equation*}
using the composite trapezoidal rule. Compute a series of approximations $L_{16}$, $L_{32}$, $L_{64}$, and $L_{128}$ using $n = 16, 32, 64, 128$, respectively. Also compute a highly accurate value $L$ using $n = 10, 000$.

Plot the errors $\abs{L_n - L}$ versus $h = (t_2 - t_1)/n$ in a log-log plot and estimate its slope.
\end{hw}

\begin{solution}
To begin with, we provide the code for the function \texttt{comp\_trap} below:
\begin{code}{MATLAB}{Code for \texttt{comp\_trap} function}
function val = comp_trap(a, b, n, f)

h = (b - a)/n;
i1 = a + h;
in = a + (n - 1)*h;
sum = 0;
for i = i1:h:in
sum = sum + f(i);
end
sum = 2*sum;
sum = sum + f(a) + f(b);
val = (h/2)*sum;
end
\end{code}

Next, we provide the results:
\begin{code}{MATLAB}{Results for $L_n$}
>> l16 = comp_trap(t1, t2, 16, g)
l32 = comp_trap(t1, t2, 32, g)
l64 = comp_trap(t1, t2, 64, g)
l128 = comp_trap(t1, t2, 128, g)

l16 =

1.162654862462451


l32 =

1.161785809250850


l64 =

1.161604753231803


l128 =

1.161556258136053

>> l10000 = comp_trap(t1, t2, 10000, g)

l10000 =

1.161540504195514
\end{code}

Finally, we include the log-log plot below, generated with the following code:
\begin{code}{MATLAB}{Plotting errors}
>> yl16 = log(abs(l16 - l10000));
yl32 = log(abs(l32 - l10000));
yl64 = log(abs(l64 - l10000));
yl128 = log(abs(l128 - l10000));
plot([log((t2-t1)/16), log((t2-t1)/32), log((t2-t1)/64), log((t2-t1)/128)], [yl16, yl32, yl64, yl128], 'o')
\end{code}

And the following graph is generated, with the $h$ values in the x-axis and the error values in the y-axis:
\begin{center}
	\includegraphics{loglog_err}
\end{center}

Using the middle two points, we then calculate the following slope:
\begin{equation*}
	\frac{-8.31301 - (-9.65274)}{-2.61402 - (-3.30717)} \approx 1.933
\end{equation*}

So, we see that the slope is around $2$.
\end{solution}
\end{document}