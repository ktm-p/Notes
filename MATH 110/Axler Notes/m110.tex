\documentclass[openany]{book}
%%%%% PREAMBLE
%BEGIN_FOLD
%%%%% PACKAGES
\usepackage{amsmath}
\usepackage{amssymb}
\usepackage{amsthm}
\usepackage{cabin} % section title font
\usepackage[default]{cantarell} % default font
\usepackage[shortlabels]{enumitem}
\usepackage{fancyhdr}
\usepackage{graphicx}
\usepackage{hyperref}
\usepackage[utf8]{inputenc}
\usepackage{mathtools}
\usepackage[framemethod=TikZ]{mdframed}
\usepackage[scr]{rsfso} % power set symbol
\usepackage{tasks} % vaguely remember this being important for something...?
\usepackage{tikz} % diagrams
\usepackage{titlesec}
\usepackage{thmtools}
\usepackage{varwidth}
\usepackage{verbatim} % longer comments
\usepackage{xcolor}
%%%%%

%%%%% COLOURS
\definecolor{darkgreen}{HTML}{19A514}
\definecolor{lightgreen}{HTML}{9DFF9A}
\definecolor{darkblue}{HTML}{3E5FE4}
\definecolor{lightblue}{HTML}{BCDEFF}
\definecolor{darkred}{HTML}{CC3333}
\definecolor{lightred}{HTML}{FFA9A9}
\definecolor{darkpurple}{HTML}{A933CD}
\definecolor{lightpurple}{HTML}{F0BAFF}
\definecolor{darkyellow}{HTML}{D2D22A}
\definecolor{lightyellow}{HTML}{FFFFAE}
\definecolor{hyperlinkblue}{HTML}{3366CC}
%%%%%

%%%%% PAGE SETUP
% basic %
\setlength\parindent{0pt} % paragraph indentation
\setlength{\parskip}{5pt} % spacing between paragraphs
\usepackage[margin=1in]{geometry} % margin size

% header/footer %
\pagestyle{fancy}
\fancyhf{}
\renewcommand{\headrulewidth}{0pt} % removing horizontal line at the top
\setlength{\headheight}{15pt}
\renewcommand{\chaptermark}[1]{\markright{#1}{}}
\renewcommand{\sectionmark}[1]{}
\fancypagestyle{contentpage}{%
	\lhead{}
	\rhead{\textit{\rightmark}}
	\cfoot{\thepage}
}
\pagestyle{contentpage}

% table of content %
\makeatletter
\newcommand\mytoc{%
	\if@twocolumn
	\@restonecoltrue\onecolumn
	\else
	\@restonecolfalse
	\fi
	\toctrue
	\chapter*{\contentsname
		\@mkboth{%
			\contentsname}{\contentsname}}\addcontentsline{toc}{chapter}{Contents}%
	\tocfalse
	\@starttoc{toc}%
	\if@restonecol\twocolumn\fi
}
\makeatother

% chapter formatting %
\newif\iftoc
\titleformat
{\chapter} % command
[display] % shape
{\cabin} % font
{} % label
{2in} % 
{
	\raggedleft
	\iftoc
	\vspace{2in}
	\else
	{\LARGE\textsc{Chapter}~{\cantarell\thechapter}} \\
	\fi
	\Huge\scshape\bfseries
}
[
\vspace{-20pt}%
\rule{\textwidth}{0.1pt}
\vspace{0.0in}
]
\titlespacing{\chapter}
{0pt}
{
	\iftoc
	-100pt+1in
	\else
	-130pt+1in
	\fi
}
{0pt}

% hyperlink formatting %
\hypersetup{
	colorlinks,    
	linkcolor=hyperlinkblue,
	urlcolor=hyperlinkblue,
	pdftitle={...},
	pdfauthor={Michael Pham},
}
%%%%%

%%%%% ENVIRONMENTS STYLES
% purple box %
\declaretheoremstyle[
mdframed={
	backgroundcolor=lightpurple,
	linecolor=darkpurple,
	rightline=false,
	topline=false,
	bottomline=false,
	linewidth=2pt,
	innertopmargin=8pt,
	innerbottommargin=8pt,
	innerleftmargin=8pt,
	leftmargin=-2pt,
	skipbelow=2pt,
	nobreak
},
headfont=\normalfont\bfseries\color{darkpurple}
]{purplebox}

% green box %
\declaretheoremstyle[
mdframed={
	backgroundcolor=lightgreen,
	linecolor=darkgreen,
	rightline=false,
	topline=false,
	bottomline=false,
	linewidth=2pt,
	innertopmargin=8pt,
	innerbottommargin=8pt,
	innerleftmargin=8pt,
	leftmargin=-2pt,
	skipbelow=2pt,
	nobreak
},
headfont=\normalfont\bfseries\color{darkgreen}
]{greenbox}

% yellow box %
\declaretheoremstyle[
mdframed={
	backgroundcolor=lightyellow,
	linecolor=darkyellow,
	rightline=false,
	topline=false,
	bottomline=false,
	linewidth=2pt,
	innertopmargin=8pt,
	innerbottommargin=8pt,
	innerleftmargin=8pt,
	leftmargin=-2pt,
	skipbelow=2pt,
	nobreak
},
headfont=\normalfont\bfseries\color{darkyellow}
]{yellowbox}

% blue box %
\declaretheoremstyle[
mdframed={
	backgroundcolor=lightblue,
	linecolor=darkblue,
	rightline=false,
	topline=false,
	bottomline=false,
	linewidth=2pt,
	innertopmargin=8pt,
	innerbottommargin=8pt,
	innerleftmargin=8pt,
	leftmargin=-2pt,
	skipbelow=2pt,
	nobreak
},
headfont=\normalfont\bfseries\color{darkblue}
]{bluebox}

% red box %
\declaretheoremstyle[
mdframed={
	backgroundcolor=lightred,
	linecolor=darkred,
	rightline=false,
	topline=false,
	bottomline=false,
	linewidth=2pt,
	innertopmargin=8pt,
	innerbottommargin=8pt,
	innerleftmargin=8pt,
	leftmargin=-2pt,
	skipbelow=2pt,
	nobreak
},
headfont=\normalfont\bfseries\color{darkred}
]{redbox}
%%%%%

%%%%% ENVIRONMENTS
% SOLUTION ENVIRONMENT %
\newenvironment{solution}{\begin{proof}[Solution]}{\end{proof}}

% INNER PROOF ENVIRONMENT %
\newenvironment{innerproof}{\renewcommand{\qedsymbol}{$\square$}\proof}{\endproof}

% purple boxes (theorems, propositions, lemmas, and corollaries) %
\declaretheorem[style=purplebox,name=Theorem,within=chapter]{thm}
\declaretheorem[style=purplebox,name=Theorem,sibling=thm]{theorem}
\declaretheorem[style=purplebox,name=Theorem,numbered=no]{thm*, theorem*}
\declaretheorem[style=purplebox,name=Proposition,sibling=thm]{prop, proposition}
\declaretheorem[style=purplebox,name=Proposition,numbered=no]{prop*, proposition*}
\declaretheorem[style=purplebox,name=Lemma,sibling=thm]{lem, lemma}
\declaretheorem[style=purplebox,name=Lemma,numbered=no]{lem*, lemma*}
\declaretheorem[style=purplebox,name=Corollary,sibling=thm]{cor, corollary}
\declaretheorem[style=purplebox,name=Corollary,numbered=no]{cor*, corollary*}

% green boxes (definitions) %
\declaretheorem[style=greenbox,name=Definition,sibling=thm]{definition, defn}
\declaretheorem[style=greenbox,name=Definition,numbered=no]{definition*, defn*}

% blue boxes (problems) %
\declaretheorem[style=bluebox,name=Problem,numberwithin=chapter]{homework, hw}
\declaretheorem[style=bluebox,name=Problem,numbered=no]{homework*, hw*}

% red boxes (remarks) %
\declaretheorem[style=redbox,name=Remark,sibling=thm]{remark, rmk}
\declaretheorem[style=redbox,name=Remark, numbered=no]{remark*, rmk*}

% yellow boxes (warnings) %
\declaretheorem[style=yellowbox,name=Warning,sibling=thm]{warn}
\declaretheorem[style=yellowbox,name=Warning,numbered=no]{warn*}
%%%%%

%%%%% PROOF FORMATTING
\renewcommand\qedsymbol{$\blacksquare$}
%%%%%

%%%%% CUSTOM COMMANDS
% basic %
\newcommand{\Mod}[1]{\ (\mathrm{mod}\ #1)}
\newcommand{\floor}[1]{\left\lfloor{#1}\right\rfloor}
\newcommand{\ceil}[1]{\left\lceil{#1}\right\rceil}
\newcommand{\norm}[1]{\left\lVert{#1}\right\rVert}

% logic %
\newcommand*\xor{\oplus}
\newcommand{\all}{\forall}
\newcommand{\bland}{\bigwedge}
\newcommand{\blor}{\bigvee}
\newcommand*{\defeq}{\mathrel{\rlap{\raisebox{0.3ex}{$\m@th\cdot$}}\raisebox{-0.3ex}{$\m@th\cdot$}}=} \makeatother

% matrices %
\newcommand\aug{\fboxsep=- \fboxrule\!\!\!\fbox{\strut}\!\!\!}\makeatletter 

% sets %
\newcommand{\CC}{\mathbb{C}}
\newcommand{\NN}{\mathbb{N}}
\newcommand{\QQ}{\mathbb{Q}}
\newcommand{\RR}{\mathbb{R}}
\newcommand{\ZZ}{\mathbb{Z}}

% title %
\newcommand{\mytitle}[2]{%
	\title{#1}
	\author{Michael Pham}
	\date{#2}
	\maketitle
	\newpage
	\mytoc
	\newpage
}
%%%%%

%%%%% REDEFINING COMMANDS
%\let\oldint\int
%\renewcommand{\int}[2]{\oldint\limits_{#1}^{#2}}
%\let\oldprod\prod
%\renewcommand{\prod}[2]{\oldprod\limits_{#1}^{#2}}
%\let\oldsum\sum
%\renewcommand{\sum}[2]{\oldsum\limits_{#1}^{#2}}
%%%%%
%END_FOLD
%%%%%


\begin{document}
\mytitle{Math 110: Introduction to Linear Algebra}{Fall 2023}

\chapter{Vector Spaces}
\section{$\RR^{n}$ and $\CC^{n}$}
\subsection{Complex Numbers}
The notion of complex numbers was created in order to deal with negative roots; by definition, we have $i \coloneq \sqrt{-1}$. The following is the formal definition:

\begin{defn}[Complex Numbers]\label{Def: Complex Numbers}
	A \textbf{complex number} is defined as an ordered pair $(a,b)$, where $a, b \in \RR$. However, we will write it as $a + bi$.
	
	The set of complex numbers is denoted by $\CC$, and is defined as:
	\begin{equation*}
		\CC = \left\{  a + bi : a, b \in \RR\right\}.
	\end{equation*}

	We define addition and multiplication on $\CC$ as follow, for $a,b,c,d \in \RR$:
	\begin{align*}
		(a+bi) + (c + di) &= (a+c) + (b+d)i \\
		(a+bi)(c+di) &= (ac-bd)+(ad+bc)i.
	\end{align*}
\end{defn}

\begin{rmk}
	We note here that the reals $\RR$ can be thought of as a subset of $\CC$, where $b = 0$.
\end{rmk}

Using multiplication as defined above, one can verify that $i^{2} = -1$; we simply substitute in $a = 0$ and $b = 1$, and recall that $i = \sqrt{-1}$.

Furthermore, when it comes to multiplying, we can simply rederive the formula by recalling that $i = \sqrt{-1}$ (and thus $i^{2} = -1$).

\begin{thm}[Properties of Complex Arithmetic]\label{Thm: Complex Arithmetic}
	The following are properties of complex arithmetic, for all $\alpha, \beta, \lambda \in \CC$:
	
	\begin{enumerate}
		\item \textbf{Commutativity}: $\alpha + \beta = \beta + \alpha$ and $\alpha\beta = \beta\alpha$, for all $\alpha, \beta \in \CC$.
		
		\item \textbf{Associativity}: $(\alpha + \beta) + \lambda = \alpha + (\beta + \lambda)$ and $(\alpha\beta)\lambda = \alpha(\beta\lambda)$.
		
		\item \textbf{Identities}: $\lambda + 0 = \lambda$, and $\lambda(1) = \lambda$.
		
		\item \textbf{Additive Inverse}: For all $\alpha \in \CC$, there exists a $\beta \in \CC$ such that $\alpha + \beta = 0$.
		
		\item \textbf{Multiplicative Inverse}: For all $\alpha \in \CC$, where $\alpha \not= 0$, there exists a $\beta \in \CC$ such that $\alpha\beta = 1$.
		
		\item \textbf{Distributive Property}: $\lambda(\alpha + \beta) = \lambda\alpha + \lambda\beta$.
	\end{enumerate}
\end{thm}

\begin{proof}
	We will now prove the properties presented above. Before we proceed, let us define $\alpha = a + bi$, $\beta = c + di$, and $\lambda = e + fi$, where $a, b, c, d, e, f \in \RR$.
	
	\begin{enumerate}
		\item \textbf{Commutativity}
		\begin{innerproof}
			We first observe the following:
			\begin{align*}
				\alpha + \beta &= (a + bi) + (c + di) \\
				&= (a + c) + (b + d)i \\
				\beta + \alpha &= (c + di) + (a + bi) \\
				&= (c + a) + (d + b)i
			\end{align*}
		
			By commutativity of real numbers, we observe then that $(a + c) + (b + d)i = (c + a) + (d + b)i$, and thus we have that $\alpha + \beta = \beta + \alpha$.
			
			Next, we observe that
			\begin{align*}
				\alpha\beta &= (a+bi)(c+di) \\
				&= (ac - bd) + (ad + bc)i \\
				\beta\alpha &= (c+di)(a+bi) \\
				&= (ca - db) + (cb + da)i
			\end{align*}
		
			Once again, we observe that by commutativity of real numbers, we have that $(ac - bd) + (ad + bc)i = (ca - db) + (cb + da)i$, and thus we have that $\alpha\beta = \beta\alpha$ as desired.
		\end{innerproof}
	
		\item \textbf{Associativity}:
		\begin{innerproof}
			We first see the following:
			\begin{align*}
				(\alpha + \beta) + \lambda &= ( (a+bi) + (c + di)) + (e + fi) \\
				&= ( (a+c) + (b+d)i) + (e + fi) \\
				&= (a + c + e) + (b + d + f)i \\
				\alpha + (\beta + \lambda) &= (a + bi) + ( (c + di) + (e+fi)) \\
				&= (a+bi) + ( (c+e) + (d + f)i) \\
				&= (a + c + e) + (b + d +f)i
			\end{align*}
		
			Thus, we see that $(\alpha + \beta) + \lambda = \alpha + (\beta + \lambda)$.
			
			Next, we observe the following:
			\begin{align*}
				(\alpha\beta)\lambda &= ( (a+bi)(c+di))(e+fi) \\
				&= ( (ac - bd) + (ad + bc)i)(e + fi) \\
				&= (ace - bde - adf - bcf) + (ade + bce + acf - bdf)i \\
				\alpha(\beta\lambda) &= (a + bi)( (c+di)(e+fi)) \\
				&= (a+bi) ( (ce - df) + (cf + de)i) \\
				&= (ace -bde -adf - bcf) + (ade+bce+acf-bdf)i
			\end{align*}
		
			We get the last lines for $(\alpha\beta)\lambda$ and $\alpha(\beta\lambda)$ using commutativity of real numbers. Thus, we see the that $(\alpha\beta)\lambda = \alpha(\beta\lambda)$, as desired.
		\end{innerproof}
	
		\item \textbf{Identities}:
		\begin{innerproof}
			We first note the following definitions $0 = 0 + 0i$, and $1 = 1 + 0i$. Now, with this in mind, we see then that:
			\begin{align*}
				\lambda + 0 &= (e + fi) + (0 + 0i) \\
				&= (e+0) + (f+0)i \\
				&= e + fi \\
				&= \lambda \\
				\lambda(1) &= (e+fi)(1 + 0i) \\
				&= (e(1) - f(0)) + (e(0) + f(1))i \\
				&= e + fi \\
				&= \lambda
			\end{align*}
		\end{innerproof}
	
	\item \textbf{Additive Inverse}:
	\begin{innerproof}
		We observe that for every $\alpha \in \CC$, there exists an element $\beta = -\alpha = -(a + bi) = -a - bi$ such that $\alpha + (-\alpha) = 0$: 
		\begin{align*}
			\alpha + -\alpha &= (a + bi) + -(a + bi) \\
			&= (a + -a) + (b + -b)i \\
			&= 0 + 0i \\
			&= 0
		\end{align*}
		
		We note here that as $\CC$ is closed under addition and multiplication, then we have that $\beta = -\alpha \in \CC$.
	\end{innerproof}

	\item \textbf{Multiplicative Inverse}:
	\begin{innerproof}
		We observe that for every $\alpha \not= 0 \in \CC$, we have an element $\beta = \frac{1}{\alpha} = $
	\end{innerproof}
	\end{enumerate}

\end{proof}

\end{document}