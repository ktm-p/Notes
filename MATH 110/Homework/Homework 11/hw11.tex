	\documentclass{article}
%%%% PREAMBLE %%%%
%BEGIN_FOLD
%%% PACKAGES
\usepackage{amsmath}
\usepackage{amssymb}
\usepackage{amsthm}
\usepackage{cabin} % section title font
\usepackage[default]{cantarell} % default font
\usepackage[shortlabels]{enumitem}
\usepackage{fancyhdr}
\usepackage{graphicx}
\usepackage{hyperref}
\usepackage{mathtools}
\usepackage[framemethod=TikZ]{mdframed}
\usepackage[scr]{rsfso} % power set symbol
\usepackage{tasks} % vaguely remember this being important for something...?
\usepackage{tikz} % diagrams
\usepackage{titlesec}
\usepackage{thmtools}
\usepackage{varwidth}
\usepackage{verbatim} % longer comments
\usepackage{xcolor}
%%%

%%% COLOURS
\definecolor{darkgreen}{HTML}{19A514}
\definecolor{lightgreen}{HTML}{9DFF9A}
\definecolor{darkblue}{HTML}{3E5FE4}
\definecolor{lightblue}{HTML}{BCDEFF}
\definecolor{darkred}{HTML}{CC3333}
\definecolor{lightred}{HTML}{FFA9A9}
\definecolor{darkpurple}{HTML}{A933CD}
\definecolor{lightpurple}{HTML}{F0BAFF}
\definecolor{darkyellow}{HTML}{D2D22A}
\definecolor{lightyellow}{HTML}{FFFFAE}
\definecolor{hyperlinkblue}{HTML}{3366CC}
%%%

%%% PAGE SETUP
% BASIC %
\setlength\parindent{0pt} % paragraph indentation
\setlength{\parskip}{5pt} % spacing between paragraphs
\usepackage[margin=1in]{geometry} % margin size

% HEADER/FOOTER %
\pagestyle{fancy}
\fancyhf{}
\fancyfoot[R]{\thepage} % page number on bottom right
\fancyhead[R]{\textit{\leftmark}} % section title
\renewcommand{\headrulewidth}{0pt} % removing horizontal line at the top

% HYPERLINK FORMATTING %
\hypersetup{
	colorlinks,    
	linkcolor=hyperlinkblue,
	urlcolor=hyperlinkblue,
	pdftitle={...},
	pdfauthor={Michael Pham},
}

%%%

%%% ENVIRONMENTS STYLES
% SOLUTION ENVIRONMENT %
\newenvironment{solution}{\begin{proof}[Solution]}{\end{proof}}

% PURPLE BOX %
\declaretheoremstyle[
mdframed={
	backgroundcolor=lightpurple,
	linecolor=darkpurple,
	rightline=false,
	topline=false,
	bottomline=false,
	linewidth=2pt,
	innertopmargin=8pt,
	innerbottommargin=8pt,
	innerleftmargin=8pt,
	leftmargin=-2pt,
	skipbelow=2pt,
	nobreak
},
headfont=\normalfont\bfseries\color{darkpurple}
]{purplebox}

% GREEN BOX %
\declaretheoremstyle[
mdframed={
	backgroundcolor=lightgreen,
	linecolor=darkgreen,
	rightline=false,
	topline=false,
	bottomline=false,
	linewidth=2pt,
	innertopmargin=8pt,
	innerbottommargin=8pt,
	innerleftmargin=8pt,
	leftmargin=-2pt,
	skipbelow=2pt,
	nobreak
},
headfont=\normalfont\bfseries\color{darkgreen}
]{greenbox}

% YELLOW BOX %
\declaretheoremstyle[
mdframed={
	backgroundcolor=lightyellow,
	linecolor=darkyellow,
	rightline=false,
	topline=false,
	bottomline=false,
	linewidth=2pt,
	innertopmargin=8pt,
	innerbottommargin=8pt,
	innerleftmargin=8pt,
	leftmargin=-2pt,
	skipbelow=2pt,
	nobreak
},
headfont=\normalfont\bfseries\color{darkyellow}
]{yellowbox}

% BLUE BOX %
\declaretheoremstyle[
mdframed={
	backgroundcolor=lightblue,
	linecolor=darkblue,
	rightline=false,
	topline=false,
	bottomline=false,
	linewidth=2pt,
	innertopmargin=8pt,
	innerbottommargin=8pt,
	innerleftmargin=8pt,
	leftmargin=-2pt,
	skipbelow=2pt,
	nobreak
},
headfont=\normalfont\bfseries\color{darkblue}
]{bluebox}

% RED BOX %
\declaretheoremstyle[
mdframed={
	backgroundcolor=lightred,
	linecolor=darkred,
	rightline=false,
	topline=false,
	bottomline=false,
	linewidth=2pt,
	innertopmargin=8pt,
	innerbottommargin=8pt,
	innerleftmargin=8pt,
	leftmargin=-2pt,
	skipbelow=2pt,
	nobreak
},
headfont=\normalfont\bfseries\color{darkred}
]{redbox}
%%%

%%% ENVIRONMENTS
% PURPLE BOXES (theorems, propositions, lemmas, and corollaries) %
\declaretheorem[style=purplebox,name=Theorem,within=section]{thm}
\declaretheorem[style=purplebox,name=Theorem,sibling=thm]{theorem}
\declaretheorem[style=purplebox,name=Theorem,numbered=no]{thm*, theorem*}
\declaretheorem[style=purplebox,name=Proposition,sibling=thm]{prop, proposition}
\declaretheorem[style=purplebox,name=Proposition,numbered=no]{prop*, proposition*}
\declaretheorem[style=purplebox,name=Lemma,sibling=thm]{lem, lemma}
\declaretheorem[style=purplebox,name=Lemma,numbered=no]{lem*, lemma*}
\declaretheorem[style=purplebox,name=Corollary,sibling=thm]{cor, corollary}
\declaretheorem[style=purplebox,name=Corollary,numbered=no]{cor*, corollary*}

% GREEN BOXES (definitions) %
\declaretheorem[style=greenbox,name=Definition,sibling=thm]{definition, defn}
\declaretheorem[style=greenbox,name=Definition,numbered=no]{definition*, defn*}

% BLUE BOXES (problems) %
\declaretheorem[style=bluebox,name=Problem,numberwithin=section]{homework, hw}
\declaretheorem[style=bluebox,name=Problem,numbered=no]{homework*, hw*}

% RED BOXES %
\declaretheorem[style=redbox,name=Remark,sibling=thm]{remark, rmk}
\declaretheorem[style=redbox,name=Remark, numbered=no]{remark*, rmk*}
\declaretheorem[style=yellowbox,name=Warning,sibling=thm]{warn}
\declaretheorem[style=yellowbox,name=Warning,numbered=no]{warn*}
%%%

%%% PROOF FORMATTING
\renewcommand\qedsymbol{$\blacksquare$}
\newenvironment{innerproof}{\renewcommand{\qedsymbol}{$\square$}\proof}{\endproof}
%%%

%% CUSTOM COMMANDS
% basic %
\newcommand{\Mod}[1]{\ (\mathrm{mod}\ #1)}
\newcommand{\floor}[1]{\left\lfloor{#1}\right\rfloor}
\newcommand{\ceil}[1]{\left\lceil{#1}\right\rceil}
\newcommand{\norm}[1]{\left\lVert{#1}\right\rVert}

% logic %
\newcommand*\xor{\oplus}
\newcommand{\all}{\forall}
\newcommand{\bland}{\bigwedge}
\newcommand{\blor}{\bigvee}
\newcommand*{\defeq}{\mathrel{\rlap{\raisebox{0.3ex}{$\m@th\cdot$}}\raisebox{-0.3ex}{$\m@th\cdot$}}=} \makeatother

% matrices %
\newcommand\aug{\fboxsep=- \fboxrule\!\!\!\fbox{\strut}\!\!\!}\makeatletter 

% sets %
\newcommand{\CC}{\mathbb{C}}
\newcommand{\NN}{\mathbb{N}}
\newcommand{\QQ}{\mathbb{Q}}
\newcommand{\RR}{\mathbb{R}}
\newcommand{\ZZ}{\mathbb{Z}}

% probability stuff %
\newcommand{\E}{\mathbb{E}}
\newcommand{\Var}{\mathrm{Var}}
\newcommand{\var}{\mathrm{Var}}
\newcommand{\cov}{\mathrm{cov}}
\newcommand{\corr}{\mathrm{Corr}}

% linalg stuff %
\DeclareMathOperator*{\Span}{\mathrm{Span}}
\DeclareMathOperator*{\Null}{\mathrm{Null}}
\DeclareMathOperator*{\Range}{\mathrm{Range}}
\DeclareMathOperator*{\vspan}{\mathrm{span}}
\DeclareMathOperator*{\vnull}{\mathrm{null}}
\DeclareMathOperator*{\vrange}{\mathrm{range}}
\newcommand{\innerproduct}[2]{\left\langle{#1}, {#2}\right\rangle}
\DeclareMathOperator*{\proj}{\mathrm{proj}}

% title %
\newcommand{\mytitle}[2]{%
	\title{#1}
	\author{Michael Pham}
	\date{#2}
	\maketitle
	\newpage
	\tableofcontents
	\newpage
}
%%


%%%
%END_FOLD
%%%

\begin{document}
	\mytitle{Math 110: Homework 11}{Fall 2023}
	\section{Riesz's Theorem}
	\begin{hw}
		Find a polynomial $p \in \mathscr{P}_{3}(\RR)$ such that
		\begin{equation*}
			q'(1) = \int_{0}^{1} p(t) q(t) \mathrm{d}t \qquad \forall q \in \mathscr{P}_{3}(\RR).
		\end{equation*}
	\end{hw}
	\begin{solution}
		To begin with, let us define the following inner product for our space:
		\begin{equation*}
			\innerproduct{p}{q} \coloneq \int_{0}^{1} p(t) q(t) \mathrm dt.
		\end{equation*}
		
		Then, let us orthonormalise the standard basis $1, x, x^{2}, x^{3}$ of our space using Gram-Schmidt as follows:
		\begin{align*}
			e_{1} &= \frac{1}{\norm{1}} = \frac{1}{\sqrt{\innerproduct{1}{1}}} = \frac{1}{\sqrt{\int_{0}^{1} 1 \mathrm{d}x}} = \frac{1}{\sqrt{1}} = 1 \\
			e_{2} &= \frac{x - \innerproduct{x}{1}(1)}{\norm{x - \innerproduct{x}{1}(1)}} = \frac{x - \int_{0}^{1} x \mathrm dx}{\norm{x - \int_0^{1} x \mathrm dx}} = \frac{x - \frac{1}{2}}{\norm{x - \frac{1}{2}}} = \frac{x - \frac{1}{2}}{\sqrt{\innerproduct{x-\frac{1}{2}}{x - \frac{1}{2}}}} \\
			&= \frac{x - \frac{1}{2}}{\sqrt{\int_{0}^{1} (x-\frac{1}{2})^{2} \mathrm dx}} = \frac{x - \frac{1}{2}}{\sqrt{\int_{0}^{1} x^{2} - x + \frac{1}{4}}} = \frac{x - \frac{1}{2}}{\sqrt{\frac{1}{3} - \frac{1}{2} + \frac{1}{4}}} = \frac{x - \frac{1}{2}}{\sqrt{\frac{1}{12}}} = \sqrt{12}\left(x - \frac{1}{2}\right) \\
			e_{3} &= \frac{x^{2} - \innerproduct{x^{2}}{e_{2}}e_{2} - \innerproduct{x^{2}}{e_{1}}e_{1} }{\norm{x^{2} - \innerproduct{x^{2}}{e_{2}}e_{2} - \innerproduct{x^{2}}{e_{1}}e_{1}}} = \frac{x^{2} - \sqrt{12}\left( x - \frac{1}{2} \right)\int_{0}^{1} \sqrt{12}x^{2}\left( x - \frac{1}{2} \right) \mathrm dx- \int_{0}^{1} x^{2}\mathrm dx}{\norm{x^{2} - \sqrt{12}\left( x - \frac{1}{2} \right)\int_{0}^{1} \sqrt{12}x^{2}\left( x - \frac{1}{2} \right)\mathrm dx - \int_{0}^{1} x^{2} \mathrm dx}} \\
			&= \frac{x^{2} - \sqrt{12}(x - \frac 1 2)\int_{0}^{1}\sqrt{12}x^{3} - 	\frac{\sqrt{12}}{2}x^{2} \mathrm dx - \int_{0}^{1}x^{2} \mathrm dx}{\norm{x^{2} - \sqrt{12}(x - \frac 1 2)\int_{0}^{1}\sqrt{12}x^{3} - \frac{\sqrt{12}}{2}x^{2} \mathrm dx - \int_{0}^{1}x^{2} \mathrm dx}} = \frac{x^{2} - \sqrt{12}(x-\frac 1 2)(\frac{\sqrt{12}}{4} - \frac{\sqrt{12}}{6}) - \frac{1}{3}}{\norm{x^{2} - \sqrt{12}(x-\frac 1 2)(\frac{\sqrt{12}}{4} - \frac{\sqrt{12}}{6}) - \frac{1}{3}}} \\
			&= \frac{x^{2} - x + \frac{1}{6}}{\norm{x^{2} - x + \frac{1}{6}}} = \frac{x^{2} - x + \frac{1}{6}}{\sqrt{\innerproduct{x^{2} - x + \frac{1}{6}}{x^{2} - x + \frac{1}{6}}}} = \frac{x^{2} - x + \frac{1}{6}}{\sqrt{\int_{0}^{1} (x^{2} - x + \frac{1}{6})^{2} \mathrm dx}} \\
			& = \frac{x^{2} - x + \frac{1}{6}}{\sqrt{\int_{0}^{1} x^{4} - 2x^{3} + \frac{4}{3}x^{2} - \frac{1}{3}x + \frac{1}{36} \mathrm dx}} = \frac{x^{2} - x + \frac{1}{6}}{\sqrt{\frac{1}{5} - \frac{2}{4} + \frac{4}{9} - \frac{1}{6} + \frac{1}{36}}} = \frac{x^{2} - x + \frac{1}{6}}{\sqrt{\frac{1}{180}}} = \sqrt{180}\left( x^{2} - x + \frac{1}{6} \right) \\
			e_{4} &= \frac{x^{3} - \innerproduct{x^{3}}{e_{3}}e_{3} - \innerproduct{x^{3}}{e_{2}}e_{2} - \innerproduct{x^{3}}{e_{1}}e_{1}}{\norm{x^{3} - \innerproduct{x^{3}}{e_{3}}e_{3} - \innerproduct{x^{3}}{e_{2}}e_{2} - \innerproduct{x^{3}}{e_{1}}e_{1}}} = \cdots = \sqrt{2800}\left(x^{3} - \frac{3}{2}x^{2} + \frac{3}{5}x - \frac{1}{20}\right)
		\end{align*}
		
		Thus, $e_{1}, e_{2}, e_{3}, e_{4}$ is an orthonormal basis for our vector space. From here, let us define $\varphi(p) = p'(1)$.
		
		Then, using Riesz's Representation Theorem, we know that there exists a unique $p$ such that $\varphi(q) = \innerproduct{q}{p} = \innerproduct{p}{q}$.
		
		Then, using our orthonormal basis for $\mathscr{P}_{3}(\RR)$, we have
		\begin{align*}
			p &= \varphi(e_{1})e_{1} + \varphi(e_{2})e_{2} + \varphi(e_{3})e_{3} + \varphi(e_{4})e_{4} \\
			&= e_{1}'(1)e_{1} + e_{2}'(1)e_{2} + e_{3}'(1)e_{3} + e_{4}'(1)e_{4} \\
			&= 12\left( x-\frac{1}{2} \right) + 180\left( x^{2} - x + \frac{1}{6} \right) + 1680\left( x^{3} - \frac{3}{2}x^{2} + \frac{3}{5}x - \frac{1}{20} \right) \\
			&= 1680x^{3} - 2340x^{2} + 840x - 60
		\end{align*}
		
		Thus, we have that $p(x) = 1680x^{3} - 2340x^{2} + 840x - 60$.
	\end{solution}
	
	\newpage
	
	\section{Finding the Orthonormal Projection}
	\begin{hw*}[Setup]
		For the next questions in this section, let $V = C[-\pi, \pi]$ with the inner product
		\begin{equation*}
			\innerproduct{f}{g} \coloneq \int_{-\pi}^{\pi} f(t)\overline{g(t)}\mathrm dt
		\end{equation*}
		
		We will want to determine the orthogonal projection of the function $h(x) = \exp(2ix)$ on the given subspaces.
	\end{hw*}
	
	Before we proceed, we will first make note of the fact that all of the subspaces given will be of the form $\vspan(1, \cos x, \sin x, \ldots, \cos nx, \sin nx)$.
	
	Next, by Euler's formula, we have that $e^{2ix} = \cos(2x) + i\sin(2x)$.
	
	Next, we make the following claim:
	\begin{lem}
		The list $1, \sin(x), \cos(x), \ldots, \sin(nx), \cos(nx)$ is orthogonal under our given inner product $\int_{-\pi}^{\pi} f(t)\overline{g(t)}\mathrm dt$.
	\end{lem}
	\begin{innerproof}
		To begin with, we note that since our bounds are $[-\pi, \pi]$, then $\overline{\sin x} = \sin x$ and $\overline{\cos x} = \cos x$. Furthermore, $\overline{1} = 1$. So, we can disregard the conjugation sign and instead see that our inner product is simply
		\begin{equation*}
			\innerproduct{f}{g} = \int_{-\pi}^{\pi} f(t)g(t) \mathrm dt.
		\end{equation*}
		
		Then, first we note that $\int_{-\pi}^{\pi} \cos(ax) \mathrm dx = \int_{-\pi}^{\pi} \sin(ax) \mathrm dx = 0$. Thus, we see that $1$ is orthogonal to the rest of our list.
		
		From here, we observe that $\sin(a(-x))\cos(b(-x)) = \sin(-ax)\cos(-bx) = -\sin(ax)\cos(bx)$. Thus, we see that $\sin(ax)\cos(bx)$ is an odd function. It follows then that
		\begin{equation*}
			\innerproduct{\sin(ax)}{\cos(bx)} = \int_{-\pi}^{\pi}\sin(ax)\cos(bx) \mathrm dx = 0.
		\end{equation*}
		
		We see then that $\sin(ax)$ and $\cos(bx)$ are thus orthogonal to each other.
		
		Next, we note that for $a\neq b$, we have that $\sin(ax)\sin(bx) = \frac{1}{2}\left[ \cos\left( (a-b)x  \right) - \cos\left( (a+b)x \right) \right]$. Then, we see that
		\begin{align*}
			\innerproduct{\sin(ax)}{\sin(bx)} &= \int_{-\pi}^{\pi} \sin(ax)\sin(bx) \mathrm dx = \int_{-\pi}^{\pi} \frac{1}{2}\left[ \cos\left( (a-b)x  \right) - \cos\left( (a+b)x \right) \right] \mathrm dx \\
			&= \frac{1}{2}\int_{-\pi}^{\pi} \cos\left( (a-b)x \right) \mathrm dx - \frac{1}{2}\int_{-\pi}^{\pi} \cos\left( (a+b)x \right)\mathrm dx = 0
		\end{align*}
		
		Then, it follows that $\sin(ax)$ and $\sin(bx)$ are orthogonal to each other as well, where $a \neq b$.
		
		Similarly, we see that $\cos(ax)\cos(bx) = \frac{1}{2}\left[ \cos\left( (a-b)x  \right) + \cos\left( (a+b)x \right) \right]$. Then, we observe that
		\begin{align*}
			\innerproduct{\cos(ax)}{\cos(bx)} &= \int_{-\pi}^{\pi}\cos(ax)\cos(bx) \mathrm dx = \int_{-\pi}^{\pi} \frac{1}{2}\left[ \cos\left( (a-b)x  \right) + \cos\left( (a+b)x \right) \right] \mathrm dx \\
			&= \frac{1}{2}\int_{-\pi}^{\pi} \cos\left( (a-b)x \right) \mathrm dx + \frac{1}{2}\int_{-\pi}^{\pi} \cos\left( (a+b)x \right)\mathrm dx = 0
		\end{align*}
		
		Thus, we see that $\cos(ax), \cos(bx)$ are orthogonal to each other too. Therefore, we can conclude then that indeed, our list is orthogonal under the given inner product.
	\end{innerproof}
	
	\begin{comment}
		From here then, we want to simply normalise the basis for our given subspace. To do this, we make note of the following observation:
	\begin{lem}
		For $\cos(ax)$ and $\sin(ax)$, we have the following:
		\begin{align*}
			\norm{\cos(ax)} = \norm{\sin(ax)} = \sqrt{\pi}.
		\end{align*}
	\end{lem}
	\begin{innerproof}
		We observe the following:
		\begin{align*}
			\norm{\cos(ax)} &= \sqrt{\innerproduct{\cos(ax)}{\cos(ax)}} \\
			&= \sqrt{\int_{-\pi}^{\pi} \cos(ax)^{2}} \\
			&= \sqrt{\pi} \\
			&= \sqrt{\int_{-\pi}^{\pi} \sin(ax)^{2}} \\
			&= \sqrt{\innerproduct{\sin(ax)}{\sin(ax)}} \\
			&= \norm{\sin(ax)}.
		\end{align*}
	\end{innerproof}
	
	We also note here that $\norm{1} = \sqrt{\innerproduct{1}{1}} = \sqrt{2\pi}$.
	\end{comment}
	
	We will now keep all of this in mind as we proceed with answering the following questions.
	\begin{hw}
		Determine the orthogonal projection of $h(x)$ on the subspace $\vspan(1, \cos x, \sin x)$.
	\end{hw}
	\begin{solution}
		To begin with, by our lemma from earlier, we know that $1, \cos x, \sin x, \ldots, \cos nx, \sin nx$ are all orthogonal to each other. Then, because $h(x) = e^{2ix} = \cos 2x + i\sin 2x$, we note then that it is in fact orthogonal to the given subpsace. Therefore, we see that the orthogonal projection of $h(x)$ onto $U$ is simply $0$.
	\end{solution}
	
	\begin{hw}
		Determine the orthogonal projection of $h(x)$ onto the subspace
		\begin{equation*}
			U \coloneq \vspan(1, \cos x, \sin x, \cos 2x, \sin 2x).
		\end{equation*} 
	\end{hw}
	\begin{solution}
		\begin{comment}
			First, we can orthonormalise the basis to get:
		\begin{equation*}
			e_{1}, e_{2}, e_{3}, e_{4}, e_{5} = \frac{1}{\sqrt{2\pi}}, \frac{\cos x}{\sqrt{\pi}}, \frac{\sin x}{\sqrt{\pi}}, \frac{\cos 2x}{\sqrt{\pi}}, \frac{\sin 2x}{\sqrt{\pi}}.
		\end{equation*}
		
		From here, we observe that to find the orthogonal projection of $h(x)$ onto $U$, we consider the following:
		\begin{equation*}
			\innerproduct{h(x)}{e_{1}}e_{1} + \innerproduct{h(x)}{e_{2}}e_{2} + \ldots + \innerproduct{h(x)}{e_{5}}e_{5}.
		\end{equation*}
		
		However, using the property of inner product along with orthogonality of each of the $1, \cos x, \ldots, \sin 2x$, we can rewrite the above expression to the following:
		\begin{equation*}
			\innerproduct{\cos(2x)}{\frac{\cos(2x)}{\sqrt{\pi}}}\frac{\cos(2x)}{\sqrt{\pi}} + \innerproduct{i\sin(2x)}{\frac{\sin(2x)}{\sqrt{\pi}}}\frac{\sin(2x)}{\sqrt{\pi}}.
		\end{equation*}
		
		From here, we note that this becomes:
		\begin{align*}
			\frac{\cos(2x)}{\pi}\innerproduct{\cos(2x)}{\cos(2x)} + \frac{i\sin(2x)}{\pi}\innerproduct{\sin(2x)}{\sin(2x)} &= \frac{\cos(2x)}{\pi}\int_{-\pi}^{\pi} \cos(2x)^{2}\mathrm dx + \frac{i\sin(2x)}{\pi} \int_{-\pi}^{\pi} \sin(2x) \mathrm dx \\
			&= \frac{\pi\cos(2x)}{\pi} + i \frac{\pi\sin(2x)}{\pi} \\
			&= \cos(2x) + i\sin(2x) \\
			&= e^{2ix} \\
			&= h(x)
		\end{align*}
		\end{comment}
		We observe that since $h(x) = \cos 2x + i\sin 2x$, it in fact is an element of our given subspace. Thus, the orthogonal projection of $h(x)$ onto $U$ is simply $h(x) = \cos 2x + i\sin 2x$.
	\end{solution}
	
	\begin{hw}
		Determine the orthogonal projection of $h(x)$ onto the subspace
		\begin{equation*}
			U \coloneq \vspan(1, \cos x, \sin x, \ldots, \cos nx, \sin nx) \tag{for $n > 2$.}
		\end{equation*}
	\end{hw}
	\begin{solution}
		Again, we note that since $h(x)$ is in fact in the span of $U$, then the orthogonal projector of $h(x)$ onto $U$ will simply be $h(x)$ itself.
	\end{solution}
	
	\newpage
	
	\section{Minimisation I}
	\begin{hw}
		Find $p \in \mathscr{P}_{3}(\RR)$ such that $p(-1) = 0$, $p'(-1) = 0$, and the following is minimised:
		\begin{equation*}
			\int_{0}^{1} (1-5x-p(x))^{2} \mathrm dx.
		\end{equation*}
	\end{hw}
	\begin{solution}
		To begin with, let us define the inner product of our space to be
		\begin{equation*}
			\innerproduct{f}{g} = \int_{0}^{1} f(t)g(t) \mathrm dt.
		\end{equation*}
		
		From here, we want to minimise $\norm{1-5x-p(x)}$. To do this, we first define $U$ to be
		\begin{equation*}
			U \coloneq \left\{  p \in \mathscr{P}_{3}(\RR) : p(-1) = 0, p'(-1) = 0 \right\}.
		\end{equation*}
		
		Then, we can see that a basis for $U$ must be $(x+1)^{2}, (x+1)^{3}$.
		
		Now, using Gram-Schmidt, we can orthonormalise these vectors to get the following:
		\begin{align*}
			e_{1} &= \frac{(x+1)^{2}}{\norm{(x+1)^{2}}} = \sqrt{\frac{5}{31}}(x + 1)^{2} \\
			e_{2} &= \frac{(x+1)^{3} - \innerproduct{(x+1)^{3}}{e_{1}}e_{1}}{\norm{(x+1)^{3} - \innerproduct{(x+1)^{3}}{e_{1}}e_{1}}} = 2\sqrt{\frac{217}{313}}\left( (x+1)^{3} - \frac{105}{62}(x+1)^{2} \right).
		\end{align*}
		
		From here, let $q(x) = 1 - 5x$. Then, the closest point $p \in U$ to $q$ is:
		\begin{equation*}
			p = \innerproduct{q}{e_{1}}e_{1} + \innerproduct{q}{e_{2}}e_{2}.
		\end{equation*}
		
		Through computations, we get then that
		\begin{equation*}
			p(x) = - \frac{95}{124}(x+1)^{2} - \frac{791}{626}\left( (x+1)^{3} - \frac{105}{62}(x+1)^{2} \right).
		\end{equation*}
	\end{solution}
	
	\newpage
	
	\section{Minimisation II}
	\begin{hw}
		Find $p \in \mathscr{P}_{3}(\RR)$ such that $p(-1) = 0$, $p'(-1) = 0$, and the following is minimised:
		\begin{equation*}
			p(0)^{2} + \int_{0}^{1} (1-5x - p'(x))^{2} \mathrm dx.
		\end{equation*}
	\end{hw}
	\begin{solution}
		\begin{comment}
			To begin with, we will define the following inner product for our space:
			\begin{equation*}
				\innerproduct{f}{g} \coloneq f(0)g(0) + \int_{0}^{1} f'(t)g'(t) \mathrm dt. 
			\end{equation*}
			
			From here, we note that once again, the basis for $U$ will once again be $(x+1)^{2}, (x+1)^{3}$. Then, we let $q(x) = x - \frac{5}{2}x^{2}$ as before, and let
			\begin{equation*}
				p = \frac{\innerproduct{q}{(x+1)^{2}}}{\innerproduct{(x+1)^{2}}{(x+1)^{2}}}(x+1)^{2} + \frac{\innerproduct{q}{(x+1)^{3}}}{\innerproduct{(x+1)^{3}}{(x+1)^{3}}}(x+1)^{3}.
			\end{equation*}
			
			Through some computations, we will see then that we will get:
			\begin{equation*}
				p(x) = -\frac{16}{31}(x+1)^{2} - \frac{285}{1136}(x+1)^{3}. 
			\end{equation*}
		\end{comment}
		
		To begin with, we will define the following inner product for our space:
		\begin{equation*}
			\innerproduct{f}{g} \coloneq f(0)g(0) + \int_{0}^{1} f'(t)g'(t) \mathrm dt. 
		\end{equation*}
		
		From here, we note that once again, the basis for $U$ will once again be $(x+1)^{2}, (x+1)^{3}$. Then, we orthonormalise it again with Gram-Schmidt to get:
		\begin{align*}
			e_{1} &= \frac{(x+1)^{2}}{\norm{(x+1)^{2}}} = \sqrt{\frac{3}{31}}(x+1)^{2} \\
			e_{2} &= \frac{(x+1)^{3} - \innerproduct{(x+1)^{3}}{e_{1}}e_{1}}{\norm{(x+1)^{3} - \innerproduct{(x+1)^{3}}{e_{1}}e_{1}}} = 2\sqrt{\frac{155}{2081}}\left( (x+1)^{3} - \frac{141}{62}(x+1)^{2} \right).
		\end{align*}
		
		From here, we let $q(x) = x - \frac{5}{2}x^{2}$ as before, and let
		\begin{equation*}
			p = \innerproduct{q}{e_{1}}e_{1} + \innerproduct{q}{e_{2}}e_{2}.
		\end{equation*}
		
		Through some computations, we will see then that we will get:
		\begin{equation*}
			p(x) = -\frac{16}{31}(x+1)^{2} - \frac{1315}{2081}\left( (x+1)^{3} - \frac{141}{62}(x+1)^{2} \right).
		\end{equation*}
		
	\end{solution}
	
	\newpage
	
	\section{Orthogonal Projector}
	\begin{hw}
		Let $V = \RR^{3}$ equipped with the standard inner product. Prove or disprove: any linear operator $P \in \mathcal{L}(V)$ such that $P^{2} = P$ is an orthogonal projector.
	\end{hw}
	\begin{solution}
		This is false. Let us consider the following example:
		\begin{equation*}
			P = \begin{bmatrix}
				1 & 0 & 0 \\ 0 & 0 & 0 \\ 0 & 1 & 1
			\end{bmatrix}
		\end{equation*}
		
		First, note that
		\begin{equation*}
			P^{2} = P(P) = \begin{bmatrix}
				1 & 0 & 0 \\ 0 & 0 & 0 \\ 0 & 1 & 1
			\end{bmatrix}
			\begin{bmatrix}
				1 & 0 & 0 \\ 0 & 0 & 0 \\ 0 & 1 & 1
			\end{bmatrix}
			= \begin{bmatrix}
				1 & 0 & 0 \\ 0 & 0 & 0 \\ 0 & 1 & 1
			\end{bmatrix}
		\end{equation*}
		
		However, to confirm that it is not an orthogonal projector, we can consider the vector $v = (0,2,2)$. We note here that $\norm{v} = \sqrt{0(0) + 2(2) + 2(2)} = \sqrt{8}$.
		
		However, note that $Pv = (0,0,4)$. Note then that $\norm{Pv} = \sqrt{0^{2} + 0^{2} + 4^{2}} = \sqrt{16}$. Then, we see that $\norm{Pv} > \norm{v}$, thus violating one of the properties of an orthogonal projector (more concretely, it doesn't follow $\norm{Pv} \leq \norm{v}$ for all $v \in V$).
		
		Therefore, we can conclude that $P$ is not an orthogonal projector.
	\end{solution}
\end{document}