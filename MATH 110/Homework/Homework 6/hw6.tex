\documentclass{article}
%%%% PREAMBLE %%%%
%BEGIN_FOLD
%%% PACKAGES
\usepackage{amsmath}
\usepackage{amssymb}
\usepackage{amsthm}
\usepackage{cabin} % section title font
\usepackage[default]{cantarell} % default font
\usepackage[shortlabels]{enumitem}
\usepackage{fancyhdr}
\usepackage{graphicx}
\usepackage{hyperref}
\usepackage{mathtools}
\usepackage[framemethod=TikZ]{mdframed}
\usepackage[scr]{rsfso} % power set symbol
\usepackage{tasks} % vaguely remember this being important for something...?
\usepackage{tikz} % diagrams
\usepackage{titlesec}
\usepackage{thmtools}
\usepackage{varwidth}
\usepackage{verbatim} % longer comments
\usepackage{xcolor}
%%%

%%% COLOURS
\definecolor{darkgreen}{HTML}{19A514}
\definecolor{lightgreen}{HTML}{9DFF9A}
\definecolor{darkblue}{HTML}{3E5FE4}
\definecolor{lightblue}{HTML}{BCDEFF}
\definecolor{darkred}{HTML}{CC3333}
\definecolor{lightred}{HTML}{FFA9A9}
\definecolor{darkpurple}{HTML}{A933CD}
\definecolor{lightpurple}{HTML}{F0BAFF}
\definecolor{darkyellow}{HTML}{D2D22A}
\definecolor{lightyellow}{HTML}{FFFFAE}
\definecolor{hyperlinkblue}{HTML}{3366CC}
%%%

%%% PAGE SETUP
% BASIC %
\setlength\parindent{0pt} % paragraph indentation
\setlength{\parskip}{5pt} % spacing between paragraphs
\usepackage[margin=1in]{geometry} % margin size

% HEADER/FOOTER %
\pagestyle{fancy}
\fancyhf{}
\fancyfoot[R]{\thepage} % page number on bottom right
\fancyhead[R]{\textit{\leftmark}} % section title
\renewcommand{\headrulewidth}{0pt} % removing horizontal line at the top

% HYPERLINK FORMATTING %
\hypersetup{
	colorlinks,    
	linkcolor=hyperlinkblue,
	urlcolor=hyperlinkblue,
	pdftitle={...},
	pdfauthor={Michael Pham},
}

%%%

%%% ENVIRONMENTS STYLES
% SOLUTION ENVIRONMENT %
\newenvironment{solution}{\begin{proof}[Solution]}{\end{proof}}

% PURPLE BOX %
\declaretheoremstyle[
mdframed={
	backgroundcolor=lightpurple,
	linecolor=darkpurple,
	rightline=false,
	topline=false,
	bottomline=false,
	linewidth=2pt,
	innertopmargin=8pt,
	innerbottommargin=8pt,
	innerleftmargin=8pt,
	leftmargin=-2pt,
	skipbelow=2pt,
	nobreak
},
headfont=\normalfont\bfseries\color{darkpurple}
]{purplebox}

% GREEN BOX %
\declaretheoremstyle[
mdframed={
	backgroundcolor=lightgreen,
	linecolor=darkgreen,
	rightline=false,
	topline=false,
	bottomline=false,
	linewidth=2pt,
	innertopmargin=8pt,
	innerbottommargin=8pt,
	innerleftmargin=8pt,
	leftmargin=-2pt,
	skipbelow=2pt,
	nobreak
},
headfont=\normalfont\bfseries\color{darkgreen}
]{greenbox}

% YELLOW BOX %
\declaretheoremstyle[
mdframed={
	backgroundcolor=lightyellow,
	linecolor=darkyellow,
	rightline=false,
	topline=false,
	bottomline=false,
	linewidth=2pt,
	innertopmargin=8pt,
	innerbottommargin=8pt,
	innerleftmargin=8pt,
	leftmargin=-2pt,
	skipbelow=2pt,
	nobreak
},
headfont=\normalfont\bfseries\color{darkyellow}
]{yellowbox}

% BLUE BOX %
\declaretheoremstyle[
mdframed={
	backgroundcolor=lightblue,
	linecolor=darkblue,
	rightline=false,
	topline=false,
	bottomline=false,
	linewidth=2pt,
	innertopmargin=8pt,
	innerbottommargin=8pt,
	innerleftmargin=8pt,
	leftmargin=-2pt,
	skipbelow=2pt,
	nobreak
},
headfont=\normalfont\bfseries\color{darkblue}
]{bluebox}

% RED BOX %
\declaretheoremstyle[
mdframed={
	backgroundcolor=lightred,
	linecolor=darkred,
	rightline=false,
	topline=false,
	bottomline=false,
	linewidth=2pt,
	innertopmargin=8pt,
	innerbottommargin=8pt,
	innerleftmargin=8pt,
	leftmargin=-2pt,
	skipbelow=2pt,
	nobreak
},
headfont=\normalfont\bfseries\color{darkred}
]{redbox}
%%%

%%% ENVIRONMENTS
% PURPLE BOXES (theorems, propositions, lemmas, and corollaries) %
\declaretheorem[style=purplebox,name=Theorem,within=section]{thm}
\declaretheorem[style=purplebox,name=Theorem,sibling=thm]{theorem}
\declaretheorem[style=purplebox,name=Theorem,numbered=no]{thm*, theorem*}
\declaretheorem[style=purplebox,name=Proposition,sibling=thm]{prop, proposition}
\declaretheorem[style=purplebox,name=Proposition,numbered=no]{prop*, proposition*}
\declaretheorem[style=purplebox,name=Lemma,sibling=thm]{lem, lemma}
\declaretheorem[style=purplebox,name=Lemma,numbered=no]{lem*, lemma*}
\declaretheorem[style=purplebox,name=Corollary,sibling=thm]{cor, corollary}
\declaretheorem[style=purplebox,name=Corollary,numbered=no]{cor*, corollary*}

% GREEN BOXES (definitions) %
\declaretheorem[style=greenbox,name=Definition,sibling=thm]{definition, defn}
\declaretheorem[style=greenbox,name=Definition,numbered=no]{definition*, defn*}

% BLUE BOXES (problems) %
\declaretheorem[style=bluebox,name=Problem,numberwithin=section]{homework, hw}
\declaretheorem[style=bluebox,name=Problem,numbered=no]{homework*, hw*}

% RED BOXES %
\declaretheorem[style=redbox,name=Remark,sibling=thm]{remark, rmk}
\declaretheorem[style=redbox,name=Remark, numbered=no]{remark*, rmk*}
\declaretheorem[style=yellowbox,name=Warning,sibling=thm]{warn}
\declaretheorem[style=yellowbox,name=Warning,numbered=no]{warn*}
%%%

%%% PROOF FORMATTING
\renewcommand\qedsymbol{$\blacksquare$}
\newenvironment{innerproof}{\renewcommand{\qedsymbol}{$\square$}\proof}{\endproof}
%%%

%% CUSTOM COMMANDS
% basic %
\newcommand{\Mod}[1]{\ (\mathrm{mod}\ #1)}
\newcommand{\floor}[1]{\left\lfloor{#1}\right\rfloor}
\newcommand{\ceil}[1]{\left\lceil{#1}\right\rceil}
\newcommand{\norm}[1]{\left\lVert{#1}\right\rVert}

% logic %
\newcommand*\xor{\oplus}
\newcommand{\all}{\forall}
\newcommand{\bland}{\bigwedge}
\newcommand{\blor}{\bigvee}
\newcommand*{\defeq}{\mathrel{\rlap{\raisebox{0.3ex}{$\m@th\cdot$}}\raisebox{-0.3ex}{$\m@th\cdot$}}=} \makeatother

% matrices %
\newcommand\aug{\fboxsep=- \fboxrule\!\!\!\fbox{\strut}\!\!\!}\makeatletter 

% sets %
\newcommand{\CC}{\mathbb{C}}
\newcommand{\NN}{\mathbb{N}}
\newcommand{\QQ}{\mathbb{Q}}
\newcommand{\RR}{\mathbb{R}}
\newcommand{\ZZ}{\mathbb{Z}}

% probability stuff %
\newcommand{\E}{\mathbb{E}}
\newcommand{\Var}{\mathrm{Var}}
\newcommand{\var}{\mathrm{Var}}
\newcommand{\cov}{\mathrm{cov}}
\newcommand{\corr}{\mathrm{Corr}}

% linalg stuff %
\DeclareMathOperator*{\Span}{\mathrm{Span}}
\DeclareMathOperator*{\Null}{\mathrm{Null}}
\DeclareMathOperator*{\Range}{\mathrm{Range}}
\DeclareMathOperator*{\vspan}{\mathrm{span}}
\DeclareMathOperator*{\vnull}{\mathrm{null}}
\DeclareMathOperator*{\vrange}{\mathrm{range}}

% title %
\newcommand{\mytitle}[2]{%
	\title{#1}
	\author{Michael Pham}
	\date{#2}
	\maketitle
	\newpage
	\tableofcontents
	\newpage
}
%%


%%%
%END_FOLD
%%%

\begin{document}
	\mytitle{Homework 6}{Fall 2023}
	
	\section{Existence of Polynomial}
	\begin{hw}
		Suppose that $q \in \mathscr{P}(\RR)$. Prove that there exists a polynomial $p \in \mathscr{P}(\RR)$ such that
		\begin{equation*}
			q(x) = (x^{2} -3x)p''(x) + (2x-3)p'(x) + p(0).
		\end{equation*}
	\end{hw}
	\begin{solution}
		Let us suppose that we have some $q \in \mathscr{P}(\RR)$. Then, we know that there exists some $n \in \NN$ such that $q \in \mathscr{P}_{n}(\RR)$.	
		
		Now, we consider the linear map $T$ such that $T(p) = (x^{2} - 3x)p'' + (2x-3)p' + p(0)$, where we send a degree $n$ polynomial $p$ to a degree $n$ polynomial $T(p)$. We note that $T$ is indeed linear as it is the sum and composition of linear maps.
		
		Now, we want to show that this linear map is surjective. To do this, we will first determine the null space of $T$. 
		
		\begin{comment}
			We require $p(0) = 0$, $(2x-3)p' = 0$, and $(x^{2}-3x)p'' = 0$ for $Tp = 0$. 
			
			First, we want $p(0) = 0$. So, we know that for $p = \alpha x^{n} + \ldots + \beta x + \gamma$, we have $\gamma  = 0$ for $p(0) = 0$. Next, since $(2x-3)p' = 0$, then we see that $(2x-3)(n\alpha x^{n-1} + \ldots + \beta) = 2n \alpha x^{n} + \ldots + 2 \beta x - 3(n\alpha x^{n-1} + \ldots + \beta) = 0$. We observe then that for this to be the case, all of the coefficients $\alpha, \ldots, \beta, \gamma$ must all be equal to zero. In other words, we observe then that the only polynomial $p$ such that $Tp = 0$ is the zero-polynomial itself. Thus, we observe that $\vnull T = \left\{  0\right\}$.
		\end{comment}
	
		Then, we first notice that $((x^{2} - 3x)p')' = (x^{2} - 3x)p'' + (2x-3)p'$. Then, we want $Tp = 0$; then, we want $((x^{2} - 3x)p')' + p(0) = 0$.
		
		Then, we observe that we can integrate from $0$ to $x$ of both sides to get:
		\begin{align*}
			(x^{2}-3x)p' + p(0)x &= 0 \\
			(x^{2} - 3x)p' &= -p(0)x
		\end{align*}
	
		Then, we let $p = \alpha x^{n} + \beta x^{n-1} + \ldots + \gamma x + \lambda$. Then, we see the following:
		\begin{align*}
			(x^{2} - 3x)p' &= -p(0)x \\
			(x^{2} - 3x)(\alpha x^{n} + \beta x^{n-1} + \ldots + \gamma x + \lambda)' &= -\lambda x \\
			(x^{2} - 3x)(n \alpha x^{n-1} + (n-1)\beta x^{n-2} + \ldots + \gamma) &= -\lambda x \\
			(n\alpha x^{n+1} + (n-1)x^{n} + \ldots + \gamma x^{2}) - 3(n \alpha x^{n} + (n-1)\beta x^{n-1} + \ldots + \gamma x) &= - \lambda x
		\end{align*}
	
		Then, we see that the only way for this to be possible is if all coefficients $\alpha, \beta, \ldots, \gamma, \lambda$ are all equal to zero. Thus, we see that only the zero polynomial is in the null space of $T$.
		
		Now, we note that this means that $\dim \vnull T = 0$. Now, by the Rank-Nullity theorem, we observe the following:
		\begin{align*}
			n &= \dim \vnull T + \dim\vrange T \\
			&= 0 + \dim\vrange T 
		\end{align*}
	
		Thus, we see that $\dim\vrange T = n$; in other words, we see that $T$ is indeed surjective, and thus for some $q \in \mathscr{P}_{n}(\RR)$, there exists a $p \in \mathscr{P}_{n}(\RR)$ which satisfies our requirement.
	\end{solution}
	
	\newpage
	
	\section{Isomorphisms}
	\begin{hw}
		Let $V$ be a vector space over $\mathbb{F}$. Give a constructive proof that $V$ and $\mathcal L (\mathbb{F},V)$ are isomorphic.
	\end{hw}
	\begin{solution}
		Before we begin, we make the following observation:
		\begin{rmk}
			For $T \in \mathcal L(\mathbb{F}, V)$, we have that $T(a) = T(a \cdot 1) = aT(1)$, for $a \in \mathbb{F}$.
		\end{rmk}
	
		\begin{comment}
			Now, with this in mind, let us define some $\varphi : V \rightarrow \mathcal L(\mathbb{F}, V)$ as follow:
		\begin{equation*}
			\varphi(v) \coloneq T_{v},
		\end{equation*}
		where we define $T_{v} : \mathbb{F} \rightarrow V $, for $a \in \mathbb{F}$, as:
		\begin{equation*}
			T_{v}(a) \coloneq av.
		\end{equation*}
	
		Now, we will first for linearity of $\varphi$. First, we consider some $\alpha, \beta \in \mathbb{F}$ and $v, v' \in V$. Then,
		\begin{align*}
			\varphi(\alpha v + \beta v') &= T
		\end{align*}
		\end{comment}
	
		Now, with this in mind, let us define a map $\varphi : \mathcal L(\mathbb{F}, V) \rightarrow V$ by the following:
		\begin{equation*}
			\varphi(T) \coloneq T(1).
		\end{equation*}
	
		Now, we will first check for linearity of $\varphi$. To do this, we consider some $\alpha, \beta \in \mathbb{F}$ and $T, T' \in \mathcal L(\mathbb{F}, V)$. Then,
		\begin{align*}
			\varphi(\alpha T + \beta T') &= (\alpha T + \beta T')(1) \\
			&= \alpha T(1) + \beta T(1) \\
			&= \alpha \varphi(T) + \beta \varphi(T').
		\end{align*}
	
		Thus, we see that $\varphi$ is indeed linear.
		
		Now, we want to show that this map is bijective. To do this, we first look at injectivity.
		
		Let us suppose that for $T, T' \in \mathcal L(\mathbb{F}, V)$, we have that $\varphi (T) = \varphi(T')$. From here, for some $a \in \mathbb{F}$, we observe the following:
		\begin{align*}
			T(a) &= T(a \cdot 1) \\
			&= a T(1) \\
			&= a \varphi(T) \\
			&= a\varphi(T') \\
			&= a T'(1) \\
			&= T'(a)
		\end{align*}
	
		Thus, we see that for all $a \in \mathbb{F}$, we have that $T(a) = T'(a)$; in other words, $T = T'$. Therefore, we see that if $\varphi (T) = \varphi (T')$, then $T = T'$; $\varphi$ is indeed injective.
		\begin{comment}
			
		Now, for surjectivity, we can proceed as follow: let us consider some $v \in V$, and some $T \in \mathcal{L} (\mathbb{F}, V)$ such that $T(a) = av$, for $a \in \mathbb{F}$.
		
		We will check for linearity of this $T$ as follows: suppose we have $\lambda, \mu, a,b \in \mathbb{F}$. Then, we have
		\begin{align*}
			T(\lambda a + \mu b) &= (\lambda a + \mu b)(v) \\
			&= \lambda av + \mu bv \\
			&= \lambda T(a) + \mu T(b)
		\end{align*}
		
		Then, from here, we observe that
		\begin{align*}
			\varphi(T) &= T(1) \\
			&= v
		\end{align*}
		\end{comment}
	
		Now, for surjectivity, suppose we have some $v \in V$. Now, we let $T : \mathbb{F} \rightarrow V$ such that $T(a) = av$, for $a \in \mathbb{F}$. Then, from here, we observe the following:
		\begin{align*}
			\varphi(T) &= T(1) \\
			&= v
		\end{align*}
	
		In other words, we see that for all $v \in V$, there is always some map $T \in \mathcal L(\mathbb{F}, V)$ such that $\varphi(T) = v$. Thus, we see that $\varphi$ is indeed surjective.
		
		Therefore, with this in mind, we see that we have created a bijection between $\mathcal{L}(\mathbb{F}, V)$ and $V$. Thus, the two are indeed isomorphic.
		
	\end{solution}

	\newpage
	
	\section{Linear Functionals}
	Which of the following maps on $\mathscr{P}(\RR)$ are linear functionals?
	\begin{hw}
		\begin{equation*}
			p(x) \mapsto \int_{-1}^{x} p(t) \mathrm dt
		\end{equation*}
	\end{hw}
	\begin{solution}
		No; we observe that the map doesn't map from $\mathscr{P}(\RR)$ to $\RR$.
		\begin{comment}
			\begin{align*}
			\int_{-1}^{x} p(t)\mathrm dt &= P(t) \big|_{-1}^{x} \\
			&= P(x) - P(-1)
		\end{align*}
	
		We see that $P(x) - P(-1)$ doesn't evaluate to a real number. Thus, the map isn't a linear functional.
		\end{comment}
	\end{solution}

	\begin{hw}
		\begin{equation*}
			p(x) \mapsto \int_0^{1} p(4t^{10} + t^{3} - 1) \mathrm dt
		\end{equation*}
	\end{hw}
	\begin{solution}
		First, we note that the map above does indeed map from $\mathscr{P}(\RR)$ to $\RR$. Now, we want to check for linearity.
		
		To do this, we observe the following: consider $p,q \in \mathscr{P}(\RR)$ and $\alpha, \beta \in \RR$. Then, we see that $T(\alpha p+\beta q)$ is:
		\begin{align*}
			T(\alpha p+\beta q) &= \int_0^{1} (\alpha p+\beta q)(4t^{10} + t^{3} - 1) \mathrm dt \\
			&= \int_0^{1} (\alpha p)(4t^{10} + t^{3} - 1) + (\beta q)(4t^{10} + t^{3} - 1) \mathrm dt \\
			&= \alpha\int_0^{1} p(4t^{10} + t^{3} - 1) \mathrm dt + \beta\int_0^{1} q(4t^{10} + t^{3} - 1) \mathrm dt \\
			&= \alpha T(p) +\beta T(q)
		\end{align*}
	
		Thus, we see that $T$ is indeed linear as desired, and thus we can conclude that it is a linear functional.
	\end{solution}

	\begin{hw}
		\begin{equation*}
			p(x) \mapsto p(0)p''(\pi)
		\end{equation*}
	\end{hw}
	\begin{solution}
		First, we see that $T$ does indeed map from $\mathscr{P}(\RR)$ to $\RR$. Now, we want to check for linearity.
		
		Again, let us consider $p,q \in \mathscr{P}(\RR)$ and $\alpha, \beta \in \RR$. Then, we see that $T(\alpha p+\beta q)$ is:
		\begin{align*}
			T(\alpha p + \beta q) &= (\alpha p + \beta q)(0) \cdot (\alpha p + \beta q)''(\pi) \\
			&= (\alpha p(0) + \beta q(0))(\alpha p''(\pi) + \beta q''(\pi)) \\
			&= \alpha p(0)\alpha p''(\pi) + \alpha p(0)\beta q''(\pi) + \beta q(0)\alpha p''(\pi) + \beta q(0) \beta q''(\pi) \\
			&= \alpha (p(0)p''(\pi)) + \alpha\beta(p(0)q''(\pi) + q(0)p''(\pi)) + \beta(q(0)q''(\pi)) \\
			&\neq \alpha(p(0)p''(\pi)) + \beta(q(0)q''(\pi)) \\
			&= \alpha T(p) + \beta T(q)
		\end{align*}
	
		Thus, since $T(\alpha p + \beta q) \neq \alpha T(p) + \beta T(q)$, we see that $T$ is not linear, and thus is not a linear functional.
	\end{solution}

	\begin{hw}
		\begin{equation*}
			p(x) \mapsto 2p(1)
		\end{equation*}
	\end{hw}
	\begin{solution}
		We observe that this map $T$ does indeed map from $\mathscr{P}(\RR)$ to $\RR$. So, we will check for linearity.
		\begin{align*}
			T(\alpha p + \beta q) &= 2(\alpha p + \beta q)(1) \\
			&= 2(\alpha p(1) + \beta q(1)) \\
			&= 2 \alpha p(1) + 2 \beta q(1) \\
			&= \alpha 2p(1) + \beta 2q(1) \\
			&= \alpha T(p) + \beta T(q)
		\end{align*}
	
		Thus, we see that $T$ is indeed linear and therefore we see that it is indeed a linear functional.
	\end{solution}

	\newpage
	
	\section{Dual Basis}
	\begin{hw}
		Let $V = \mathscr{P}_{2}(\RR)$ and suppose that $\varphi_{j}(p) = p(j-1)$, $j = 1,2,3$. Prove that $(\varphi_1,\varphi_2,\varphi_3)$ is a basis for $\mathscr{P}_{2}(\RR)'$ and find a basis $(p_{1}, p_{2}, p_{3})$ of $\mathscr{P}_{2}(\RR)$ whose dual is $(\varphi_1,\varphi_2,\varphi_3)$.
	\end{hw}
	\begin{solution}
		We observe the following:
		\begin{equation*}
			\begin{bmatrix}
				\varphi_1(1) & \varphi_2(1) & \varphi_3(1) \\
				\varphi_{1}(x) & \varphi_2(x) &\varphi_3(x) \\
				\varphi_1(x^{2}) & \varphi_{2}(x^{2}) &\varphi_3(x^{2})
			\end{bmatrix}
			=
			\begin{bmatrix}
				1 & 1 & 1 \\
				0 & 1 & 2 \\
				0 & 1 & 4
			\end{bmatrix}
		\end{equation*}
	
		From here, we see that this matrix is indeed invertible, as we can row reduce it down to the identity matrix
		\begin{equation*}
			\begin{bmatrix}
				1 & 0 & 0 \\ 0 & 1 & 0 \\ 0 & 0 & 1
			\end{bmatrix}
		\end{equation*}
	
		Thus, we observe that, indeed, $\varphi_1, \varphi_2, \varphi_3$ are linearly independent, and thus they form a basis for $\mathscr{P}_{2}(\RR)'$. To find a basis $(p_{1}, p_{2}, p_{3})$ of $\mathscr{P}_{2}(\RR)$ which is dual to $(\varphi_1, \varphi_2, \varphi_3)$, we can choose the following:
		\begin{align*}
			p_{1} &= \dfrac{(x-1)(x-2)}{2} \\
			&= \frac{1}{2}(x^{2} - 3x + 2) \\
			p_{2} &= -x(x-2) \\
		%	&= -x^{2} + 2x \\
			&= -(x^{2} - 2x) \\
			p_{3} &= \dfrac{x(x-1)}{2} \\
			&= \dfrac{1}{2}(x^{2} - x)
		\end{align*}
	
		First, we check that these are indeed linearly independent. To do this, we suppose that only the trivial solution satisfies the following:
		\begin{equation*}
			ap_{1} + bp_{2} + cp_{3} = 0.
		\end{equation*}
	
		Then, we see that this gives us the following:
		\begin{align*}
			a\left(\frac{1}{2}(x^{2} - 3x + 2)\right) + b\left(-(x^{2} - 2x)\right) + c\left(\frac{1}{2}(x^{2} - x)\right) &= 0
		\end{align*}
	
		With this, we see that we have:
		\begin{align*}
			\dfrac{a}{2}x^{2} -bx^{2} + \dfrac{c}{2}x^{2} &= 0 \\
			-\dfrac{3a}{2} x + 2bx - \dfrac{c}{2}x  &= 0 \\
			a &= 0
		\end{align*}
	
		Then, from here, we have the following:
		\begin{align*}
			\dfrac{a}{2} - b + \dfrac{c}{2} &= 0 \\
			-\dfrac{3a}{2} + 2b - \dfrac{c}{2} &= 0 \\
			a &= 0 \\
			\\
			-b + \dfrac{c}{2} &= 0 \\
			2b - \dfrac{c}{2} &= 0 \\
			b &= 0
			\\
			\dfrac{c}{2} &= 0 \\
			c &= 0
		\end{align*}
	
		So, we know that these three vectors are indeed linearly independent, and thus forms a basis for $\mathscr{P}_{2}(\RR)$. Now, we want to verify that they are indeed dual to $\varphi_{1}, \varphi_{2}, \varphi_{3}$. To do this, we proceed as follows:
			
		\begin{align*}
			\begin{bmatrix}
				\varphi_{1}(p_{1}) & \varphi_{2}(p_{1}) & \varphi_{3}(p_{1}) \\
				\varphi_1(p_{2}) & \varphi_2(p_{2}) & \varphi_{3}(p_{2}) \\
				\varphi_1(p_{3}) & \varphi_2(p_{3}) & \varphi_3(p_{3})
			\end{bmatrix}
			&=
		%	\begin{equation*}
				\begin{bmatrix}
					p_{1}(0) & p_{1}(1) & p_{1}(2) \\
					p_{2}(0) & p_{2}(1) & p_{2}(2) \\
					p_{3}(0) & p_{3}(1) & p_{3}(2)
				\end{bmatrix} \\
		%	\end{equation*}
	&=
		\begin{bmatrix}
			1 & 0 & 0 \\ 0 & 1 & 0 \\ 0 & 0 & 1
		\end{bmatrix}
		\end{align*}
	
		Thus, we see that this basis $(p_{1}, p_{2}, p_{3})$ is dual to $(\varphi_{1}, \varphi_{2}, \varphi_3)$
	\end{solution}
	
	\newpage
	
	\section{A Funny Title}
	\begin{hw}
		Let $V$ be a finite-dimensional vector space, and let $U$ be its proper subspace. Prove that there exists some $\varphi \in V'$ such that $\varphi(u) = 0$ for all $u \in U$ but $\varphi \neq 0$.
	\end{hw}
	\begin{solution}
		Let us suppose that $V$ is finite-dimensional, and let $\dim V = n + m$.
		
		We begin by observing that since $U \subsetneq V$, and since $V$ is finite-dimensional, then we see that $U$ must be finite-dimensional too. So, let $v_{1}, \ldots, v_{n}$ be a basis for $U$.
		
		From here, we can extend this basis to be a basis of $V$. We denote this basis for $V$ by $v_{1}, \ldots, v_{n}, v_{n+1}, \ldots, v_{n+m}$.
		
		Now, we define a linear map $\varphi : V \rightarrow \mathbb{F}$ as follows:
		\begin{equation*}
			\varphi(v_{i}) =
			\begin{cases}
				0 & 1 \leq i \leq n \\
				1 & n+1 \leq i \leq n+m
			\end{cases}
		\end{equation*}
	
		Thus, we see the following:
		\begin{align*}
			\varphi\left( \sum_{i=1}^{n} a_{i}v_{i} \right) &= \sum_{i=1}^{n} \varphi(a_{i} v_{i}) \\
			&= \sum_{i=1}^{n} a_{i}\varphi(v_{i}) \\
			&= 0
		\end{align*}
	
		Thus, we see that there indeed exists some $\varphi \in V'$ such that $\varphi(u) = 0$ but $\varphi \neq 0$.
	\end{solution}

\end{document}