\documentclass{article}
%%%% PREAMBLE %%%%
%BEGIN_FOLD
%%% PACKAGES
\usepackage{amsmath}
\usepackage{amssymb}
\usepackage{amsthm}
\usepackage{cabin} % section title font
\usepackage[default]{cantarell} % default font
\usepackage[shortlabels]{enumitem}
\usepackage{fancyhdr}
\usepackage{graphicx}
\usepackage{hyperref}
\usepackage{mathtools}
\usepackage[framemethod=TikZ]{mdframed}
\usepackage[scr]{rsfso} % power set symbol
\usepackage{tasks} % vaguely remember this being important for something...?
\usepackage{tikz} % diagrams
\usepackage{titlesec}
\usepackage{thmtools}
\usepackage{varwidth}
\usepackage{verbatim} % longer comments
\usepackage{xcolor}
%%%

%%% COLOURS
\definecolor{darkgreen}{HTML}{19A514}
\definecolor{lightgreen}{HTML}{9DFF9A}
\definecolor{darkblue}{HTML}{3E5FE4}
\definecolor{lightblue}{HTML}{BCDEFF}
\definecolor{darkred}{HTML}{CC3333}
\definecolor{lightred}{HTML}{FFA9A9}
\definecolor{darkpurple}{HTML}{A933CD}
\definecolor{lightpurple}{HTML}{F0BAFF}
\definecolor{darkyellow}{HTML}{D2D22A}
\definecolor{lightyellow}{HTML}{FFFFAE}
\definecolor{hyperlinkblue}{HTML}{3366CC}
%%%

%%% PAGE SETUP
% BASIC %
\setlength\parindent{0pt} % paragraph indentation
\setlength{\parskip}{5pt} % spacing between paragraphs
\usepackage[margin=1in]{geometry} % margin size

% HEADER/FOOTER %
\pagestyle{fancy}
\fancyhf{}
\fancyfoot[R]{\thepage} % page number on bottom right
\fancyhead[R]{\textit{\leftmark}} % section title
\renewcommand{\headrulewidth}{0pt} % removing horizontal line at the top

% HYPERLINK FORMATTING %
\hypersetup{
	colorlinks,    
	linkcolor=hyperlinkblue,
	urlcolor=hyperlinkblue,
	pdftitle={...},
	pdfauthor={Michael Pham},
}

%%%

%%% ENVIRONMENTS STYLES
% SOLUTION ENVIRONMENT %
\newenvironment{solution}{\begin{proof}[Solution]}{\end{proof}}

% PURPLE BOX %
\declaretheoremstyle[
mdframed={
	backgroundcolor=lightpurple,
	linecolor=darkpurple,
	rightline=false,
	topline=false,
	bottomline=false,
	linewidth=2pt,
	innertopmargin=8pt,
	innerbottommargin=8pt,
	innerleftmargin=8pt,
	leftmargin=-2pt,
	skipbelow=2pt,
	nobreak
},
headfont=\normalfont\bfseries\color{darkpurple}
]{purplebox}

% GREEN BOX %
\declaretheoremstyle[
mdframed={
	backgroundcolor=lightgreen,
	linecolor=darkgreen,
	rightline=false,
	topline=false,
	bottomline=false,
	linewidth=2pt,
	innertopmargin=8pt,
	innerbottommargin=8pt,
	innerleftmargin=8pt,
	leftmargin=-2pt,
	skipbelow=2pt,
	nobreak
},
headfont=\normalfont\bfseries\color{darkgreen}
]{greenbox}

% YELLOW BOX %
\declaretheoremstyle[
mdframed={
	backgroundcolor=lightyellow,
	linecolor=darkyellow,
	rightline=false,
	topline=false,
	bottomline=false,
	linewidth=2pt,
	innertopmargin=8pt,
	innerbottommargin=8pt,
	innerleftmargin=8pt,
	leftmargin=-2pt,
	skipbelow=2pt,
	nobreak
},
headfont=\normalfont\bfseries\color{darkyellow}
]{yellowbox}

% BLUE BOX %
\declaretheoremstyle[
mdframed={
	backgroundcolor=lightblue,
	linecolor=darkblue,
	rightline=false,
	topline=false,
	bottomline=false,
	linewidth=2pt,
	innertopmargin=8pt,
	innerbottommargin=8pt,
	innerleftmargin=8pt,
	leftmargin=-2pt,
	skipbelow=2pt,
	nobreak
},
headfont=\normalfont\bfseries\color{darkblue}
]{bluebox}

% RED BOX %
\declaretheoremstyle[
mdframed={
	backgroundcolor=lightred,
	linecolor=darkred,
	rightline=false,
	topline=false,
	bottomline=false,
	linewidth=2pt,
	innertopmargin=8pt,
	innerbottommargin=8pt,
	innerleftmargin=8pt,
	leftmargin=-2pt,
	skipbelow=2pt,
	nobreak
},
headfont=\normalfont\bfseries\color{darkred}
]{redbox}
%%%

%%% ENVIRONMENTS
% PURPLE BOXES (theorems, propositions, lemmas, and corollaries) %
\declaretheorem[style=purplebox,name=Theorem,within=section]{thm}
\declaretheorem[style=purplebox,name=Theorem,sibling=thm]{theorem}
\declaretheorem[style=purplebox,name=Theorem,numbered=no]{thm*, theorem*}
\declaretheorem[style=purplebox,name=Proposition,sibling=thm]{prop, proposition}
\declaretheorem[style=purplebox,name=Proposition,numbered=no]{prop*, proposition*}
\declaretheorem[style=purplebox,name=Lemma,sibling=thm]{lem, lemma}
\declaretheorem[style=purplebox,name=Lemma,numbered=no]{lem*, lemma*}
\declaretheorem[style=purplebox,name=Corollary,sibling=thm]{cor, corollary}
\declaretheorem[style=purplebox,name=Corollary,numbered=no]{cor*, corollary*}

% GREEN BOXES (definitions) %
\declaretheorem[style=greenbox,name=Definition,sibling=thm]{definition, defn}
\declaretheorem[style=greenbox,name=Definition,numbered=no]{definition*, defn*}

% BLUE BOXES (problems) %
\declaretheorem[style=bluebox,name=Problem,numberwithin=section]{homework, hw}
\declaretheorem[style=bluebox,name=Problem,numbered=no]{homework*, hw*}

% RED BOXES %
\declaretheorem[style=redbox,name=Remark,sibling=thm]{remark, rmk}
\declaretheorem[style=redbox,name=Remark, numbered=no]{remark*, rmk*}
\declaretheorem[style=yellowbox,name=Warning,sibling=thm]{warn}
\declaretheorem[style=yellowbox,name=Warning,numbered=no]{warn*}
%%%

%%% PROOF FORMATTING
\renewcommand\qedsymbol{$\blacksquare$}
\newenvironment{innerproof}{\renewcommand{\qedsymbol}{$\square$}\proof}{\endproof}
%%%

%% CUSTOM COMMANDS
% basic %
\newcommand{\Mod}[1]{\ (\mathrm{mod}\ #1)}
\newcommand{\floor}[1]{\left\lfloor{#1}\right\rfloor}
\newcommand{\ceil}[1]{\left\lceil{#1}\right\rceil}
\newcommand{\norm}[1]{\left\lVert{#1}\right\rVert}

% logic %
\newcommand*\xor{\oplus}
\newcommand{\all}{\forall}
\newcommand{\bland}{\bigwedge}
\newcommand{\blor}{\bigvee}
\newcommand*{\defeq}{\mathrel{\rlap{\raisebox{0.3ex}{$\m@th\cdot$}}\raisebox{-0.3ex}{$\m@th\cdot$}}=} \makeatother

% matrices %
\newcommand\aug{\fboxsep=- \fboxrule\!\!\!\fbox{\strut}\!\!\!}\makeatletter 

% sets %
\newcommand{\CC}{\mathbb{C}}
\newcommand{\NN}{\mathbb{N}}
\newcommand{\QQ}{\mathbb{Q}}
\newcommand{\RR}{\mathbb{R}}
\newcommand{\ZZ}{\mathbb{Z}}

% probability stuff %
\newcommand{\E}{\mathbb{E}}
\newcommand{\Var}{\mathrm{Var}}
\newcommand{\var}{\mathrm{Var}}
\newcommand{\cov}{\mathrm{cov}}
\newcommand{\corr}{\mathrm{Corr}}

% linalg stuff %
\DeclareMathOperator*{\Span}{\mathrm{Span}}
\DeclareMathOperator*{\Null}{\mathrm{Null}}
\DeclareMathOperator*{\Range}{\mathrm{Range}}
\DeclareMathOperator*{\vspan}{\mathrm{span}}
\DeclareMathOperator*{\vnull}{\mathrm{null}}
\DeclareMathOperator*{\vrange}{\mathrm{range}}
\newcommand{\innerproduct}[2]{\left\langle{#1}, {#2}\right\rangle}
\DeclareMathOperator*{\proj}{\mathrm{proj}}

% title %
\newcommand{\mytitle}[2]{%
	\title{#1}
	\author{Michael Pham}
	\date{#2}
	\maketitle
	\newpage
	\tableofcontents
	\newpage
}
%%


%%%
%END_FOLD
%%%

\begin{document}
\mytitle{Math 110: Homework 12}{Fall 2023}

\section{Injectivity and Surjectivity}
\begin{hw}
	Let $T \in \mathcal{L}(V,W)$. Prove that
	\begin{enumerate}
		\item $T$ is injective if and only if $T^{*}$ is surjective.
		\item $T^{*}$ is injective if and only if $T$ is surjective.
	\end{enumerate}
\end{hw}
\begin{solution}
	To prove the first statement, we will work in reverse and suppose that $T^{*}$ is surjective. From here, we recall that $\vrange T^{*} = (\vnull T)^{\perp}$. This means the following:
	\begin{align*}
		T^{*} \text{ is surjective} &\iff \vrange T^{*} = V \\
		&\iff (\vnull T)^{\perp} = V \\
		&\iff \vnull T = \left\{  0 \right\} \\
		&\iff T \text{ is injective}
	\end{align*}

	For the second statement, we note that if we replaced $T$ with $T^{*}$, and note that $(T^{*})^{*} = T$, then we observe that first, $\vrange T = (\vnull T^{*})^{\perp}$, and also that this means then that:
	\begin{align*}
		T \text{ is surjective} &\iff \vrange T = V \\
		&\iff (\vnull T^{*})^{\perp} = V \\
		&\iff \vnull T^{*} = \left\{  0 \right\} \\
		&\iff T^{*} \text{ is injective}
	\end{align*}
\end{solution}

\newpage

\section{Self-Adjointness}
\begin{hw}
	Suppose $S,T \in \mathcal{L}(V)$ are self-adjoint. Prove that $ST$ is self-adjoint if and only if $ST = TS$.
\end{hw}
\begin{solution}
	To begin with, we proceed with the forwards direction. Suppose that $ST$ is self-adjoint. Then, by definition, we observe that $ST = (ST)^{*}$. Then, we observe the following:
	\begin{align*}
		ST &= (ST)^{*} \\
		&= T^{*}S^{*}
	\end{align*}

	However, we note that $T,S$ are also self-adjoint, meaning that $T^{*} = T$ and $S^{*} = S$. So, we have that $T^{*}S^{*} = TS$. Thus, we see that $ST = TS$ as desired.
	
	For the backwards direction, suppose that $ST = TS$. Furthermore, we note that in fact, $S, T$ are self-adjoint, so we have:
	\begin{align*}
		ST &= TS \\
		&= T^{*}S^{*} \\
		&= (ST)^{*}
	\end{align*}

	And since $ST = (ST)^{*}$, we see that indeed, $ST$ is self-adjoint as desired.
\end{solution}

\newpage

\section{Projector}
\begin{hw}
	Let $P \in \mathcal L(V)$ be such that $P^{2} = P$. Prove that there is a subspace $U$ of $V$ such that $P_{U} = P$ if and only if $P$ is self-adjoint.
\end{hw}
\begin{solution}
	We shall begin with the forward direction. Suppose that $P$ is an orthogonal projection $P_{U}$ onto a subspace $U$ of $V$. From here, we observe the following:
	\begin{align*}
		\innerproduct{P_{U}v}{w} &= \innerproduct{P_{U}v}{P_{U}w + (I - P_{U})w} \tag{Rewriting $Iw$ as $P_{U} + (I - P_{U})w$}\\
		&= \innerproduct{P_{U}v}{P_{U}w} + \innerproduct{P_{U}v}{(I - P_{U})w} \tag{Properties of Inner Product}\\
		&= \innerproduct{P_{U}v}{P_{U}w} \tag{$P_{U}v \in U$, but $w - P_{U}w \in U^{\perp}$}\\
		&= \innerproduct{P_{U}v}{P_{U}w} + \innerproduct{(I - P_{U})v}{P_{U}w} \\
		&= \innerproduct{P_{U}v + (I - P_{U})v}{P_{U}w} \\
		&= \innerproduct{v}{P_{U}w}
	\end{align*}

	Thus, we see that, indeed, $P$ is self-adjoint as desired.
	
	For the backwards direction, let us suppose that $P$ is self-adjoint. Then, by definition, we see that $P = P^{*} = P^{2}$. Furthermore, we have that $\innerproduct{Pv}{w} = \innerproduct{w}{Pv}$.
	
	Now, with this in mind, let us denote $U = \vrange P$. From here, let us take $v,w \in V$ and we observe the following:
	\begin{align*}
		\innerproduct{Pv}{(I - P)w} &= \innerproduct{(I-P)^{*}Pv}{w} \\
		&= \innerproduct{(I-P)Pv}{w} \\
		&= \innerproduct{Pv - P^{2}v}{w} \\
		&= \innerproduct{Pv - Pv}{w} \\
		&= \innerproduct{0}{w} \\
		&= 0
	\end{align*}

	Then, we note that since the inner product is equal to zero, this maens then that $Pv$ and $(I - P)w$, for $v, w \in V$, are orthogonal to each other. Since $v,w$ are arbitrary, we observe that this means that $P$ is orthogonal to $I - P$.
	
	From here, we note that for any $v \in V$, we have that $v = u + w = Pv + (I - P)v$. And we note that since we let $U = \vrange P$, then $Pv \in U$, and $(I - P)v \in U^{\perp}$. Thus, by definition, we observe that, indeed, $P$ is an orthogonal projection $P_{U}$.
\end{solution}

\newpage

\section{Anti-Hermitian}
\begin{hw}
	Let $n \in \NN$ be fixed. Consider the real space $V \coloneq \vspan\left\{  1, \cos x, \sin x,\ldots, \cos nx, \sin nx \right\}$, equipped with the inner product space
	\begin{equation*}
		\innerproduct{f}{g} \coloneq \int_{-\pi}^{\pi} f(x) g(x) \mathrm dx.
	\end{equation*}

	Show that the differentiation operator $D \in \mathcal L(V)$ satisfies $D^{*} = -D$. 
\end{hw}
\begin{solution}
	To begin with, we note that all functions in $V$ are periodic with a period of $2\pi$. In other words, we have that $f(\pi) = f(-\pi)$ for $f \in V$.
	
	Now, with this in mind, we observe the following:
	\begin{align*}
		\innerproduct{Df}{g} &= \int_{-\pi}^{\pi} f'(x)g(x) \mathrm dx \\
		&= f(x)g(x)\big|_{-\pi}^{\pi} - \int_{-\pi}^{\pi} f(x)g'(x) \mathrm dx \\
		&= f(\pi)g(\pi) - f(-\pi)g(-\pi) - \int_{-\pi}^{\pi} f(x)g'(x)\mathrm dx \\
		&= f(\pi)g(\pi) - f(\pi)g(\pi) - \int_{-\pi}^{\pi} f(x)g'(x)\mathrm dx \\
		&= -\int_{-\pi}^{\pi} f(x)g'(x)\mathrm dx \\
		&= \innerproduct{f}{(-D)g}
	\end{align*}

	Thus, we observe that, indeed, we have that $D^{*} = -D$ as desired.
\end{solution}

\newpage

\section{A Normal Problem}
\begin{hw}
	Suppose that $T$ is normal. Prove that, for any $\lambda \in \mathbb{F}$ and any $k \in \NN$, we have:
	\begin{equation*}
		\vnull (T-\lambda I)^{k} = \vnull (T-\lambda I).
	\end{equation*}
\end{hw}
\begin{solution}
	We note that since $T$ is normal, then we have that $S = (T - \lambda I)$ is normal as well. From here, we will first introduce the following lemma, then instead shall prove a more general result.
	
	To begin with, we introduce the following lemma:
	\begin{lem}
		Given that $T$ is a normal operator, and for any $k \in \NN$ then we have:
		\begin{equation*}
			(TT^{*})^{k} = T^{k}(T^{*})^{k}
		\end{equation*}
	\end{lem}
	\begin{innerproof}
		We shall proceed with induction.
		
		To begin with, we observe that for the base case of $k = 1$, we have:
		\begin{equation*}
			(TT^{*})^{1} = TT^{*} = T^{1}(T^{*})^{1}.
		\end{equation*}
	
		Thus, our claim holds for $k = 1$. For $k = 2$, we see that
		\begin{align*}
			(TT^{*})^{2} &= TT^{*}TT^{*} \\
			&= TTT^{*}T^{*} \\
			&= T^{2}(T^{*})^{2}
		\end{align*}
		
		Now, suppose that for $n = k > 2$, our claim holds. We now will show that it holds for $n = k + 1$ as well. To do this, we first observe the following:
		\begin{align*}
			(TT^{*})^{k+1} &= (TT^{*})^{k}(TT^{*}) \\
			&= T^{k}(T^{*})^{k}(T^{*}T) \\
			&= T^{k}(T^{*})^{k+1}T
		\end{align*}
	
		Then from here, we can simply switch the $T$ and $T^{*}$ with each other repeatedly, moving the $T$ inside until we eventually get $T^{k+1}(T^{*})^{k+1}$. This is possible due to $T$ being normal.
		
		Thus, we have proven our claim.
	\end{innerproof}

	We note here that $(T^{*}T)^{k} = (T^{*})^{k}T^{k}$; we simply replace $T$ with $T^{*}$ in the lemma above (and use the fact that $(T^{*})^{*} = T$) to see that this is the case.

	We will also introduce the following observation:
	\begin{rmk}
		If we have some $T$ which is self-adjoint, and we consider some $v$ such that $T^{k}v = 0$, then we see the following:
		\begin{align*}
			\innerproduct{T^{k}v}{T^{k-2}v} &= \innerproduct{T^{k-1}v}{T^{k-1}v} \\
			&= 0
		\end{align*}
	
		Then, we observe that $T^{k-1}v$ must be equal to $0$, since $\innerproduct{v}{v} = 0$ if and only if $v = 0$.
		
		Then, from here, we can simply keep on repeating this process recursively until we hit $Tv = 0$.
	\end{rmk}

	Now, we will prove a more generalised lemma:
	\begin{lem}
		Given that $T$ is a normal operator, then $\vnull T^{k} = \vnull T$.
	\end{lem}
	\begin{innerproof}
		To do this, we first show that $\vnull T \subseteq \vnull T^{k}$.
		
		For this inclusion, we note that for any $v \in \vnull T$, we have that $Tv = 0$. Then, it follows that $T^{k} = T^{k-1}(Tv) = T^{k-1}(0) = 0$. Thus, we see that the inclusion holds.
		
		Next, for the other inclusion, we want to show that $\vnull T^{k} \subseteq \vnull T$. To do this, let us consider some $v \in \vnull T^{k}$.
		
		Then, from here, we have that $T^{k}v = 0$. Then, this means that $(T^{*}T)^{k}v = (T^{*})^{k}T^{k}v = (T^{*})^{k}0 = 0$.
		
		Thus, we have that $(T^{*}T)^{k}v = 0$.
		
		Now, we note here that since $T$ is normal, we have that $(T^{*}T)^{*} = T^{*}(T^{*})^{*} = T^{*}T$. In other words, $T^{*}T$ is self-adjoint.
		
		From our observation in Remark 5.2, we observe then that, in fact, we also have that $T^{*}Tv = 0$. Then, with this in mind, we do the following:
		\begin{align*}
			\innerproduct{T^{*}Tv}{v} &= \innerproduct{Tv}{Tv} \\
			&= 0
		\end{align*}
	
		And since $\innerproduct{v}{v} = 0$ if and only if $v = 0$, we see that, in fact, we have that $Tv = 0$. In other words, given that $v \in \vnull T^{k}$, we have that $v \in \vnull T$ as well.
		
		Thus, we have that $\vnull T^{k} \subseteq \vnull T$.
	
		Then, since we have these two inclusions, we can in fact conclude that they are equal.
	\end{innerproof}

	Now, with this lemma proven, we note that since $S$ is normal, then we can apply the lemma above to $S$ and thus we have proven our claim as desired.
\end{solution}

\end{document}