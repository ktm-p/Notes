\documentclass{article}
%%%%%%% PREAMBLE %%%%%%%
%BEGIN_FOLD
%%%%% PACKAGES
\usepackage{amsmath}
\usepackage{amssymb}
\usepackage{amsthm}
\usepackage{cabin} % section title font
\usepackage[default]{cantarell} % default font
\usepackage[shortlabels]{enumitem}
\usepackage{fancyhdr}
\usepackage{graphicx}
\usepackage{hyperref}
\usepackage{mathtools}
\usepackage[framemethod=TikZ]{mdframed}
\usepackage[scr]{rsfso} % power set symbol
\usepackage{tasks} % vaguely remember this being important for something...?
\usepackage{tikz} % diagrams
\usepackage{titlesec}
\usepackage{thmtools}
\usepackage{varwidth}
\usepackage{verbatim} % longer comments
\usepackage{xcolor}
%%%%%

%%%%% COLOURS
\definecolor{darkgreen}{HTML}{19A514}
\definecolor{lightgreen}{HTML}{9DFF9A}
\definecolor{darkblue}{HTML}{3E5FE4}
\definecolor{lightblue}{HTML}{BCDEFF}
\definecolor{darkred}{HTML}{CC3333}
\definecolor{lightred}{HTML}{FFA9A9}
\definecolor{darkpurple}{HTML}{A933CD}
\definecolor{lightpurple}{HTML}{F0BAFF}
\definecolor{darkyellow}{HTML}{D2D22A}
\definecolor{lightyellow}{HTML}{FFFFAE}
\definecolor{hyperlinkblue}{HTML}{3366CC}
%%%%%

%%%%% PAGE SETUP
% BASIC %
\setlength\parindent{0pt} % paragraph indentation
\setlength{\parskip}{5pt} % spacing between paragraphs
\usepackage[margin=1in]{geometry} % margin size

% HEADER/FOOTER %
\pagestyle{fancy}
\fancyhf{}
\fancyfoot[R]{\thepage} % page number on bottom right
\fancyhead[R]{\textit{\leftmark}} % section title
\renewcommand{\headrulewidth}{0pt} % removing horizontal line at the top

% HYPERLINK FORMATTING %
\hypersetup{
	colorlinks,    
	linkcolor=hyperlinkblue,
	urlcolor=hyperlinkblue,
	pdftitle={...},
	pdfauthor={Michael Pham},
}

%%%%%

%%%%% ENVIRONMENTS STYLES
% SOLUTION ENVIRONMENT %
\newenvironment{solution}{\begin{proof}[Solution]}{\end{proof}}

% PURPLE BOX %
\declaretheoremstyle[
mdframed={
	backgroundcolor=lightpurple,
	linecolor=darkpurple,
	rightline=false,
	topline=false,
	bottomline=false,
	linewidth=2pt,
	innertopmargin=8pt,
	innerbottommargin=8pt,
	innerleftmargin=8pt,
	leftmargin=-2pt,
	skipbelow=2pt,
	nobreak
},
headfont=\normalfont\bfseries\color{darkpurple}
]{purplebox}

% GREEN BOX %
\declaretheoremstyle[
mdframed={
	backgroundcolor=lightgreen,
	linecolor=darkgreen,
	rightline=false,
	topline=false,
	bottomline=false,
	linewidth=2pt,
	innertopmargin=8pt,
	innerbottommargin=8pt,
	innerleftmargin=8pt,
	leftmargin=-2pt,
	skipbelow=2pt,
	nobreak
},
headfont=\normalfont\bfseries\color{darkgreen}
]{greenbox}

% YELLOW BOX %
\declaretheoremstyle[
mdframed={
	backgroundcolor=lightyellow,
	linecolor=darkyellow,
	rightline=false,
	topline=false,
	bottomline=false,
	linewidth=2pt,
	innertopmargin=8pt,
	innerbottommargin=8pt,
	innerleftmargin=8pt,
	leftmargin=-2pt,
	skipbelow=2pt,
	nobreak
},
headfont=\normalfont\bfseries\color{darkyellow}
]{yellowbox}

% BLUE BOX %
\declaretheoremstyle[
mdframed={
	backgroundcolor=lightblue,
	linecolor=darkblue,
	rightline=false,
	topline=false,
	bottomline=false,
	linewidth=2pt,
	innertopmargin=8pt,
	innerbottommargin=8pt,
	innerleftmargin=8pt,
	leftmargin=-2pt,
	skipbelow=2pt,
	nobreak
},
headfont=\normalfont\bfseries\color{darkblue}
]{bluebox}

% RED BOX %
\declaretheoremstyle[
mdframed={
	backgroundcolor=lightred,
	linecolor=darkred,
	rightline=false,
	topline=false,
	bottomline=false,
	linewidth=2pt,
	innertopmargin=8pt,
	innerbottommargin=8pt,
	innerleftmargin=8pt,
	leftmargin=-2pt,
	skipbelow=2pt,
	nobreak
},
headfont=\normalfont\bfseries\color{darkred}
]{redbox}
%%%%%

%%%%% ENVIRONMENTS
% PURPLE BOXES (theorems, propositions, lemmas, and corollaries) %
\declaretheorem[style=purplebox,name=Theorem,within=section]{thm}
\declaretheorem[style=purplebox,name=Theorem,sibling=thm]{theorem}
\declaretheorem[style=purplebox,name=Theorem,numbered=no]{thm*, theorem*}
\declaretheorem[style=purplebox,name=Proposition,sibling=thm]{prop, proposition}
\declaretheorem[style=purplebox,name=Proposition,numbered=no]{prop*, proposition*}
\declaretheorem[style=purplebox,name=Lemma,sibling=thm]{lem, lemma}
\declaretheorem[style=purplebox,name=Lemma,numbered=no]{lem*, lemma*}
\declaretheorem[style=purplebox,name=Corollary,sibling=thm]{cor, corollary}
\declaretheorem[style=purplebox,name=Corollary,numbered=no]{cor*, corollary*}

% GREEN BOXES (definitions) %
\declaretheorem[style=greenbox,name=Definition,sibling=thm]{definition, defn}
\declaretheorem[style=greenbox,name=Definition,numbered=no]{definition*, defn*}

% BLUE BOXES (problems) %
\declaretheorem[style=bluebox,name=Problem,numberwithin=section]{homework, hw}
\declaretheorem[style=bluebox,name=Problem,numbered=no]{homework*, hw*}

% RED BOXES %
\declaretheorem[style=redbox,name=Remark,sibling=thm]{remark, rmk}
\declaretheorem[style=redbox,name=Remark, numbered=no]{remark*, rmk*}
\declaretheorem[style=yellowbox,name=Warning,sibling=thm]{warn}
\declaretheorem[style=yellowbox,name=Warning,numbered=no]{warn*}
%%%%%

%%%%% PROOF FORMATTING
\renewcommand\qedsymbol{$\blacksquare$}
%%%%%

%%% CUSTOM COMMANDS
% basic %
\newcommand{\Mod}[1]{\ (\mathrm{mod}\ #1)}
\newcommand{\floor}[1]{\left\lfloor{#1}\right\rfloor}
\newcommand{\ceil}[1]{\left\lceil{#1}\right\rceil}
\newcommand{\norm}[1]{\left\lVert{#1}\right\rVert}

% logic %
\newcommand*\xor{\oplus}
\newcommand{\all}{\forall}
\newcommand{\bland}{\bigwedge}
\newcommand{\blor}{\bigvee}
\newcommand*{\defeq}{\mathrel{\rlap{\raisebox{0.3ex}{$\m@th\cdot$}}\raisebox{-0.3ex}{$\m@th\cdot$}}=} \makeatother

% matrices %
\newcommand\aug{\fboxsep=- \fboxrule\!\!\!\fbox{\strut}\!\!\!}\makeatletter 

% sets %
\newcommand{\CC}{\mathbb{C}}
\newcommand{\NN}{\mathbb{N}}
\newcommand{\QQ}{\mathbb{Q}}
\newcommand{\RR}{\mathbb{R}}
\newcommand{\ZZ}{\mathbb{Z}}

% probability stuff %
\newcommand{\E}{\mathbb{E}}
\newcommand{\Var}{\mathrm{Var}}
\newcommand{\var}{\mathrm{Var}}
\newcommand{\cov}{\mathrm{cov}}
\newcommand{\corr}{\mathrm{Corr}}

% title %
\newcommand{\mytitle}[2]{%
	\title{#1}
	\author{Michael Pham}
	\date{#2}
	\maketitle
	\newpage
	\tableofcontents
	\newpage
}
%%%
%%%%%
%END_FOLD
%%%%%

\begin{document}
\mytitle{Math 110: Homework 1}{Fall 2023}

\section{Complex Solutions to Polynomials}
\begin{hw}
	Determine, with explanation, which complex numbers $x$ satisfy the equation:
	\begin{equation*}
		x^{3} + x^{2} + x + 1 = 0
	\end{equation*}
\end{hw}
\begin{solution}
	We begin by first grouping the expression as follow: $x^{2}(x+1) + 1(x+1)$. We see then that the polynomial above can be factored into $(x^{2}+1)(x+1) = 0$.
	
	With this in mind, we see that the solutions to the equation are ones that satisfies $x+1 = 0$ or $x^{2} + 1 = 0$.
	
	For $x+1=0$, we get:
	\begin{align*}
		x + 1 &= 0 \\
		x &= -1 \\
	\end{align*}

	For $x^{2} + 1 = 0$, we get:
	\begin{align*}
		x^{2} + 1 &= 0 \\
		x^{2} &= -1 \\
		x &= \pm\sqrt{-1} \\
		x &= \pm i
	\end{align*}

	Therefore, the solutions to the given equation are $x = -i, i, 1$.
\end{solution}

\newpage

\section{Finding $\lambda$}
\begin{hw}
	Does there exist a complex number $\lambda$ such that
	\begin{equation*}
		\lambda(2-3i, 5+4i, -6+7i) = (2+i, 3-i, 4)?
	\end{equation*}

	Justify your answer.
\end{hw}
\begin{solution}
\begin{comment}
We want to find some complex number $\lambda = a + bi$ such that the following holds:
\begin{align*}
	(a + bi)(2-3i) &= 2+i \\
	(a+bi)(5+4i) &= 3-i \\
	(a+bi)(-6+7i) &= 4
\end{align*}

With this in mind, we have the following:
\begin{align*}
	(2a + 3b) + (-3a + 2b)i &= 2+i \\
	(5a - 4b) + (4a + 5b)i &= 3 - i \\
	(-6a - 7b) + (7a - 6b)i &= 4 + 0i \\
\end{align*}

From here, we can see from the last line that $(7a - 6b)i = 0i \implies 7a - 6b = 0 \implies 7a = 6b$. Then, we have that $a = \frac{6}{7}b$.

Through substituting, we get $-\frac{36}{7}b - \frac{49}{7}b = 4$. So, we get $-\frac{85}{7}b = 4$, we $b = -\frac{28}{85}$.

Then, we must have that $a = -\frac{4(7)}{85}\frac{6}{7} = -\frac{24}{85}$.

However, when checking if these solutions work with the first line, we see that $2a + 3b = -\frac{132}{85} \not= 2$.
\end{comment}
We observe that in order for $\lambda(-6 + 7i) = 4$ to occur, then it must mean that $\lambda = \frac{4}{-6 + 7i}$. In order words, we have:
\begin{align*}
	\lambda &= \frac{4}{-6 + 7i} \\
	&= \frac{4(-6-7i)}{(-6+7i)(-6-7i)} \\
	&= \frac{-24 - 28i}{36 + 49} \\
	&= -\frac{24}{85} -\frac{28}{85}i
\end{align*}

However, we see from here that $(2-3i)(-\frac{24}{85} - \frac{28}{85}i) \not= 2 + i$. Thus, there does not exist such a $\lambda$.
\end{solution}

\newpage

\section{Fields}
\begin{hw}
	Suppose that $\left\{  0, 1, x\right\}$ is a field with exactly three elements. What do the addition and multiplication tables have to be in that case? Based on the addition and multiplication tables you get, check this is
	indeed a field. What is the ``natural" way to think of this field (and of $x$)?
\end{hw}

\begin{solution}
	We define $\mathbb{F}_{3}$ to be a field with elements $\left\{  0, 1, x\right\}$ and the operations $(+, \cdot)$.
	
	First, we will deal with addition. We know that $0 + 0 = 0, 0 + 1 = 1, 0 + x = x$. 
	
	Next, we know that $1 + 0 = 1$. 

	We will now want to find out what $1 + 1$ is equal to. First, we suppose that $1 + 1 = 0$ for contradiction. If this is the case, then we have the following scenarios for $1 + x$:
	\begin{enumerate}
		\item $1 + x = 0$, which implies that $x = 1$ which is a contradiction as $x \not= 1$.
		
		\item $1 + x = 1$, which implies that $x = 0$. Again, this is a contradiction as $x \not= 0$.
		
		\item $1 + x = x$. In this case, this implies that $1 = 0$, which again is a contradiction.
	\end{enumerate}
	
	 So, we know that $1 + 1 \not= 0$. Now, we suppose for contradiction that $1 + 1 = 1$. For $1 + 1 = 1$, we must have then that $1 = 0$; but this is a contradiction, as $1 \not= 0$. Therefore, it must be that $1 + 1 = x$.
	 
	 To find what $1 + x$ is equal to, we first assume for contradiction that $1 + x = 1$. Then, this means that $x = 0$, which is a contradiction as $x \not= 0$. Similarly, if $1 + x = x$, then $1 = 0$ which again is a contradiction. Therefore, we must have that $1 + x = 0$.
	
	From here, we see that $x + 0 = x$. 
	
	We now want to find what $x + 1$ is: suppose for contradiction that $x + 1 = 1$, then $x = 0$ which is a contradiction. If $x+1 = x$, then we have that $1 = 0$, which also leads to a contradiction. Thus, we see that $x + 1 = 0$.
	
	Finally, for $x + x$, we suppose for contradiction that $x + x = 0$. However, for this to be the case, one of the $x$'s must be equal to 1 as established previously, which is a contradiction. Similarly, if it's equal to $x$ then we must have that $x = 0$ for one of the $x$'s, which again is a contradiction as $x \not = 0$. Thus, we conclude that $x + x = 1$. 
	
	Next, we will deal with multiplication. 
	
	For multiplication involving $0$, we have $0(0) = 0, 0(1) = 0, 0(x) = 0$.
	
	Similarly, in the case of $1$, we have that $1(0) = 0, 1(1) = 1, 1(x) = x$.
	
	For $x$, we see that $x(0) = 0$ and $x(1) = x$. 
	
	Now, we only have the case of $x(x)$ to deal with.
	
	We know by the field axioms that any element $a \in \mathbb{F}_{3}$ where $a \not= 0$ must have a multiplicative inverse $a^{-1}$; that is, $aa^{-1} = 1$. However, we see that since $x(0) = 0$ and $x(1) = x$, then it follows that $x(x) = 1$; $x$ is its own multiplicative inverse.
	
	Then, the addition table for $\mathbb{F}_{3}$ is as follow:
	\begin{center}
		\begin{tabular}{c|c c c}
			+ & 0 & 1 & x \\
			\hline
			0 & 0 & 1 & x \\
			1 & 1 & x & 0 \\
			x & x & 0 & 1
		\end{tabular}
	\end{center}

	And we have the multiplication table for $\mathbb{F}_{3}$ to be:
	\begin{center}
		\begin{tabular}{c|c c c}
			$\cdot$ & 0 & 1 & x \\
			\hline
			0 & 0 & 0 & 0 \\
			1 & 0 & 1 & x \\
			x & 0 & x & 1
		\end{tabular}
	\end{center}

	Since every entry in the tables consist of only elements in $\left\{  0, 1, x\right\}$, we see that closure under addition and multiplication holds for $\mathbb{F}_{3}$. We also see that we have the elements 0 (the additive identity) and 1 (the multiplicative identity).

	Furthermore, since the tables are symmetric along the main diagonal, commutativity holds.
	
	Also, since every row/column contains a 1 in the addition table, we see that the additive inverse exists. For the multiplication table, every non-zero row/column has a 1 as well, so a multiplicative inverse exists.
	
	
	To check for associativity and distributivity, we first remark the fact that the field $\mathbb{F}_{3}$ can actually be thought of as the ring $\ZZ_{3}$, with $x \equiv 2 \pmod{3}$. Then, we observe the following tables for $\ZZ_{3}$:
	
	\begin{center}
		\begin{tabular}{c|c c c}
			+ & 0 & 1 & 2 \\
			\hline
			0 & 0 & 1 & 2 \\
			1 & 1 & 2 & 0 \\
			2 & 2 & 0 & 1
		\end{tabular}
	\end{center}
	\begin{center}
	\begin{tabular}{c|c c c}
		$\cdot$ & 0 & 1 & 2 \\
		\hline
		0 & 0 & 0 & 0 \\
		1 & 0 & 1 & 2 \\
		2 & 0 & 2 & 1
	\end{tabular}
\end{center}
	
	Then, with this in mind, we see that associativity and distributivity must hold for $\mathbb{F}_{3}$ as well, as they hold for $\ZZ_{3}$.
\end{solution}

\newpage

\section{Differentiable Functions}
\begin{hw}
	Suppose $a$ is a fixed real number, and consider the set of all real-valued twice differential functions $f$ on the interval $\left[0, \infty\right)$ such that $f''(2) + af'(1) - af(0) = 2a$ (equipped with the usual addition of functions
	and multiplication by real scalars). For which values of a is this a vector space over $\RR$? Justify your answer.
\end{hw}
\begin{solution}
	Let us consider the set of all real-valued, twice differential functions $f$ on $\left[0, \infty\right)$ where $f''(2) + af'(1) - af(0) = 2a$. We want to find the value(s) of $a$ which would form a vector space.
	
	First, let us take functions $g, h$ which satisfies the equation. From here, we want to find when $g + h$ is also a solution to the differential equation given; i.e., we want there to be closure under addition.
	
	With this in mind, we observe the following:
	\begin{align*}
		(g+h)''(2) + a(g+h)'(1) - a(g+h)(0) &= g''(2) + h''(2) + ag'(1) + ah'(1) - ag(0) - ah(0) \\
		&= (g''(2) + ag'(1) - ag(0)) + (h''(2) + ah'(1) - ah(0)) \\
		&= 2a + 2a \\
		&= 4a \\
		&\not= 2a
	\end{align*}
	
	We observe that the only way for $4a = 2a$ to be true is if $a = 0$. Now, with $a = 0$, our equation becomes $f''(2) = 0$.
	
	Now, we will verify the other properties of a vector space for $a = 0$. To begin with, consider the functions $g, h, j$ on the interval $\left[ 0, \infty \right)$ which satisfies the equation given, along with scalars $\lambda, \mu \in \RR$. We now observe the following:
	\begin{itemize}
		\item Commutativity:
		\begin{align*}
			(g+h)''(2) &= g''(2) + h''(2) \\
			&= 0 \\
			&= h''(2) + g''(2) \\
			&= (h+g)''(2)
		\end{align*}
	
		\item Addition Associativity:
		\begin{align*}
			(g + (h + j))''(2) &= g''(2) + (h''(2) + j''(2)) \\
			&= 0 + (0 + 0) \\
			&= 0 \\
			&= (0 + 0) + 0 \\
			&= (g''(2) + h''(2)) + j''(2) \\
			&= ( (g+h) + j)''(2)
		\end{align*}
	
		\item Additive Identity: Let us take the function $f(x) = 0$ for all $x \in \left[ 0, \infty \right)$. Now, we observe the following:
		\begin{align*}
			(f + g)''(x) &= f''(x) + g''(x) \\
			&= 0 + g''(x) \\
			&= g''(x) \\
			&= 0
		\end{align*}
	
		\item Additive Inverse: Let us take the function $f(x) = -g(x)$. Then, we see the following:
		\begin{align*}
			(f+g)''(2) &= f''(2) + g''(2) \\
			&= -g(2)'' + g''(2) \\
			&= 0
		\end{align*}
	
		\item Scalar Multiplication Associativity:
		\begin{align*}
			\lambda \cdot (\mu \cdot f''(2)) &= \lambda \cdot (\mu \cdot 0) \\
			&= \lambda \cdot 0 \\
			&= 0 \\
			&= (\lambda \cdot \mu) \cdot 0 \\
			&= (\lambda \cdot \mu) \cdot f''(2)
		\end{align*}
	
		\item Scalar Multiplication Identity: Let us take $\lambda = 1$. Then, we see that
		\begin{align*}
			\lambda \cdot f''(2) &= 1 \cdot f''(2) \\
			&= f''(2) \\
			&= 0
		\end{align*}
		
		\item Distributivity: For distributivity, we first observe the following:
		\begin{align*}
			\lambda \cdot ( (f+g)''(2)) &= \lambda \cdot (f''(2) + g''(2)) \\
			&= \lambda \cdot (0) \\
			&= 0 \\
			&= 0 + 0 \\
			&= (\lambda \cdot 0) + (\lambda \cdot 0) \\
			&= (\lambda \cdot f''(2)) + (\lambda \cdot g''(2))
		\end{align*}
	
		We also see that
		\begin{align*}
			(\lambda + \mu) \cdot f''(2) &= (\lambda + \mu) \cdot 0 \\
			&= 0 \\
			&= 0 + 0 \\
			&= (\lambda \cdot 0) + (\mu \cdot 0) \\
			&= (\lambda \cdot f''(2)) + (\mu \cdot f''(2))
		\end{align*}
	\end{itemize}

	Thus, we see that only for $a=0$ do we have a vector space over $\RR$. 
\end{solution}

\newpage

\section{Functions and Vector Spaces}
\begin{hw}
	Suppose $S$ is a non-empty set and $V$ is a vector space. Let $V^{S}$ denote the set of functions from $S$ to $V$. Define a natural addition and scalar multiplication on $V^{S}$, and show that $V^{S}$ is a vector space with these definitions.
\end{hw}
\begin{solution}
	Let us suppose that we have a non-empty set $S$, and we shall denote $V^{S}$ as the set of functions from $S$ to $V$.
	
	Now, suppose we have functions $f, g \in V^{S}$. For some $s \in S$, let us define addition to be $(f \oplus g)(s) = f(s) + g(s)$. For scalar multiplication, for some scalar $\lambda \in \mathbb{F}$, we define it as $(\lambda \otimes f)(s) = \lambda \cdot f(s)$.
	
	From here, we will want to show that $V^{S}$ form a vector space with the provided definitions. Let us consider the functions $f,g,h \in V^{S}$, and the scalars $\lambda, \mu \in \mathbb{F}$.
	
	Now, we will first deal with addition:
	\begin{itemize}
		\item To begin with, for commutativity we have $(f \oplus g)(s) = f(s) + g(s) = g(s) + f(s) = (g \oplus f)(s)$.
	
		\item For associativity, we observe that $(f \oplus (g \oplus h))(s) = f(s) + (g(s) + h(s)) = (f(s) + g(s)) + h(s) = ((f \oplus g) \oplus h)(s)$.
		
		\item To show that there exists an additive identity, we simply define some function $f(s) = 0$, for all $s \in S$. We see then that $(f \oplus g)(s) = 0 + g(s) = g(s)$. Thus, we have the additive identity in $V^{S}$ as desired.
		
		\item To show that there exists an additive inverse, we define $f(s) = (-1)g(s)$. We see that $(f \oplus g)(s) = (-1)g(s) + g(s) = 0$, as desired.
	\end{itemize}

	Next, we look at multiplication. We observe the following:
	\begin{itemize}
		\item For associativity, we see that $( (\lambda \cdot \mu) \otimes f)(s) = (\lambda \cdot \mu) \cdot f(s) = \lambda \cdot (\mu \cdot f(s)) = (\lambda \cdot (\mu \otimes f))(s)$.
		
		\item We observe that for the multiplicative identity, we have $(1 \otimes f)(s) = 1 \cdot f(s) = f(s)$, where $1 \in \mathbb{F}$, and thus the multiplicative identity element in $V$.
	\end{itemize}
	
	Finally, we check distributivity:
	\begin{itemize}
		\item We observe that $(\lambda \otimes (f \oplus g))(s) = \lambda \cdot (f(s) + g(s)) = (\lambda \cdot f(s)) + (\lambda \cdot g(s))$.
		\item We also have that $( (\lambda + \mu) \otimes f)(s) = (\lambda + \mu) \cdot f(s) = (\lambda \cdot f(s)) + (\mu \cdot f(s))$.
	\end{itemize}

	Thus, we see that all of the vector space axioms are satisfied, and thus $V^{S}$ is a vector space with these definitions.
\end{solution}

\end{document}