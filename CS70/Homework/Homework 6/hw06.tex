\documentclass{article}
%%%%%%% PREAMBLE %%%%%%%
%BEGIN_FOLD
%%%%% PACKAGES
\usepackage{amsmath}
\usepackage{amssymb}
\usepackage{amsthm}
\usepackage{cabin} % section title font
\usepackage[default]{cantarell} % default font
\usepackage[shortlabels]{enumitem}
\usepackage{fancyhdr}
\usepackage{graphicx}
\usepackage{hyperref}
\usepackage{mathtools}
\usepackage[framemethod=TikZ]{mdframed}
\usepackage[scr]{rsfso} % power set symbol
\usepackage{tasks} % vaguely remember this being important for something...?
\usepackage{tikz} % diagrams
\usepackage{titlesec}
\usepackage{thmtools}
\usepackage{varwidth}
\usepackage{verbatim} % longer comments
\usepackage{xcolor}
%%%%%

%%%%% COLOURS
\definecolor{darkgreen}{HTML}{19A514}
\definecolor{lightgreen}{HTML}{9DFF9A}
\definecolor{darkblue}{HTML}{3E5FE4}
\definecolor{lightblue}{HTML}{BCDEFF}
\definecolor{darkred}{HTML}{CC3333}
\definecolor{lightred}{HTML}{FFA9A9}
\definecolor{darkpurple}{HTML}{A933CD}
\definecolor{lightpurple}{HTML}{F0BAFF}
\definecolor{darkyellow}{HTML}{D2D22A}
\definecolor{lightyellow}{HTML}{FFFFAE}
\definecolor{hyperlinkblue}{HTML}{3366CC}
%%%%%

%%%%% PAGE SETUP
% BASIC %
\setlength\parindent{0pt} % paragraph indentation
\setlength{\parskip}{5pt} % spacing between paragraphs
\usepackage[margin=1in]{geometry} % margin size

% HEADER/FOOTER %
\pagestyle{fancy}
\fancyhf{}
\fancyfoot[R]{\thepage} % page number on bottom right
\fancyhead[R]{\textit{\leftmark}} % section title
\renewcommand{\headrulewidth}{0pt} % removing horizontal line at the top

% HYPERLINK FORMATTING %
\hypersetup{
	colorlinks,    
	linkcolor=hyperlinkblue,
	urlcolor=hyperlinkblue,
	pdftitle={...},
	pdfauthor={Michael Pham},
}

%%%%%

%%%%% ENVIRONMENTS STYLES
% SOLUTION ENVIRONMENT %
\newenvironment{solution}{\begin{proof}[Solution]}{\end{proof}}

% PURPLE BOX %
\declaretheoremstyle[
mdframed={
	backgroundcolor=lightpurple,
	linecolor=darkpurple,
	rightline=false,
	topline=false,
	bottomline=false,
	linewidth=2pt,
	innertopmargin=8pt,
	innerbottommargin=8pt,
	innerleftmargin=8pt,
	leftmargin=-2pt,
	skipbelow=2pt,
	nobreak
},
headfont=\normalfont\bfseries\color{darkpurple}
]{purplebox}

% GREEN BOX %
\declaretheoremstyle[
mdframed={
	backgroundcolor=lightgreen,
	linecolor=darkgreen,
	rightline=false,
	topline=false,
	bottomline=false,
	linewidth=2pt,
	innertopmargin=8pt,
	innerbottommargin=8pt,
	innerleftmargin=8pt,
	leftmargin=-2pt,
	skipbelow=2pt,
	nobreak
},
headfont=\normalfont\bfseries\color{darkgreen}
]{greenbox}

% YELLOW BOX %
\declaretheoremstyle[
mdframed={
	backgroundcolor=lightyellow,
	linecolor=darkyellow,
	rightline=false,
	topline=false,
	bottomline=false,
	linewidth=2pt,
	innertopmargin=8pt,
	innerbottommargin=8pt,
	innerleftmargin=8pt,
	leftmargin=-2pt,
	skipbelow=2pt,
	nobreak
},
headfont=\normalfont\bfseries\color{darkyellow}
]{yellowbox}

% BLUE BOX %
\declaretheoremstyle[
mdframed={
	backgroundcolor=lightblue,
	linecolor=darkblue,
	rightline=false,
	topline=false,
	bottomline=false,
	linewidth=2pt,
	innertopmargin=8pt,
	innerbottommargin=8pt,
	innerleftmargin=8pt,
	leftmargin=-2pt,
	skipbelow=2pt,
	nobreak
},
headfont=\normalfont\bfseries\color{darkblue}
]{bluebox}

% RED BOX %
\declaretheoremstyle[
mdframed={
	backgroundcolor=lightred,
	linecolor=darkred,
	rightline=false,
	topline=false,
	bottomline=false,
	linewidth=2pt,
	innertopmargin=8pt,
	innerbottommargin=8pt,
	innerleftmargin=8pt,
	leftmargin=-2pt,
	skipbelow=2pt,
	nobreak
},
headfont=\normalfont\bfseries\color{darkred}
]{redbox}
%%%%%

%%%%% ENVIRONMENTS
% PURPLE BOXES (theorems, propositions, lemmas, and corollaries) %
\declaretheorem[style=purplebox,name=Theorem,within=section]{thm}
\declaretheorem[style=purplebox,name=Theorem,sibling=thm]{theorem}
\declaretheorem[style=purplebox,name=Theorem,numbered=no]{thm*, theorem*}
\declaretheorem[style=purplebox,name=Proposition,sibling=thm]{prop, proposition}
\declaretheorem[style=purplebox,name=Proposition,numbered=no]{prop*, proposition*}
\declaretheorem[style=purplebox,name=Lemma,sibling=thm]{lem, lemma}
\declaretheorem[style=purplebox,name=Lemma,numbered=no]{lem*, lemma*}
\declaretheorem[style=purplebox,name=Corollary,sibling=thm]{cor, corollary}
\declaretheorem[style=purplebox,name=Corollary,numbered=no]{cor*, corollary*}

% GREEN BOXES (definitions) %
\declaretheorem[style=greenbox,name=Definition,sibling=thm]{definition, defn}
\declaretheorem[style=greenbox,name=Definition,numbered=no]{definition*, defn*}

% BLUE BOXES (problems) %
\declaretheorem[style=bluebox,name=Problem,numberwithin=section]{homework, hw}
\declaretheorem[style=bluebox,name=Problem,numbered=no]{homework*, hw*}

% RED BOXES %
\declaretheorem[style=redbox,name=Remark,sibling=thm]{remark, rmk}
\declaretheorem[style=redbox,name=Remark, numbered=no]{remark*, rmk*}
\declaretheorem[style=yellowbox,name=Warning,sibling=thm]{warn}
\declaretheorem[style=yellowbox,name=Warning,numbered=no]{warn*}
%%%%%

%%%%% PROOF FORMATTING
\renewcommand\qedsymbol{$\blacksquare$}
%%%%%

%%% CUSTOM COMMANDS
% basic %
\newcommand{\Mod}[1]{\ (\mathrm{mod}\ #1)}
\newcommand{\floor}[1]{\left\lfloor{#1}\right\rfloor}
\newcommand{\ceil}[1]{\left\lceil{#1}\right\rceil}
\newcommand{\norm}[1]{\left\lVert{#1}\right\rVert}

% logic %
\newcommand*\xor{\oplus}
\newcommand{\all}{\forall}
\newcommand{\bland}{\bigwedge}
\newcommand{\blor}{\bigvee}
\newcommand*{\defeq}{\mathrel{\rlap{\raisebox{0.3ex}{$\m@th\cdot$}}\raisebox{-0.3ex}{$\m@th\cdot$}}=} \makeatother

% matrices %
\newcommand\aug{\fboxsep=- \fboxrule\!\!\!\fbox{\strut}\!\!\!}\makeatletter 

% sets %
\newcommand{\CC}{\mathbb{C}}
\newcommand{\NN}{\mathbb{N}}
\newcommand{\QQ}{\mathbb{Q}}
\newcommand{\RR}{\mathbb{R}}
\newcommand{\ZZ}{\mathbb{Z}}

% title %
\newcommand{\mytitle}[2]{%
	\title{#1}
	\author{Michael Pham}
	\date{#2}
	\maketitle
	\newpage
	\tableofcontents
	\newpage
}
%%%

%%% REDEFINING COMMANDS
\let\oldint\int
\renewcommand{\int}[2]{\oldint\limits_{#1}^{#2}}
\let\oldprod\prod
\renewcommand{\prod}[2]{\oldprod\limits_{#1}^{#2}}
\let\oldsum\sum
\renewcommand{\sum}[2]{\oldsum\limits_{#1}^{#2}}
%%%
%%%%%
%END_FOLD
%%%%%

\begin{document}
\mytitle{CS70 Homework 6}{Spring 2023}

\section{Unions and Intersections}
Given:
\begin{itemize}
	\item $X$ is a countable, non-empty set. For all $i \in X$, $A_i$ is an uncountable set.
	\item $Y$ is an uncountable set. For all $i \in Y$, $B_i$ is a countable set.
\end{itemize}

For each of the following, decide if the expression is
"Always countable", "Always uncountable", "Sometimes countable,
Sometimes uncountable".

For the "Always" cases, prove your claim. For the "Sometimes" case, provide
two examples -- one where the expression is countable, and one where
the expression is uncountable.

\begin{hw}
	$X \cap Y$
\end{hw}
\begin{solution}
	$X \cap Y$ is always countable. 
	
	We know that $X$ is a countable, non-empty set. Furthermore, we observe that $X \cap Y \subset X$. From here, it follows that $\lvert X \cap Y \rvert \subset X$; in other words, $X \cap Y$ is always countable.
\end{solution}

\begin{hw}
	$X \cup Y$
\end{hw}
\begin{solution}
	$X \cup Y$ is always uncountable. We observe that as $Y$ is an uncountable set, and $Y \subset X \cup Y$, it follows then that $\lvert Y \rvert \leq \lvert X \cup Y \rvert$.
	
	Therefore, we see that $X \cup Y$ must always be uncountable.
\end{solution}

\begin{hw}
	$\bigcup_{i \in X} A_i$
\end{hw}
\begin{solution}
	$\bigcup_{i \in X} A_{i}$ is always uncountable.
	
	We see that as $A_{i} \subset \bigcup_{i \in X} A_{i}$, then $\lvert A_{i} \rvert \leq \lvert \bigcup_{i \in Y} A_i \rvert$. Now, as $A_{i}$ is an uncountable set, it follows that $\bigcup_{i \in Y} A_{i}$ must be uncountable as well.
\end{solution}

\begin{hw}
	$\bigcap_{i \in X} A_i$
\end{hw}
\begin{solution}
	$\bigcap_{i \in X} A_i$ is sometimes countable, sometimes uncountable.
	
	\textbf{\underline{Example of Countable}}: Let $X = \left\{  0, 1\right\}$. Since $\lvert X \rvert = 2$, we see that it's countable.
	
	Now, let $A_{i} = \left\{ i \leq x \leq i + 1 : x \in \RR \right\}$, which is an uncountable set.
	
	We observe then that $\bigcap_{i \in X} A_i = \left\{  1 \right\}$, which is a countable set. Thus, $\bigcap_{i \in X} A_i$ is countable.
	
	\textbf{\underline{Example of Uncountable}}: Like the previous example, let $X = \left\{  0, 1\right\}$, which is countable.
	
	Now, define $A_{i} = \RR$, which is an uncountable set.
	
	We then see that $\bigcap_{i \in X} A_i = \RR$, which is an uncountable set/ Therefore, $\bigcap_{i \in X} A_i$ is uncountable.
	
	Thus, we can conclude that $\bigcap_{i \in X} A_i$ is sometimes countable, sometimes uncountable.
\end{solution}

\begin{hw}
	$\bigcup_{i \in Y} B_i$
\end{hw}
\begin{solution}
	$\bigcup_{i \in Y} B_{i}$ is sometimes countable, sometimes uncountable.
	
	\textbf{\underline{Example of Countable}}: First, an example of where it's countable: suppose that we have the set $\mathscr{P}(\mathbb{N})$. We know that $\lvert \NN \rvert = \aleph_0$, and we also know that $\mathscr{P}(\NN) = 2^{\aleph_{0}}$. In other words, $\mathscr{P}(\mathbb{N})$ is uncountable.
	
	Now, let us define $Y = I$, where $I$ is the index set of $\mathscr{P}(\NN)$. Then, let $B_{i}$ be the $i^{th}$ subset of $\mathscr{P}(\NN)$, for $i \in Y$.
	
	From here, we observe that $\bigcup_{i \in Y} B_{i}$ is simply $\mathbb{N}$, which is countable.
	
	Therefore, we can conclude that $\bigcup_{i \in Y} B_{i}$ can be countable.
	
	\textbf{\underline{Example of Uncountable}}: Next, we consider a case in which $\bigcup_{i \in Y} B_{i}$ is uncountable.
	
	Suppose we have the set $Y = \left\{0 \leq y \leq 1 : y \in \RR \right\}$. We observe that $Y$ is an uncountable set. Now, we let $B_{i} = \left\{ n + i : n \in \NN \right\}$.
	
	Then, it follows that $\bigcup_{i \in Y} B_{i} = \RR$, which we know is an uncountable set.
	
	Therefore, we can conclude that $\bigcup_{i \in Y} B_{i}$ can sometimes be countable, or it can be uncountable.
\end{solution}

\begin{hw}
	$\bigcap_{i \in Y} B_i$
\end{hw}
\begin{solution}
%	$\bigcap_{i \in Y} B_i$ is sometimes countable, sometimes uncountable.
%	
%	\textbf{\underline{Example of Countable}}: Like before, let us define $Y = I$, where $I$ is the index set of $\mathscr{P}(\NN)$. Then, let $B_{i}$ be the $i^{th}$ subset of $\mathscr{P}(\NN)$, for $i \in Y$.
%	
%	Now, we observe that $\bigcap_{i \in Y} B_i = \emptyset$, whose cardinality is 0; in other words, it's a countable set.

	$\bigcap_{i \in Y} B_{i}$ is always countable.
	
	We observe that because $B_{i}$ is a countable set, and $\bigcap_{i \in Y} B_{i} \subset B_{i}$, then it must follow that $\lvert \bigcap_{i \in Y} B_{i} \rvert \leq \lvert B_{i} \rvert$. In other words, we have that $\bigcap_{i \in Y} B_{i}$ is a countable set.
\end{solution}

\newpage

\section{Finite and Infinite Graphs}
The graph material that we learned in lecture still applies if the set of vertices of a graph is infinite. We thus make a distinction between finite and infinite graphs: a graph $G= (V,E)$ is finite if $V$ and $E$ are both finite. Otherwise, the graph is infinite. As examples, consider the infinite graphs 
\begin{itemize}
	\item $G_1 = (V=\mathbb{N},\; E=\{(i,j)\in \mathbb{N}\times \mathbb{N} \mid |i-j|=1\})$
	\item $G_2 = (V=\mathbb{Z},\; E=\{(i,j)\in \mathbb{Z}\times \mathbb{Z} \mid |i-j|=1\})$
	\item $G_3 = (V=\mathbb{Z},\; E=\{(i,j)\in \mathbb{Z}\times \mathbb{Z} \mid i<j\})$
	\item $G_4 = (V=\mathbb{Z}^2,\; E=\{((i,j),(k,l))\in \mathbb{Z}^2\times \mathbb{Z}^2 \mid (i=k \land |j-l|=1) \lor (j=l \land |i-k|=1)\})$
\end{itemize}
Observe that $G_1$ is a line of natural numbers, $G_2$ is a line of integers, $G_3$ is a complete graph over all integers, and $G_4$ is a grid of integers.

Prove whether the following sets of graphs are countable or uncountable.

\begin{hw}
	The set of all finite graphs $G = (V, E)$, for $V \subseteq \mathbb{N}$
\end{hw}
\begin{solution}
	The set of all finite graphs $G = (V, E)$ for $V \subseteq \NN$ is countable.
	
	To show this, we observe that for $v \in V : v = 1$, there's only one graph. For $v \geq 2$, we observe that there are $2^{\binom{v}{2}}$ graphs. It then follows that for each $v \in V$, there is a finite number of graphs. With this in mind, we observe that for each $G_{v}$, we can map it to an $n \in N$, meaning that there is a bijection between the two sets; thus, we can conclude that $G$ is countable.
\end{solution}

\begin{hw}
	The set of all graphs over a fixed, countably infinite set of vertices, where the degree of each vertex is exactly two. For instance, every vertex in $G_2$ (defined above) has degree $2$.
\end{hw}
\begin{solution}
	content...
\end{solution}

\begin{hw}
	We say that graphs $G = (V, E)$ and $G' = (V', E')$ are isomorphic if
	the exists some bijection $f:V\rightarrow V'$ such that $(u,v)\in E$ iff
	$(f(u),f(v))\in E'$. Such a bijection $f$ is called a \emph{graph isomorphism}.
	Suppose we consider two graphs to be the equivalent if they are isomorphic. The
	idea is that if we relabel the vertices of a graph, it is still the same graph.
	
	Using this definition of ``being the same graph'', prove that the set of trees over countably infinite vertices is uncountable.
\end{hw}
\begin{solution}
	The set of trees over countably infinite vertices $T$ is uncountable.
	
	We will be showing this by mapping a subset of the set of trees over countably infinite vertices $T'$ to the set of infinite bitstrings $S$, then showing that there is an injection between $S$ and $T'$. And since this is the case, we know then that $T'$ (and thus $T$) is uncountable.
	
	To begin with, let us consider the graph $G_{1} = (V=\mathbb{N},\; E=\{(i,j)\in \mathbb{N}\times \mathbb{N} \mid |i-j|=1\})$. For convenience, let us label the degree 1 node $v_{0}$, the node adjacent to it $v_{1}$, the unlabeled node adjacent to $v_{1}$ as $v_{2}$, and so on. Then, the $i^{th}$ node will be $v_{i}$:
	
	Also, observe that the graph above is a tree; any disconnecting of edges will result in there being two connected components.
	
	Now, from here, we can map an infinite bitstring $s \in S$ by doing the following: if the $i^{th}$ bit $s_{i}$ is a 1, we can attach a node to $v_{i}$ on our graph. Below is an example of this:
	
	Notice here that the set of trees formed by doing this, $T'$ is only a subset of $T$. Then, if we show that there is an injection between $S$ (and uncountable set) and $T'$, it follows that $T'$ is uncountable, and thus $T$ is uncountable as well.
	
	Now, we want to show that there is an injection between $S$ and $T'$. To do this, we want to show that if two bitstrings $s, s' \in S : s \not= s'$, then it follows that $f(s) \not= f(s')$ (where $f(s)$ and $f(s')$ are the trees generated by the mapping we detailed above).
	
	Notice that if $s \not= s'$, then it means that they must differ by at least one bit. Then, at $i^{th}$ bit where $s$ and $s'$ differ, we see that $\mathrm{deg}(v_{i}) \not= \mathrm{deg}(v'_{i})$ (where $v$ is the vertex of $f(s)$ and $v'$ is the vertex of $f'(s)$); thus, we see that $f(s) \not= f'(s)$. Thus, there exists an injection between $S$ and $T'$.
	
	Then, since there is an injection between the set $S$ and $T'$, it means that $T'$ must be uncountable as well; thus, it follows that the set of trees over countably infinite vertices is also uncountable
%	Suppose then that $s$ and $s'$ differ by one bit at the $i^{th}$ bit; $s_{i} = 0$ and $s'_{i} = 1$.
%	
%	We observe then that $f(s) \not= f(s')$ because the degree at vertex $f(s)_{i} = 2$, while $f(s')_{i} = 3$; as such, they are different graphs.
\end{solution}

\section{Countability Proof Practice}
\begin{hw}
	A disk is a 2D region of the form $\{(x, y) \in \R^2 : (x - x_0)^2 + (y - y_0)^2 \le r^2\}$, for some $x_0, y_0, r \in \R$, $r > 0$.
	Say you have a set of disks in $\R^2$ such that none of the disks overlap.
	Is this set always countable, or potentially uncountable?
\end{hw}
\begin{solution}
	The set of disks in $R^{2}$ such that there are no overlapping disks, which we will denote by $D$, is always countable.
	
	To see this, let us pick an arbitrary point $(x,y)$ in each disc such that $x, y \in \QQ$. Now from here, we observe that since there the disks don't overlap with one another, it follows then that we can map each point chosen in disc $D_{n}$ to a pair $\QQ \times \QQ$.
	
	Since there is an injection between the coordinates of each disc $D_{n}$ to $\QQ \times \QQ$, and we know that $\QQ \times \QQ$ is a countable set, it follows then that $D$ must also countable.
\end{solution}

\begin{hw}
	A circle is a subset of the plane of the form $\{(x, y) \in \R^2 : (x - x_0)^2 + (y - y_0)^2 = r^2\}$ for some $x_0, y_0, r \in \R$, $r > 0$. Now say you have a set of circles in $\R^2$ such that none of the circles overlap. Is this set always countable, or potentially uncountable?
\end{hw}
\begin{solution}
	This set is potentially uncountable.
	
	We observe that we can construct a set of circles where the set of its radius $R = \RR$ and there will still be no overlap. Since we observe that this can be done, we see that the set can be uncountable.
\end{solution}

\begin{hw}
	Is the set containing all increasing functions $f : \N \rightarrow \N$ (i.e., if $x \geq y $, then $f(x) \geq f(y)$) countable or uncountable? Prove your answer.
\end{hw}
\begin{solution}
	The set containing all increasing functions is uncountable.
\end{solution}

\begin{solution}
	Is the set containing all decreasing functions $f : \N \rightarrow \N$ (i.e., if $x \geq y$, then $f(x) \leq f(y))$ countable or uncountable? Prove your answer. We observe that
\end{solution}
\begin{solution}
	The set containing all decreasing functions in countable. We observe that 
\end{solution}

\newpage

\section{Fixed Points}
Consider the problem of determining if a program $P$ has any fixed points. Given any program $P$, a fixed point is an input $x$ such that $P(x)$ outputs $x$.  

\begin{hw}
	Prove that the problem of determining whether a program has a fixed point is uncomputable.
\end{hw}


\end{document}